\section{Uniform convergence}
\subsection{Limiting value of functions}
\declareexercise{ii.3.1.1}
\begin{description}
	\item[If] Suppose that $\lim_{x\to x_0;x\in E\setminus\{x_0\}} f(x)$ exists and equals to $f(x_0)$. This means that, for every $\varepsilon > 0$, there exists $\delta > 0$ such that $x \in \{x \in E\setminus\{x_0\} : d_X(x,x_0) < \delta\} \to d_Y(f(x),f(x_0)) < \varepsilon$.
	
	However, considering that $d_Y(f(x_0),f(x_0)) = 0 < \varepsilon$ for every $\varepsilon > 0$, we may conclude that $x \in \{x \in E : d_X(x,x_0) < \delta\} \to d_Y(f(x),f(x_0)) < \varepsilon$. By definition, this means that $\lim_{x\to x_0;x\in E} f(x) = f(x_0)$.
	
	\item[Only If] Suppose that $\lim_{x\to x_0;x\in E} f(x)$ exists. Say it equals to $L$. Then, for every $\varepsilon > 0$, there exists $\delta > 0$ such that $x \in \{x \in E : d_X(x,x_0) < \delta\} \to d_Y(f(x),L) < \varepsilon$. 
	
	In particular, since $d_X(x_0,x_0) = 0$, which is smaller than any $\delta > 0$, I conclude that $d_Y(f(x_0), L)$ is smaller than any $\varepsilon > 0$, which means it equals to $0$.
	
	Given any $\delta > 0$ and any $x \in \{x \in E\setminus\{x_0\} : d_X(x,x_0) < \delta\}$, consider the triangular in equality:
	$$
	d_Y(f(x),f(x_0)) \le d_Y(f(x),L) + d_Y(L, f(x_0)) = d_Y(f(x),L)
	$$
	Hence, if $d_Y(f(x),L) < \varepsilon$, then $d_Y(f(x),f(x_0)) < \varepsilon$. Because of the existence of $\lim_{x\to x_0;x\in E} f(x)$, we can always find $\delta > 0$ to get some $x$ to satisfy the $\varepsilon$. It means that $\lim_{x\to x_0;x\in E\setminus\{x_0\}} f(x) = f(x_0)$, as desired.
\end{description}

Note that $f(x_0) = L$ immediately follows from $d_Y(f(x_0),L) = 0$. Then, I have finished all the proofs.

One might be eager to arrive at the conclusion that the two definitions of limit are the same, as we are able to say now
\begin{quotation}
	$\lim_{x\to x_0;x\in E\setminus\{x_0\}} = f(x_0)$ if and only if $\lim_{x\to x_0;x\in E} = f(x_0)$.
\end{quotation}
However, it is not exactly the case, because when saying $\lim \cdots = L$, we are saying two things:
\begin{enumerate}
	\item $\lim \cdots$ exists.
	\item It equals to $L$.
\end{enumerate}
Hence, when $\lim_{x\to x_0;x\in E\setminus\{x_0\}} = L \ne f(x_0)$, $\lim_{x\to x_0;x\in E}$ does NOT exist!