% Copyright (C) He Guanyuming 2020
% The file is licensed under the MIT license.

\section*{General Principles}
This section describes the overall principles of the document: in what style and format it is typeset,
in what structure this document is written, and so forth.

\subsection*{Definitions}
\begin{description}
\item[The document] The phrase \emph{the document} means this document (what you are reading) itself.
\item[The book] The phrase \emph{the book} represents Tao's \emph{Analysis} (both volume I and II).
\end{description}

\subsection*{Indices}
The book has two volumes: 
\emph{Analysis I} and \emph{Analysis II}. We may notice that the indices of the two volumes both start 
from 1. It may lead to some confusions. So in the document, the indices are organized in such a way that:
If the content comes from \emph{Analysis I}, the corresponding index is the same as the book's. 
Otherwise, the corresponding index is prefixed with ``2.''.

For example, Exercise 3.1.3 in \emph{Analysis I} is indexed as Exercise 3.1.3 in the document, but 
Exercise 3.1.3 in \emph{Analysis II} is indexed as Exercise 2.3.1.3.

\subsection*{Notations}
Professor Tao uses many kinds of statements: Theorems, Propositions, Lemmas, etc. However, in this document, I will use them, too. Therefore, to distinguish between his and my statements,
I will prepend all my statements with ``My''. For example, Proposition 2.2.1 in this document will be written as My Proposition 2.2.1 in this document.

Logical statements are often involved throughout this book, especially heavily in Chapter 3, Set Theory. Therefore, I will use some logical symbols that Prof.~Tao didn't mention or uses other notations.
\begin{itemize}
	\item I will use $\forall$ for ``for all'', and $\exists$ for ``there exists.''
	\item I will use $\to$ to mean implies, while Prof.~Tao uses $\Rightarrow$ for it.
	\item I will use $\equiv$ to mean two logical statements are the same. That is, if I say $p \equiv q$, then $q$ is true whenever $p$ is true, and vice versa.
	\item I don't know yet, but perhaps sometimes it is more appropriate to say $p \dashv\vdash q$, if I want to express that, from $p$ we can have a proof for $q$, and vice versa. But it really makes no difference, as far as truth is concerned, with saying that $p \equiv q$, thanks to the soundness and completeness of propositional and first-order predicate logic, which are the only logical systems involved in this book.
\end{itemize}

Although in logical contexts, people often refer to something that can hold a truth value ($\true$ or $\false$) as a formula, I will use the same name, ``statement'', as Prof.~Tao, to avoid confusion.

\subsection*{Abbreviations}
We often leave off some descriptions for a number's properties. For example, we may refer a 
\emph{positive natural number} $n$ as a \emph{positive number} when we haven't learned the rationals and 
the reals.