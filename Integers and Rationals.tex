% Copyright (C) He Guanyuming 2020
% The file is licensed under the MIT license.

\section{Integers and Rationals}
Now we are going to extend natural numbers to integers and rationals.

\subsection{The Integers}

\declareexercise{4.1.1}
\begin{proof}
It is immediately given by the fact that 
\[
a+b = a+b \equiv a -- b = a -- b
\]
\end{proof}

\paragraph{Lemma 4.1.3}
\[
(m--0)+(n--0) = (m+n)--0
\]
\[
(m--0) \times (n--0) = (mn) -- 0
\]
ensures that the definition $m--0:=m$ is consistent with addition and multiplication.

\declareexercise{4.1.2}
\begin{proof}
\[
a--b = a'--b' \equiv a=b \wedge a'=b'
\]
Then, 
\[
(b--a) = (b'--a') \equiv -(a--b) = -(a'--b')
\]
\end{proof}

\declareexercise{4.1.3}
\begin{proof}
\begin{align*}
-1 \times a 
&= (0 -- 1) \times (a -- 0) \\
&= (0\times a + 1 \times 0) -- (0 \times 0 + 1 \times a) \\
&= 0 -- a \\
&= -a
\end{align*}
\end{proof}

\declareexercise{4.1.4}
\begin{proof}
Let $x=(a--b),y=(c--d),z=(e--f)$.

(1)
\begin{align*}
(a--b) + (c--d) 
&= (a+c) -- (b+d) \\
&= (c+a) -- (d+b) \\
&= (c--d) + (a--b)
\end{align*}

(2)
\begin{align*}
((a--b) + (c--d)) + (e--f)
&= ((a+c)+e) -- ((b+d)+f) \\
&= (a+(c+e)) -- (b+(d+f)) \\
&= (a--b) + ((c--d) + (e--f))
\end{align*}

(3)
First ,
\[
(a--b) + (0--0) = (a--b)
\].

Second, by (1) we have $0+x=x+0$.

(4)
First, 
\begin{align*}
(a--b) + (b--a) 
&= (a+b) -- (a+b) \\
&= 0 -- 0 \tag{$a+b+0=a+b+0$}
\end{align*}

Second, by (1) we have $x+(-x) = (-x) + x$.

(5)
\begin{align*}
(a--b)(c--d)
&= (ac + bd) -- (ad + bc) \\
&= (ca + db) -- (cb + da) \\
&= (c--d)(a--b)
\end{align*}

(6)
The book proved this.

(7)
First,
\[
(1--0)(a--b) = (1a + 0b) -- (1b+0a) = (a--b)
\]

Second, by (5) we have $1x=x1$.

(8)
\begin{align*}
&(a--b)((c--d)+(e--f)) \\
&= (a--b)((c+e)--(d+f)) \\
&= (a(c+e) + b(d+f)) -- (a(d+f) + b(c+e)) \\
&= ((ac + bd)+(ae + bf)) -- ((ad + bc)+(af + be)) \\
&= (ac+bd)--(ad+bc) + (ae+bf)--(af+be) \\
&= (a--b)(c--d) + (a--b)(e--f)
\end{align*}

(9)
This can be easily concluded from (5) and (8).
\end{proof}

\declareexercise{4.1.5}
\begin{proof}
We need to show that 
\[
a \neq 0 \wedge b \neq 0 \Longrightarrow ab \neq 0
\]

Since $a,b$ are not 0, they can be either positive or negative. If they are both positive, the case is 
already proven.

When at least one of them is negative, we can divide the $-1$ from the negative ones. That is, if $a=-m$, 
where $m$ is positive, then we substitute $a$ with $-1 \times m$. Then we may get $ab$ in either the form 
$(-1)(-1) mn$ or $(-1) mn$, where the former is a positive number because $(-1)(-1) =1$ and the latter is 
negative.
\end{proof}

\declareexercise{4.1.6}
\begin{proof}
We check the value of $ac-bc$. We know that $ac=bc$, so $ac - bc = 0 - 0 = 0$. According to (9) in 
Proposition 4.1.6, 
\[
ac - bc = ac+(-b)c = (a+(-b))c = 0
\]

As stated by Proposition 4.1.8, since that $c \neq 0$, $a+(-b) = 0$, which means $a-b=0$. Then we have 
$a=b$.
\end{proof}

\declareexercise{4.1.7}
In the following contents, $p$ stands for a positive natural number, $n$ stands for a natural number.

\begin{proof}
(a)
\begin{align*}
a>b 
&\equiv a = b+p \\
&\equiv a+(-b) = b + (-b) + p  \tag{See the following explanation} \\
&\equiv a-b = p
\end{align*}
We now explain why $a = b+p \equiv a+(-b) = b + (-b) + p$. Using the substitution law and the 
commutativity of addition, it is clear to see that $a = b+p \Longrightarrow a+(-b) = b + (-b) + p$. We now 
show the cancellation law of addition, that is,
\begin{lem}
\[
a+c = b+c \Longrightarrow a = b
\]
\end{lem}
\begin{proof}
\begin{align*}
a+c=b+c
&\Longrightarrow a+c+(-c) = b+c+(-c) \\
&\Longrightarrow a+(c+(-c)) = b + (c+(-c)) \\
&\Longrightarrow a=b
\end{align*}
\end{proof}

So we get the inverse result: $a = b+p \Longleftarrow a+(-b) = b + (-b) + p$.

Note that by the definition of integer and what we have know now, we can conclude that 
\begin{lem}
For every integer 
$i = a - b, j = c - d$, there exists exactly one integer $k$ such that $i = j+k$.
\end{lem}

(b)
\begin{align*}
a>b
&\equiv a = b + p \\
&\Longrightarrow a+c = b+c+p \\
&\Longrightarrow a+c>b+c
\end{align*}

(c)
\begin{align*}
a>b
&\equiv a=b+p \\
&\Longrightarrow ac = (b+p)c = bc + pc \\
&\Longrightarrow ac > bc \tag{$pc > 0$ by Lemma 2.3.3}
\end{align*}

(d)
\[
a>b \equiv a = b+p
\]
Then
\[
-a = -(b+p) = (-1)(b+p) = -b - p
\]
So
\[
-a+p=-b-p+p
\]
That is,
\[
-b=-a+p \equiv -b>-a
\]

(e)
Let
\[
a = b+p_1,b=c+p_2
\]
Then $a = c+(p_1+p_2)$. Obviously $p_1+p_2$ is positive, so $a>c$.

Note that $-a,-b$ are also integers, and plus that $-(-a)=a$, so we can give a stronger conclusion:
\[
a>b \equiv -a<-b
\]

(f)
If $a,b$ are all natural numbers, the statement was proven before. 

If one of them (say $a$) is negative, 
the other ($b$) is a natural number, then $a=-n$, and we know that $b>0$ and 
$0 = a+n \Longrightarrow 0 >-a$, so by (e) we have $b>a$.

If they are both negative, then their negations satisfy the statement. Then
\[
-a<-b \equiv a>b, -a=-b \equiv a=b, -a>-b \equiv a<b
\].
\end{proof}

\declareexercise{4.1.8}
An example: $P(i): i>=0$. It is obvious that $P(0)$ and $P(i) \Longrightarrow P(i+1)$ is true. But for any  
negative integer $n$, $P(n)$ is not true.

We additionally prove one more property:
\begin{lem}
For integers $a,b$,
\[
a-b=0\Longrightarrow a=b
\]
\end{lem}
\begin{proof}
We can add $b$ to both side to obtain $a=b$.
\end{proof}

\subsection{The Rationals}
\declareexercise{4.2.1}
\begin{proof}
Reflectivity:
\[
a//b = a//b \equiv ab=ab
\] 

Being Symmetric:
\begin{align*}
a//b = c//d 
&\equiv ad = bc \\
&\equiv cb = da \\
&\equiv c//d = a//d
\end{align*}

Transitivity:
\[
a//b = c//d \equiv ad = bc
\]
\[
c//d = e//f \equiv cf = de
\]
Thus,
\[
(ad)(cf) = (bc)(de)
\]
We then have
\[
afcd = becd
\]
We can cancel $d$ since $d \neq 0$ to obtain $afc=bec$. If $a=0$, we can conclude that $c,e$ also must be 
$0$. Under this occasion, $af=be$ is also true because they all equal to $0$. 
\end{proof}

\paragraph{Definition 4.2.2}
It is useful to prove that 
\begin{lem} \label{lem4.2.3}
\[
(-a)//b = a//(-b)
\]
,
\[
a//b = (-a)//(-b)
\]
\end{lem}
\begin{proof}
The first is immediately given since $(-a)(-b) = ab$. The latter is proven as $a(-b) = b(-a) = -ab$.
\end{proof}

We may notice that subtraction is not mentioned here. This is because that we can get $a-b$ by adding $a$ 
and $-b$, where addition $+$ and negation $-$ are mentioned.

\declareexercise{4.2.2}
\begin{proof}
(1) is deduced in the book.

(2)
I don't quiet understand why Tao used this $*$ sign instead of $\times$. I know it is a new definition, 
but the $\times$ sign is undefined for rationals (except for integers, but for which we can verify that 
the two definitions are the same). We will use the $\times $ sign or just leave it off here.

\[
(a'//b')(c//d) \equiv a'd = b'c \equiv ad=bc \equiv (a//b)(c//d)
\]
Similarly we can verify this for $c'//d'$.

(3)
\[
-ab'=-a'b \equiv (-a)//b = (-a')//b'
\]
\end{proof}

For the sake of simplification, we hereby introduce some useful lemmas:
\begin{lem} \label{lem4.2.1}
\[
b=d\neq 0 \Longrightarrow (a//b = c//d \equiv a=c)
\]
\end{lem}
\begin{proof}
Assume that $b=d\neq 0$.

On one hand, if $a//b=c//d$, then $ad=bc$. Since that $b=d \neq 0$, we can cancel them to obtain $a=c$.

On the other hand, if $a=c$, then if we multiply them by the same integer (namely $b=d$), and the 
results are still equal ($ad=bc$). So $a//b=c//d$.
\end{proof}
\begin{lem} \label{lem4.2.2}
\[
c \neq 0 \Longrightarrow a//b=ac//bc
\]
\end{lem}
\begin{proof}
Assume that $c\neq 0$.

First we know that $ab=ab$. Then we can further obtain $abc=abc$, which 
means $a//b=ac//bc$.
\end{proof}

\declareexercise{4.2.3}
\begin{proof}
(1)
We have
\[
a//b + c//d = (ad+bc)//(bd)
\]
\[
c//d + a//b = (cb+da)//(db)
\]
It is easy to see that they are equal.

(2)
It is proven in the book.

(3)
We just deduce $x+0=x$ here, for we have $0+x=x+0$ according to (1).
\[
a//b + 0//1 = (a1+b0)//(b1) = a//b
\]

(4)
We only prove $x+(-x)=0$ here, for we have $x+(-x)=(-x)+x$ according to (1).
\[
a//b+(-a)//b= (ab-ab)//bb = 0//b^2=0
\]

(5)
\begin{align*}
a//b \times c//d
&= ac//bd \\
&= ca//db \\
&= c//d \times a//b
\end{align*}

(6)
\begin{align*}
&(a//b \times c//d) \times e//f \\
&= ac//bd \times e//f \\
&= ace//bdf \\
&= a//b \times ce//df \\
&= a//b \times (c//d \times e//f)
\end{align*}

(7)
We only prove $x1=x$ here, for we have $x1=1x$ according to (4).
\[
a//b \times 1//1 = a1//b1 = a//b
\]

(8)
\begin{align*}
&a//b (c//d + e//f) \\
&= a//b ((cf+ed) // (df)) \\
&= a(cf+ed)//bdf \\
&= ab(cf+ed) // b^2df \tag{See Lemma \ref{lem4.2.2}} \\
&= ((ac)(bf) + (bd)(ae)) // (bd)(bf) \\
&= ac//bd + ae//bf \\
&= (a//b \times c//d) + (a//b + e//f)
\end{align*}

(9)
This can be deduced from (5) and (8).

(10)
We merely conclude $xx^{-1} = 1$ here, since we have $xx^{-1} = x^{-1}x$ from (5).

\[
a//b \times b//a  = ab//ba = (ab)1//(ab)1 = 1//1
\]
The last step is done by Lemma \ref{lem4.2.2}.
\end{proof}

\declareexercise{4.2.4}
\begin{proof}
For any rational $r = a/b$, $a,b$ are integers.
They are either positive, $0$, or negative (except that $b$ cannot be 0). When $a,b$ are both positive, 
then $r$ is also positive. When $a$ is positive but $b$ is negative, then let $b=-p$, where $p$ is 
positive, thus $a/b = a/(-p) = (-a)/p$ is negative. When $a=0$, $r=0$. When $a$ is negative, and $b$ is 
positive, then by definition $r$ is negative. When $a,b$ are both negative, according to Lemma 
\ref{lem4.2.3}, $r$ is positive.

Therefore, we have iterated through all possible situations and verified that there is and only is one 
statement for a rational is true.
\end{proof}

\declareexercise{4.2.5}
\begin{proof}
Let $x=a/b,y=c/d,z=e/f$. Before proving the following components, we will introduce some useful 
propositions here.
\begin{lem} \label{lem4.2.4}
\begin{enumerate}
\item $x>0$ is logically equivalent to $x$ being positive.
\item $x<0$ is logically equivalent to $x$ being negative.
\end{enumerate}
\end{lem}
\begin{proof}
\[
x-0 =x
\]
is itself, so whether $x$ is positive or negative, the same is $x-0$, then we can deduce $x>0$ or $x<0$, 
and vice versa.
\end{proof}

We can now use simplified notation $x>0$ to express the same meaning: $x$ is positive. 

(a)
We check the value of 
\[
\delta = x-y = a/b +(-c)/d = (ad-bc)/bd
\]
$\delta$ is also a rational number. According to the previous exercise, it is either positive, negative, 
or $0$. So $x$ either $>y$, $<y$, or $=y$ (We haven't yet proven $x-y=0 \Longrightarrow x = y$. Let's 
prove it now. We can add $y$ to both side of $x-y=0$ to obtain the result).

(b)
According to Lemma \ref{lem4.2.4}, $x<y \Longrightarrow x-y<0$. Then we multiply $-1/1$ with $x-y$ to 
obtain (It is easy to see that for rational number $r$, $-1r = -r$ and $-(-r)=r$)
\[
-1/1 \times (x + (-y)) = -x + -(-y) = y - x
\]
Since $x-y$ is negative, and the negation of a positive number is negative, so the negation of $x-y$, 
$y-x$, is positive, which means that $y>x$.

(c)
By the hypothesis, $x-y<0 \wedge y-z<0$. We are now proving that $i,j<0 \Longrightarrow i + j <0$. We can 
write $i,j$ as $o/p,q/s$ respectively. Let $p,s>0$, then $o,q<0$. Then $o/p+q/s = (os+pq)/ps$. We know 
that $os,pq<0$ (Write a negative integer as a negation of a positive integer to see that the product of a 
positive and a negative is also negative). 

Now we show that for two positive integers, their sum is still 
positive. Integers who are positive are also natural number, and their sum remains a natural number. So 
the sum itself equals to $0$ plus itself, which means it is positive. The negation of this sum, which is 
also $-m+(-n)$, is thus negative. Since that $-m,-n$ can present any negative integer, the fact means that 
the sum of two negative integers remains a negative integer.

So $os+pq<0$. But $ps>0$, so $i+j<0$. Thus, $(x-y) + (y-z) = x-z<0$, which means $x<z$.

(d)
\begin{align*}
x+z-(y+z) 
&= x+z + (-)(y+z) \\
&= x+z + (-1)(y+z) \\
&= x+z -z - y \\
&= x-y <0
\end{align*}

(e)
It is easy to verify that the product of two positive rationals is still positive (Writing them as 
$a/b,c/d$, where $a,b,c,d>0$, then $ac/bd$ also $>0$). Then $xz-yz=z(x-y)$, which is the product of a 
positive number and a negative number, and is thus a negative number.
\end{proof}

\declareexercise{4.2.6}
\begin{proof}
According to (e) of Proposition 4.2.9, we need only to show that $x<y \Longrightarrow -x>-y$. Then we can 
multiply $xz>yz$ with $-1$ to obtain what we want.

We know that the negation operation will turn a positive into negative and vice versa. Now we have 
$x-y<0$, so the negation $-(x-y) = -x+y = -x - (-y)>0$, which means that $-x>-y$.
\end{proof}

There are still many properties about rationals that we use for granted (e.g. $x^-1$ has the same sign as 
$x$; the two definitions of order are the same, that is, $x-y>0 \equiv x=y+p \equiv x>y$, where $p>0$). 
Although they need to be proven prior to being used, we can not cover all of them here. We will 
prove some of them in the future only if they are used. Also, most of them are not hard to prove. We need 
not to worry.

We will prove some important ones here:
\begin{prop} \label{prop.different.def.of.order}
For rational numbers $x,y,p>0$
\[
x-y>0 \equiv x=y+p \equiv x>y
\]
\end{prop}
\begin{proof}
$x-y>0 \equiv x>y$ is the definition of order. We merely need to prove $x-y>0 \equiv x=y+p$ here.

On one hand, if $x-y>0$, then $p$ is not others, but the very number $x-y$.

On the other hand, if there exists a $p>0$ such that $x=y+p$. Then add $-y$ to the both side of the 
equation, and we can get $x-y = p$, which means $x-y$ is positive. So $x-y>0$.
\end{proof}

\begin{prop} \label{prop.4.2.add.ineq}
We can add two inequalities together. That is,
\[
a<b \wedge c<d \Longrightarrow a+c<b+d
\]
\end{prop}
\begin{proof}
We know that
\[
a<b \Longrightarrow a+c<b+c 
\]
, and
\[
c<d \Longrightarrow b+c<d+b 
\]
According to the transitivity of order, we can derive that
\[
a+c<b+c<b+d
\]
\end{proof}

\begin{prop} \label{prop.4.2.multiply.ineq}
We can multiply two inequalities of positives or negatives together. That is,
\[
a,b,c,d >0 \wedge a<b \wedge c<d \Longrightarrow ac<bd
\]
, and
\[
a,b,c,d <0 \wedge a<b \wedge c<d \Longrightarrow ac>bd
\]
\end{prop}
\begin{proof}
When they are all positive,
we know that
\[
a<b \Longrightarrow ac<bc
\]
, and
\[
c<d \Longrightarrow bc<bd
\]
According to the transitivity of order, we can derive that
\[
ac<bc<bd
\]

When they are all negative,
we know that
\[
a<b \Longrightarrow ac>bc
\]
, and
\[
c<d \Longrightarrow bc>bd
\]
According to the transitivity of order, we can derive that
\[
ac>bc>bd
\]
\end{proof}

Note that we can already add or multiply equations because of the axiom of substitution, so we can change 
the $<$ in the inequalities to $\leq$ in the previous two propositions whenever needed.

\subsection{Absolute Value and Exponentiation}
\declareexercise{4.3.1}
\begin{proof}
(a)
$x>0 \Rightarrow |x| >0$, $x=0\Rightarrow |x|=0$, $x<0 \Rightarrow |x| >0$. So $|x| \geq 0$.

And we can see that only when $x=0$ can $|x| = 0$.

(b)
This one is very tedious to prove. Let's enumerate all conditions:
\begin{enumerate}
\item $x,y>0$. On this occasion, 
\[
|x+y| = x+y = |x|+|y|
\]

\item At least one of them is $0$. On this occasion, let's just let $x$ be 0, the other 
situations are similar.
\[
|x+y| = |0+y| = |y| = |0| + |y| = |x| + |y|
\]

\item $x=y>0$. On this occasion, $|x+y| = |2x| = 2x = |x|+|x|$.

\item $x=y<0$. On this occasion, $|x+y| = |2x| = -2x = |x|+|x|$.

\item $x,y<0$. On this occasion, $|x+y| = -(x+y) = -x -y = |x| + |y|$.

\item One of them is positive, the other is negative. We specify $x>0,y<0$ here. But the other conditions 
are similar. Under this condition, we further divide the situation into three occasions:
\begin{itemize}
\item $x+y>0$ On this occasion, $|x+y| = x+y$, $|x| + |y| = x-y$. Note that $x-y = x+y +2(-y)$, where 
$2(-y) >0$, so $|x+y| < |x|+|y|$ (See Proposition \ref{prop.different.def.of.order}).
\item $x+y<0$ On this occasion, $|x+y| = -x-y$, $|x|+|y|=x-y$. Note that $-x-y - (x-y) = 2(-x) < 0$, so 
$|x+y| < |x| + |y|$.
\item $x+y=0$ On this occasion, $|x+y| = 0 \leq |x| + |y|$ (Recall that the sum of two positive rationals 
remains positive).
\end{itemize}
\end{enumerate}

We have iterated through all conditions.

(c)
We shall prove that 
\[
-|x| \leq x \leq |x|
\]
first.

\begin{enumerate}
\item If $x>0$, then $x=|x|>0$. And $0>-|x|$, so by the transitivity of order, $-|x| < x$.
\item If $x=0$, then $|x|=-|x| = x=0$.
\item If $x<0$, then $x=-|x|<0$ And $0<|x|$, so $x<|x|$.
\end{enumerate}
This also means that $x$ either equals to $|x|$ or $-|x|$.

Then we prove that $-y \leq x \leq y \equiv y \geq |x|$.

On one hand, if $-y \leq x \leq y$, then when $x=|x|$, we have $|x| \leq y$; when $x=-|x|$, we have 
$-y \leq -|x| \equiv y \geq |x|$. As stated previously, we know that at least one of the two conditions 
are satisfied.

On the other hand, if $y \geq |x|$, then $-y \leq -|x|$. But since that $-|x| \leq x \leq |x|$, we can 
obtain what we want by the transitivity of order.

(d)
\begin{enumerate}
\item If $x=y=0$, then $|xy| = 0 = |x||y|$.
\item If $x,y>0$, then $|xy| = xy = |x||y|$.
\item If $x,y<0$, then $xy>0$, $|xy| = xy = (-x)(-y) = |x||y|$.
\item If one of them is positive, and the other is negative, (say $x>0,y<0$), then 
$|xy|=-xy=x(-y)=|x||y|$. The other conditions are similar.
\end{enumerate}

Thus, $|-x| = |-1||x| = 1|x| = |x|$.

(e)
This can be easily conclude from (a).

(f)
Since that $|-x|=|x|$, we have $|x-y| = |-(x-y)| = |y-x|$.

(g)
Note that $x-z = (x-y) = (y-z)$. Then from (b) we can deduce that $|x-z| \leq |x-y| + |y-z|$, which is 
$d(x,z) \leq d(x,y) + d(y,z)$.
\end{proof}

\declareexercise{4.3.2}
\begin{proof}
(a)
If $x=y$, then $|x-y| = 0$. And any positive rational $\varepsilon >0$, so $|x-y|\leq\varepsilon$.

The other statement is much better easier to prove after we have know the denseness of rationals. We 
essentially repeat some of the proof work that are done afterwards here.
On the other hand, suppose the negation, that is, $(\forall \varepsilon>0)(|x-y| \leq \varepsilon)$, but 
$x\neq y$. Then $x-y \neq 0$. Let $\delta = |x-y| \neq 0$. We know that $2^{-1} = 1/2 > 0$, so 
$\delta / 2 > 0$. Also we have $\delta /2 + \delta /2 = \delta \Longrightarrow \delta /2 < \delta$. Then let 
$\varepsilon = \delta /2$. So we have both $|x-y| < \delta/2$ and $|x-y| > \delta /2$, which is impossible. 

(b)
It is immediately derived from $|x-y| = |y-x|$.

(c)
\[
|x-z| \leq |x-y| + |y-z| \leq \varepsilon + \delta
\]

(d)
\[
|x+z - (y+w)| = |x-y + z-w| \leq |x-y| + |z-w| \leq \varepsilon + \delta
\]
\[
|x-z - (y-w)| = |x-y + w-z)| \leq |x-y| + |w-z| \leq \varepsilon + \delta
\]

(e)
\[
|x-y| \leq \varepsilon < \varepsilon'
\]

(f)
From (c) of Proposition 4.3.3, we can derive that
\[
|x-z| \leq \varepsilon \equiv -\varepsilon \leq x-z \leq \varepsilon 
\equiv z-\varepsilon \leq x \leq z+ \varepsilon
\]
, and that
\[
|x-y| \leq \varepsilon \equiv y-\varepsilon \leq x \leq y + \varepsilon
\]
Thus we have
\[
y-\varepsilon \leq x \leq z + \varepsilon
\]

We will only prove the statement when $z\leq w \leq y$, another one is similar. On this occasion, 
$-y \leq -w \leq -z$. Add this inequality to $y-\varepsilon \leq x \leq z + \varepsilon$ to obtain that 
\[
-\varepsilon \leq x-w \leq \varepsilon
\]

(g)
\[
|xz-yz| = |x-y||z| \leq \varepsilon|z|
\]

(h)
We will explain why $|a| \leq \varepsilon \wedge |b| \leq \delta$ implies 
$|a||z| + |b||x| + |a||b| \leq \varepsilon|z| + \delta|x| + \varepsilon\delta$.

First, multiply $|a| \leq \varepsilon$ with $|z|$ to obtain $|a||z| \leq \varepsilon |z|$. Then add both sides 
of the inequality with $|b||x| + |a||b|$ to gain 
\[
|a||z| + |b||x| + |a||b| \leq \varepsilon |z| + |b||x| + |a||b| \tag{1}
\]
Similarly,
\[
|a||z| + |b||x| + |a||b| \leq |a||z| + \delta |x| + |a||b| \tag{2}
\]
Finally, we can multiply $|a| \leq \varepsilon$ with $|b| \leq \delta$ as stated by Proposition 
\ref{prop.4.2.multiply.ineq} to derive $|a||b| \leq \varepsilon\delta$. So after some addition we have
\[
|a||z| + |b||x| + |a||b| \leq |a||z| + |b||x| + \varepsilon\delta \tag{3}
\]

Using Proposition \ref{prop.4.2.add.ineq}, we add (1),(2) and (3) together:
\[
3(|a||z| + |b||x| + |a||b|) \leq (\varepsilon|z| + \delta|x| + \varepsilon\delta) + 2(|a||z| + |b||x| + |a||b|)
\],
which can be simplified to
\[
|a||z| + |b||x| + |a||b| \leq \varepsilon|z| + \delta|x| + \varepsilon\delta
\]

Also note that if we use $x-y$ as $a$, $z-w$ as $b$, and derive $|xz-yw|$ from 
\[
xz=(y+a)(w+b),
\]
then what we will get is that $xz,yw$ are ($\delta|y| + \varepsilon|w| + \delta\varepsilon$) close.

This consequence may seem obvious, but in fact it isn't. And should we change some variables of them, the 
result may vary. This example tells us that we should be very cautious when dealing with inequalities. What we 
should do is to carefully derive conclusions from what we have proven instead of taking intuitive things for 
granted.
\end{proof}

\declareexercise{4.3.3}
\begin{proof}
(a)
\begin{enumerate}
\item Use induction. We induct on $m$. First, $x^nx^0 = x^n1 = x^{n+0}$.

Suppose that for $m$, the statement is already true. Then 
\begin{align*}
x^nx^{m+1} 
&= x^n(x^m\times x)\\
&= x^nx^m \times x \\
&= x^{n+m} \times x \tag{The induction hypothesis} \\
&= x^{n+m+1}
\end{align*}

\item Use induction. We induct on $m$. First, $(x^n)^0 = 1 = x^{n\times 0}$.

Suppose that for $m$, the statement is already true. Then 
\begin{align*}
(x^n)^{m+1} 
&= (x^n)^m \times x^n \\
&= x^{mn} \times x^n \tag{The induction hypothesis} \\
&= x^{mn + n} \tag{By the previous statement} \\
&= x^{n(m+1)}
\end{align*}

\item Use induction. We induct on $n$. First, $(xy)^0 = 1 = x^0y^0$.

Suppose that for $m$, the statement is already true. Then 
\begin{align*}
(xy)^{m+1} 
&= (xy)^m \times xy \\
&= x^my^m \times xy \tag{The induction hypothesis} \\
&= x^m \times x \times y^m \times y \\
&= x^{m+1} y^{m+1}
\end{align*}
\end{enumerate}

(b)
On one hand, if $x=0$, then for $n>0$, $0^n=0$.

On the other hand, if for $n>0$, $x^n=0$, we need to prove $x=0$. We try to show that 
$x \neq 0 \Longrightarrow x^n \neq 0$. Use induction. Since that $n\neq 0$, we start from $n=1$. 
$x^1 = x^0 \times x = x$. 

Suppose that for $n$, the statement is already true. Then 
\[
x^{n+1} = x^n \times x,
\]
which is the product of two positive rationals, and which is thus positive. 

(c)
(1)
Use induction: $x^0 = y^0 =1>0$.

Suppose that for $n$, the statement is already true. Then we have two inequalities here:
\[
x^n \geq y^n \geq 0
\]
and
\[
x \geq y \geq 0
\]
We can multiply the two because of Proposition \ref{prop.4.2.multiply.ineq}. Then we have 
\[
x^{n+1} \geq y^{n+1} \geq 0
\]

If $n>0$, then we induct from $1$. The process resembles to what we have just done, so I don't write it here.

(d)
Use induction: $|x^0| = |1| =1 =|1|^0$. 

Suppose that for $n$, the statement is already true. Then 
\[
|x^{n+1}| = |x^n \times x| = |x^n| |x| = |x|^n |x| = |x|^{n+1}
\]
\end{proof}

\paragraph{Definition 4.3.11}
We can see that there are now two versions of $x^{-1}$. Now we try to show that they express the same thing. 
Write $x$ as $a/b$. The first version is $x^{-1} = b/a$. 

The second version is $x^{-1} = 1/x = 1 \times x^{-1}\text{(version 1)} = x^{-1} = b/a$.

Note that only after we have known this can we say that for the second version of $x^{-1}$, 
$(x^{-1})^{-1} = x$.

Now we can also say that $x^{-n} = (x^n)^{-1}$

\declareexercise{4.3.4}
\begin{proof}
Except for (3),
we have already proven these properties when $m,n \in \mathbb{N}$. Then we will just write $m,n$ as $-m,-n$.

(a)
(1)
\[
x^{-m}x^{-n} = \frac{1}{x^m}\frac{1}{x^n} = \frac{1}{x^mx^n}=1/x^{m+n} = x^{-m-n}
\]

(2)
Before doing this, we must derive that for integers $a,b$ and natural number $n$, $(a/b)^n = a^n/b^n$. Use 
induction: $(a/b)^0 = 1 = 1/1 = a^0/b^0$.

Suppose that for $n$, the statement is already true. Then 
\[
(a/b)^{n+1} = (a^n/b^n)(a/b) = (a^{n+1}/b^{n+1})
\]

We can now close the induction.

Thus, 
\begin{align*}
&(x^{-n})^{-m} \\
&= ((1/(x^n))^m)^{-1} \tag{$x^{-n} = (x^n)^{-1}$} \\
&= (1/(x^n)^m)^{-1} \\
&= (x^n)^m \\
&= x^{mn}
\end{align*}

(3)
\begin{align*}
(xy)^{-n} 
&= 1/(xy)^n \\
&= 1/(x^ny^n) \\
&= (1/x^n) (1/y^n) \\
&= x^{-n}y^{-n}
\end{align*}

(b)
\begin{lem}
For $x = (a/b) >  y = (c/d) > 0$,
\[
0 < x^{-1} < y{-1}
\]
\end{lem}
\begin{proof}
Let $a,b,c,d>0$. We know that
\[
a \times b^{-1} > c \times d^{-1}
\]
Multiply it with $bd>0$, we have
\[
ad>bc
\].
Multiply it with $a^{-1}c^{-1} >0$, we have
\[
d/c>b/a
\]
That is,
\[
y^{-1} > x^{-1}
\]
And they are obviously bigger than $0$.
\end{proof}

According to the lemma,
\[
x^{n} \geq y^{n} \Longrightarrow ((x^{n})^{-1} \leq (y^{n})^{-1} \equiv x^{-n} \leq y^{-n})
\]

(c)
We first show that for positive integer $n$, the statement is true. We assume that $x>y$. Another situation is 
similar.
Use induction, we try to prove that $x\neq y \Longrightarrow x^n \neq y^n$. We need to start from $n=1$. First, 
$x^1=x > y^1=y$. 

Suppose that for $n$, the statement is already true. Then
Multiply $x>y$ with $x^n > y^n$, 
we have $x^{n+1} > y^{n+1}$.

We can now close the induction.

Then for negative ones, we know that $1/x = 1/y$ iff $x=y$, so $x^n \neq y^n \equiv x^{-n} \neq y^{-n}$.

(d)
It is immediately derived since $1/x = 1/y \equiv x=y$.
\end{proof}

\declareexercise{4.3.5}
\begin{proof}
Use induction from $1$.

$2^1 = 2 >1$.

Suppose that for $N$, the statement is already true. Then
\[
2^{N+1} = 2\times 2^N > 2N \geq N+1
\]
(Note that $N\geq 1 \Longrightarrow 2N \geq N+1$)
\end{proof}

\subsection{Gaps In The Rational Numbers}
\declareexercise{4.4.1}
\begin{proof}
Existence: We show that when $x \geq 0$, $(\exists n \in \mathbb{Z})(n \leq x < n+1)$. Write $x$ as $a/b$, 
where $a \geq 0,b>0$. If $a= 0$, then $x=0$, $n=0$. If $a\neq 0$, then according to Proposition 2.3.9, 
\[
\exists m \exists r(a = mb+r),
\]
where $m,r \in \mathbb{N}$ and $r < b$. Because of this, $mb+b >a$, so $(mb+b)/b>a/b=x$, which means $m+1>x$. 
On the other hand, $mb \leq a$, so $mb/b \leq a/b =x$, which means $m \leq x$.

Then when $x<0$, then $-x>0$, and $(\exists n \in \mathbb{Z})(n \leq -x < n+1)$, so 
\[
-n \geq x > -n -1
\]
$\geq$ means $>$ or $=$ (exclusive). When $-n > x >-n-1$, let $m=-n-1$, then $m \leq x < m+1$ is true. When 
$-n =x > -n-1$, let $m=-n$, then $m \leq x < m+1$ is also true. So $m$ is the integer we want if $x<0$.

Uniqueness:
For $n \leq x < n+1$, and $m \neq n$, we try to prove that $m \leq x<m+1$ is not possible. Before doing this, 
we need some lemmas:
\begin{lem}
(1) For integers $i,j$, $i<j \equiv i+1\leq j$.
(2) For integer $i$, there is no integer $j$ such that $i<j<i+1$.
\end{lem}
\begin{proof}
(1) It has already been proven for natural numbers. If $i,j<0$, then $-i,-j>0$.
\[
i<j \equiv -i>-j \equiv -i \geq -j+1 \equiv i \leq j-1 \equiv i+1 \leq j
\]

(2)
Suppose the negation, that there exists a integer $j$ such that $i<j<i+1$. We know that 
$i<j \equiv i+1 \leq j \equiv j \geq i+1$. But we also have $j<i+1$, which is impossible.
\end{proof}

$m$ either $<$ or $>n$. On the former case, $m+1\leq n\leq x$, so $m+1>x$ is not possible. One the latter case, 
$x<n+1\leq m$, so $m \leq x$ is impossible.
\end{proof}

\declareexercise{4.4.2}
\begin{proof}
(a) We will use a different approach from the hint the book provided here. After assuming the negation, we 
try to prove that 
$a_n \leq a_0 -n$. Note that subtraction may results in a overflow (that is, natural numbers flows to negative 
integers). So we will first define $a_n$ as integers. And we try to show that no such infinite descent 
sequences can only lie in  $\mathbb{N}$.

Use induction: $a_0 \leq a_0-0$.

Suppose the statement for $n$ is already true, then 
$a_{n+1} < a_n \equiv a_{n+1} +1 \leq a_n \equiv a_{n+1} \leq a_n - 1$. Then we have 
$a_{n+1} \leq (a_0 -n -1 = a_n -(n+1))$. We can now close the induction.

However, let $n = a_0$, then $a_{n+1} \leq n-n-1 = -1$, which means that $a_{n+1}$ does not lie in 
$\mathbb{N}$.

(b)
(1) Yes. For example, $a_n:=-n$ satisfies our restrictions.

(2) Yes. Because it is always possible to find a rational between $0$ and $a_0$.
\end{proof}

\declareexercise{4.4.3}
\begin{proof}
(1) Suppose the negation, that natural number $n=2k=2k'+1$, where $k,k'$ are also natural. Then $2k=2k'+1$
Then we have $2k > 2k' \Longrightarrow k>k'$. But $2k+1 = 2(k'+1)$, so $2(k'+1) > 2k \Longrightarrow k'+1 >k$. 
But we know that between $k',k'+1$ there exists no natural numbers. so it is impossible. (Note that we don't 
have a proposition saying $ac>bc \Longrightarrow a>b$ for natural numbers, but we can first deal with them with 
the range of rationals. Multiply them with $c^{-1}$, and we will see that the result of the two sides are also 
natural numbers, so for natural numbers this is true.)

(2)
\[
p^2 = (2k+1)^2 = 4k^2+4k + 1 = 2(2k^2+2k) +1
\]

(3)
Treat $p,q$ as rationals.
$p^2/2=q^2 \Longrightarrow q^2<p^2$. We show that $q \geq p$ can not be true. It is obvious that $q \neq p$. 
And when $q>p$, multiply it with itself, $q^2>p^2$, which is impossible.
\end{proof}