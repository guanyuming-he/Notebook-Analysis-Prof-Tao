% Copyright (C) He Guanyuming 2024
% The file is licensed under the MIT license.

\documentclass[oneside]{book}

\usepackage{realanalysis}

\author{Guanyuming He}
\title{Notebook for Analysis by Prof Tao}
\date{\today}

\usepackage{hyperref}
\hypersetup
{
colorlinks=true
}

\usepackage{mystyles}

\includeonly{Natural Numbers,Set Theory}

\begin{document}
\pagestyle{empty}
\pagenumbering{gobble}
\maketitle

This document serves as a notebook for Prof Terence Tao's \emph{Analysis, Fourth Edition}. I write it mainly to record all the solutions I make to the exercises. Along the way, I also write down important propositions and my thoughts that Prof Tao didn't mention in the book.

\vfill

\begin{center}
Copyright \copyright{} Guanyuming He 2020--\the\year{}. 

This document is licensed under the MIT license.

You may get a copy of source code of the document at 
\url{https://github.com/real-guanyuming-he/Notebook-for-Analysis-of-Tao}.
\end{center}

\newpage
Thank my parents for, despite the financial challenges and the differences in our views of the world, supporting my education.

Thank Prof.~Tao for writing this amazing book and providing many insightful blogs/talks/papers that inspired me to be persistent and think better about maths.

Finally, thank God; Jesus Christ be praised.

\newpage
\pagestyle{headings}
\pagenumbering{roman}
% Copyright (C) He Guanyuming 2020
% The file is licensed under the MIT license.

\section*{General Principles}
This section describes the overall principles of the document: in what style and format it is typeset,
in what structure this document is written, and so forth.

\subsection*{Definitions}
\begin{description}
\item[The document] The phrase \emph{the document} means this document (what you are reading) itself.
\item[The book] The phrase \emph{the book} represents Tao's \emph{Analysis} (both volume I and II).
\end{description}

\subsection*{Indices}
The book has two volumes: 
\emph{Analysis I} and \emph{Analysis II}. We may notice that the indices of the two volumes both start 
from 1. It may lead to some confusions. So in the document, the indices are organized in such a way that:
If the content comes from \emph{Analysis I}, the corresponding index is the same as the book's. 
Otherwise, the corresponding index is prefixed with ``2.''.

For example, Exercise 3.1.3 in \emph{Analysis I} is indexed as Exercise 3.1.3 in the document, but 
Exercise 3.1.3 in \emph{Analysis II} is indexed as Exercise 2.3.1.3.

\subsection*{Notations}
Professor Tao uses many kinds of statements: Theorems, Propositions, Lemmas, etc. However, in this document, I will use them, too. Therefore, to distinguish between his and my statements,
I will prepend all my statements with ``My''. For example, Proposition 2.2.1 in this document will be written as My Proposition 2.2.1 in this document.

Logical statements are often involved throughout this book, especially heavily in Chapter 3, Set Theory. Therefore, I will use some logical symbols that Prof.~Tao didn't mention or uses other notations.
\begin{itemize}
	\item I will use $\forall$ for ``for all'', and $\exists$ for ``there exists.''
	\item I will use $\to$ to mean implies, while Prof.~Tao uses $\Rightarrow$ for it.
	\item I will use $\equiv$ to mean two logical statements are the same. That is, if I say $p \equiv q$, then $q$ is true whenever $p$ is true, and vice versa.
	\item I don't know yet, but perhaps sometimes it is more appropriate to say $p \dashv\vdash q$, if I want to express that, from $p$ we can have a proof for $q$, and vice versa. But it really makes no difference, as far as truth is concerned, with saying that $p \equiv q$, thanks to the soundness and completeness of propositional and first-order predicate logic, which are the only logical systems involved in this book.
\end{itemize}

Although in logical contexts, people often refer to something that can hold a truth value ($\true$ or $\false$) as a formula, I will use the same name, ``statement'', as Prof.~Tao, to avoid confusion.

\subsection*{Abbreviations}
We often leave off some descriptions for a number's properties. For example, we may refer a 
\emph{positive natural number} $n$ as a \emph{positive number} when we haven't learned the rationals and 
the reals.

\newpage
\tableofcontents

%\listofanswers

\newpage
\pagenumbering{arabic}
\pagestyle{headings}

% Copyright (C) He Guanyuming 2020
% The file is licensed under the MIT license.

% There is another section with no exercise before this one, so this section should be the second section.
\setcounter{chapter}{1}
\chapter{Natural Numbers}
\section{The Peano Axioms}
No notes for this section.

\section{Addition of Natural Numbers}
\begin{why}{20}
The sum of two natural numbers is still a natural number. That is, for any natural numbers $m,n$, $n+m$ is also a natural number.
\end{why}
\begin{proof}
We use induction on $n$. 

\mybcbox When $n = 0$, $0 + m$ by definition equals to $m$, which is a natural number.

\myisbox Suppose that it is already true for some $N$, then the sum $\successor{N} + m$ by definition is $\successor{(N+m)}$. The induction hypothesis tells that $N+m$ is a natural number, so by Axiom 2.2 its successor also is.

We can now close the induction.
\end{proof}

\begin{exercise}{2.2.1}
Addition of natural numbers is associative. That is, for any natural numbers $a,b,c$, $a+(b+c) = (a+b)+c$.	
\end{exercise}
\begin{proof}
Induct on $a$.

\mybcbox When $a = 0$, $0 + (b+c)$ by definition equals to $b+c$. $(0+b)+c = (b)+c$, which equals to the former result. 

\myisbox $\successor{a}+(b+c) = \successor{(a+(b+c))}$. We have $(\successor{a}+b)+c = \successor{(a+b)}+c = \successor{((a+b)+c)}$. But by the inductive hypothesis we know that $(a+(b+c)) = ((a+b)+c)$, hence they are equal.

We can now close the induction.
\end{proof}

After we have the cancellation law, it would be nice to have an additional proposition that Tao didn't mention in the book. Our intuition would let us think that whenever we count, we get a new, different number, that is, $1\ne 2,3,4,5,\dots;\quad 2\ne 3,4,5,6,\dots$ and so on. But can we formalise it for all natural numbers?
\begin{prop}\label{my.prop.natural.number.ne}
Let $m,n$ be such natural numbers that $m=n+a$ for natural number $a\ne 0$, which means $m$ is produced by counting $a$ times from $n$. Then, we have $m \ne n$.
\end{prop}
\begin{proof}
Surprisingly, we don't need induction here. If we were to use it, we would first need to prove the following corollary.

Suppose for the sake of contradiction that $m=n$, then by the definition and commutativity of addition, $m=0+n=n+0$. Hence, $n+a=n+0$, and by the cancellation law we have $a=0$, a contradiction.
\end{proof}

Following it we have this useful corollary.
\begin{coro}
Let $n$ be any natural number. Then, $n\ne \successor{n}$.
\end{coro}

\begin{proof}
It immediately follows the fact that $\successor{n} = n+1$ and My Proposition~\ref{my.prop.natural.number.ne}.
\end{proof}

\begin{exercise}{2.2.2}
For any positive natural number $n$, there is exactly one natural number $m$ that $\successor{m} = n$. 
\end{exercise}

\begin{proof}
Note that, because of Axiom 2.4, if $\successor{m} = n$, then $m$ must be unique. Therefore, we only need to prove that for each positive $n$, there exists some $m$ such that $\successor{m} = n$.
	
We induct on $n$. However, as $n$ does not start from $0$, we need to rewrite the statement a little bit:
$$
n \ne 0 \to (\exists m, \successor{m} = n)
$$

\mybcbox When $n = 0$, this statement is vacuously true.

\myisbox Set $m=n$, then $\successor{m} = \successor{n}$.

We can now close the induction.
\end{proof}

\begin{exercise}{2.2.3}
\begin{enumerate}
	\item $a \ge a$
	\item $a \ge b \wedge b \ge c \to a \ge c$.
	\item $a \ge b \wedge b \ge a \to a = b$.
	\item $a \ge b \to a+c \ge b+c$.
	\item $a < b$ iff $\successor{a} \le b$.
	\item $a < b$ iff $b = a+d$, where $d \ne 0$.
\end{enumerate}
\end{exercise}
\begin{proof}\leavevmode
\begin{enumerate}
	\item $a=a+0$.
	\item Let $a = b+m$ and $b = c+n$, then $a = (c+n)+m = c +(n+m)$, by the associativity of addition.
	\item Let $a=b+m$ and $b=a+n$. Then, $a=(a+n)+m = a+(n+m)$, by the associativity of addition.
	
	Because $a=a+0$, we have $a+0 = a+(n+m)$. By the cancellation law, $0 = m+n$. By Corollary 2.2.9, $m=n=0$. Hence, $a=b+0=b$.
	\item Let $a = b+m$. Then $a+c=(b+m)+c = b+(m+c) = b+(c+m) = (b+c)+m$.
	\item 
	\begin{enumerate}
		\item If $a < b$, then $b=a+m$ for $m \ne 0$. By Lemma 2.2.10, $m = \successor{n}$ for some $n$. Hence, $b=a+\successor{n} = \successor(a+n) = \successor{a}+n$, which by definition means $b \ge \successor{a}$.
		
		\item
		If $\successor{a} \le b$, then $b=\successor{a}+n$ for some $n$. Hence, $b=a+\successor{n}$. By Axiom 2.3, $\successor{n}$ is positive, thus $b=a+m$ for some positive number $m$. Using My Proposition~\ref{my.prop.natural.number.ne}, we have $b \ne a$. 
		
		However, to show that $a < b$, we also need to show that $a \le b$. Using the transitivity of order, this can be done by observing that $a \le \successor{a} \le b$. Therefore, we have $a \le b \wedge a \ne b$, and it follows that $a < b$.
	\end{enumerate}

	\item 
	\begin{enumerate}
		\item If $b=a+d$, then $a \le b$. But because $d$ is positive, using My Proposition~\ref{my.prop.natural.number.ne} we know that $a \ne b$ and hence $a < b$.
		
		\item If $a<b$, then $a \le b \wedge a \ne b$. Which means $b=a+d \wedge b \ne a+0$. Therefore, $a+d \ne a+0$, and we cannot have $d=0$.
	\end{enumerate}
\end{enumerate}
\end{proof}

\begin{exercise}{2.2.4}
The exercise consists of these (Why?)'s
\begin{why}{23}
$0 \le b$ for all natural numbers $b$.
\end{why}
\begin{why}{23}
$a>b \to \successor{a}>b$.
\end{why}
\begin{why}{23}
$a=b \to \successor{a}>b$.
\end{why}
\end{exercise}
\begin{proof}\leavevmode
\begin{enumerate}
	\item We have $b=0+b$ for all $b$. As $b$ is a natural number, by the definition of order we have $b \ge 0$.
	\item If $a>b$, then $a=b+m$. Hence, $\successor{a} = \successor(b+m) = b+\successor{m}$. By Axiom 2.3, $\successor{m} \ne 0$. Hence, $\successor{a} > b$ by Exercise 2.2.3 (f).
	\item If $a=b$, then $\successor{a} = \successor{b} = b+1$. Hence, $\successor{a} > b$ by Exercise 2.2.3 (f).
\end{enumerate}
\end{proof}

\begin{exercise}{2.2.5}
Let $m_0$ be a natural number, and $P(m)$ be a property pertaining to any natural number $m$. Suppose that whenever
$$
(P(m') \text{ for all } m_0 \le m' < m) \to P(m)
$$
, we have $P(m)$.

Then, we can conclude that $P(m)$ is true for all $m \ge m_0$.
\end{exercise}
\begin{proof}
Quite obviously, induction is the only tool that we can use now to prove something is true for all natural numbers. 

However, if we try to apply induction directly to $P$, we will see that the inductive hypothesis only gives a single $P(n)$, not $P(n)$ over the whole range $m_0 \le n < m'$. This failure makes us realise that Prof Tao's hint is very important.

Using Prof Tao's hint: define $Q(n)$ to be true iff $P(m)$ is true for all $m_0 \le m < n$. If we show that $Q(n)$ is true for all $n$, then automatically $P(m)$ is true for all $m \ge m_0$: to let $P$ be true for some $m$, just set $n=\successor{m}$, and since $m < \successor{m}$, $P(m)$ is true.

To show that $Q(n)$ is true for all $n$, we induct on $n$.

\mybcbox 
$Q(0)$ is vacuously true, which is because $0 \le m_0$ for all $m_0$. To show this, consider that $m_0 = 0 + m_0$.

If we are to have a $m$ such that $m_0 \le m < 0$, then by the transitivity of order we also have $m \ge 0 \wedge m < 0$, which is impossible, according to Proposition 2.2.13.

\myisbox 
By the inductive hypothesis, $Q(n)$ is true, which means $P(m)$ is true for all $m_0 \le m < n$. By the definition of $P$, $P(n)$ is also true. Given any natural number $a \ge m_0$, we have, by Proposition 2.2.13, three situations:
\begin{enumerate}
	\item If $a < n$, then $P(a)$ is true by the inductive hypothesis.
	\item If $a = n$, then $P(a)$ is also true, as we just showed.
	\item If $a > n$, then we don't know yet.
\end{enumerate}
The first two situations are enough to conclude that for all $m_0 \le a \le n$, $P(a)$ is true. Using proposition 2.2.12 (e), we can rewrite this as $m_0 \le a < \successor{n}$. 

We can now close the induction.
\end{proof}

\begin{exercise}{2.2.6}
	Let $n$ be a natural number. Let $P(m)$ be a property pertaining to all natural numbers. If
	\begin{enumerate}
		\item $P(n)$ is true.
		\item $P(\successor{m}) \to P(m)$,
	\end{enumerate}
	then $P(m)$ is true for all $m \le n$.
\end{exercise}
\begin{proof}
	Using Prof Tao's hint, induct on $n$, on this property $Q(n)$, defined as the principle of backwards induction works on $n$.
	
	\mybcbox 
	When $n = 0$, we need to show $Q(0)$, which means $P(m)$ is true for all $m \le 0$. We can show that only $m=0$ can satisfy $m \le n$. Recall the fact that $m \ge 0$ by $m = 0+m$. Hence, for $m$ to both $\le 0$ and $\ge 0$, $m$ can only $=0$, by Proposition 2.2.13.
	
	\myisbox 
	If $P(m)$ is already true for all $m \le n$, then we need to show that the principle works for $\successor{n}$. However, the premise of the principle automatically gives $P(\successor{n})$. By proposition 2.2.13, $m \le n$ together with $m = \successor{n}$ give all such $m$'s that $m \le \successor{n}$. Hence, $P(m)$ is true for all of them.
	
	We can now close the induction.
\end{proof}

\begin{exercise}{2.2.7}
		Let $P$ be such a property pertaining to the natural numbers that
		\begin{enumerate}
			\item $P(n)$ is true.
			\item $P(m) \to P(\successor{m})$,
		\end{enumerate}
		then $P(m)$ is true for all $m \ge n$.
\end{exercise}
\begin{proof}
	Although we could walk a similar path as we walked to prove the previous two exercises, that is, define a property $Q$ such that $Q(n)$ means that the principle works for $n$, I choose to walk a different path this time.

	Let $Q(m)$ be true for all $m < n$ and be equal to $P(m)$ for all $m \ge n$. By Proposition 2.2.13, this definition includes every natural number $n$.
	
	To prove that $P(m)$ is true for all $m \ge n$, we can show that $Q(m)$ is true for every natural number $m$. This is easy, just induct on $m$.
	
	\mybcbox 
	By definition, $Q(0)$ is true, because $0 \le n$ for any $n$. If $0 < n$, then by definition $Q(0)$ is true. If $0 = n$, then by the premise $Q(0) = P(0)$ is true.

	\myisbox 
	There are two situations to consider
\begin{enumerate}
	\item If $m < n$, then $\successor{m} \le n$, and $Q(\successor{m})$ is true by definition.
	\item If $m \ge n$, then $P(m)$ is defined and equals to $Q(m)$, which by the inductive hypothesis is true. By the premise, this implies $P(\successor{m})$ is true. By definition, $Q(\successor{m})$ equals it, and is also true.
\end{enumerate}

We can now close the induction.
\end{proof}

\section{Multiplication of Natural Numbers}
\begin{exercise}{2.3.1}
	Let $a,b$ be natural numbers. Then, $a \times b = b \times a$.
\end{exercise}
To prove this exercise, we can always follow the path we walked when we try to prove the commutativity of addition. In fact, since we can regard the definition of multiplication as replacing $\successor{}$ with $+$ in the definition of addition, we can simply replace it with $+$ in the proofs, too.

We introduce the two lemmas:
\begin{lem}\label{my.cpmmutative.mul.lem.1}
	Let $m$ be any natural number, then $0 \times m = m \times 0$.
\end{lem}
\begin{proof}
	We induct on $m$.
	
	\mybcbox $0 \times 0 = 0 \times 0$.
	
	\myisbox $0 \times \successor{m} = 0$ by definition. $\successor{m} \times 0 = (m \times 0) + 0$, and $m \times 0 = 0 \times m = 0$, according to the inductive hypothesis.
	
	We can now close the induction.
\end{proof}

\begin{lem}\label{my.cpmmutative.mul.lem.2}
	Let $m, n$ be any natural number, then $n \times \successor{m} = n \times m + n$.
\end{lem}
\begin{proof}
	We induct on $n$.
	
	\mybcbox $0 \times \successor{m} = 0 = 0 + 0$.
	
	\myisbox We need to prove $\successor{n} \times \successor{m} = (\successor{n} \times m) + \successor{n}$.
	
	The left side equals $(n \times \successor{m}) + \successor{m}$, by definition. By the inductive hypothesis, it also equals to $(n \times m + n) + \successor{m} = n \times m + n + m + 1$.
	
	The right side, by definition, equals $n \times m + m + \successor{n} = n \times m + m +n +1$. By the properties of addition, the two are equal.
	
	We can now close the induction.
\end{proof}

Now, we prove the exercise.
\begin{proof}
	Induct on $m$.
	
	\mybcbox $0 \times n = n \times 0$, by My Lemma~\ref{my.cpmmutative.mul.lem.1}.
	
	\myisbox We need to prove $\successor{m} \times n = n \times \successor{m}$. The left side equals $m \times n + n$, by definition. The right side equals to $n \times m + n$, according to My Lemma~\ref{my.cpmmutative.mul.lem.2}. However, they are the same, by the inductive hypothesis.
	
	We can now close the induction.
\end{proof}

\begin{exercise}{2.3.2}
	Let $m, n$ be natural numbers. Then, $mn = 0$ iff $m = 0 \vee n = 0$. In particular, if $m,n$ are both positive, that is, $\neg(m = 0 \vee n = 0)$, then $mn \ne 0$.
\end{exercise}
\begin{proof}\leavevmode\par % To align the two paragraph start markers.
	\myifbox 
	If one of $m,n$ is zero, then by the definition of multiplication and the commutativity, their product is also zero.
	
	\myoifbox 
	We induct on $m$.

	\leavevmode\hskip\parindent\mybcbox
		If $m = 0$, then $(mn = 0) \to (m = 0 \vee n = 0)$ is true, because the both sides of $\to$ are true.
	
	\leavevmode\hskip\parindent\myisbox
		If $(\successor{m})n = 0$, then, by definition, $mn + n = 0$. According to Corollary 2.2.9, they can only both be 0. Hence, $n = 0$, and $\successor{m} = 0 \vee n = 0$ is true.
	
	We can now close the induction.
\end{proof}

\begin{exercise}{2.3.3}
	Let $a,b,c$ be natural numbers. Then, $(a\times b)\times c = a \times (b \times c)$.
\end{exercise}
\begin{proof}
	We induct $a$. We could induct on $b$ or $c$, but when handling their successors, we would need to additionally use the commutativity.
	
	\mybcbox $(0 \times b) \times c = 0 \times c = 0 = 0 \times (b \times c)$.
	
	\myisbox $(\successor{a} \times b) \times c = (ab + b) \times c$. This equals to $(ab)c + bc$ by distributive law. We also have $\successor{a} \times bc = a(bc) + bc$ by definition. According to the inductive hypothesis, $(ab)c = a(bc)$.
	
	We can now close the induction.
\end{proof}

\begin{exercise}{2.3.4}
	\[
		(a+b)^2 = a^2 + 2ab + b^2
	\]
\end{exercise}
\begin{proof}
\begin{align*}
	&(a+b)(a+b) \\
	&= (a+b)a + (a+b)b &\text{distributive law} \\
	&= (a^2 + ba) + (ab + b^2) &\text{distributive law} \\
	&= \bigl((a^2 + ba)+ab\bigr) +b^2 &\text{associativity of addition} \\
	&= \bigl(a^2 + (ba+ab)\bigr) +b^2 &\text{associativity of addition} \\
	&= \bigl(a^2 + (ab+ab)\bigr) +b^2 &\text{commutativity of multiplication} \\
	&= \bigl(a^2 + 2ab\bigr) +b^2 &\text{def.~of multiplication} \\
	&= a^2 + 2ab + b^2 &\text{associativity of addition}
\end{align*}
\end{proof}

\begin{exercise}{2.3.5}[Euclid's Division Lemma]
	Let $n$ be a natural number, and $q$ be a positive natural number (so that it is not 0 as the divisor). Then, $n$ ``divided'' by $q$ will always get a quotient and a remainder.
	
	That is, there exists such natural numbers $m,r$ that 
	\[
	0 \le r < q \quad \wedge \quad n = mq + r
	\]
\end{exercise}
\begin{proof}
	We induct on $n$.
	
	\mybcbox Set $m = r = 0$, and we have $0 = 0q + 0$, as desired.
	
	\myisbox $\successor{n}$, by the inductive hypothesis, equals to $\successor{(mq+r)} = mq+\successor{r}$.
	
	As $0 \le r < q$, $0 < \successor{r} \le q$, by Proposition 2.2.12 (e). According to the trichotomy of order, $\successor{r}$ either $<q$ or $=q$
	\begin{itemize}
		\item If $\successor{r}<q$, then we set $m' = m, r' = \successor{r}$.
		\item If $\successor{r}=q$, then we have $\successor{n} = mq + q = (\successor{m})q$, by the definition of multiplication. However, it also equals $(\successor{m})q + 0$.
		
		Therefore, we set $m' = \successor{m}, r' = 0$.
	\end{itemize}

	We can now close the induction.
\end{proof}

\newpage
% Copyright (C) He Guanyuming 2020
% The file is licensed under the MIT license.

\chapter{Set Theory}
Before I start, I would like to write an honest opinion of mine: many definitions and axioms of the set theory involve the encapsulation and manipulation of basic logic constructs: $\forall, \exists, \wedge, \vee, \to, \dots$

As a result, the proofs in this chapter will also include rely on them heavily.

Some additional notes:
\begin{enumerate}
	\item $p \leftrightarrow q$ is defined as $p \to q \wedge p \leftarrow q$.
\end{enumerate}

\section{Fundamentals}
\begin{why}{29}\label{why.unique.empty.set}
If $\varnothing$ and $\varnothing'$ are both empty sets, then they are equal.
\end{why}
\begin{proof}
	Observe that 
	\[
	\forall x (x \in \varnothing \leftrightarrow x \in \varnothing')
	\]
	because in either direction the statement is vacuously true.
\end{proof}

\begin{why}{30}
	Let $A = \{a\}, A' = \{a\}$, then $A = A'$.
\end{why}
\begin{proof}
	Let $x$ by any object. By the property of $=$, we have exclusively either $x=a$ or $x \ne a$.
	
	\begin{itemize}
		\item If $x = a$, then $x \in A \leftrightarrow x \in A'$ is true.
		\item If $x \ne a$, then $x \in A \leftrightarrow x \in A'$ is vacuously true.		
	\end{itemize}

	Therefore, $x \in A \leftrightarrow x \in A'$ is always true. By the definition of set equality, $A = A'$.
\end{proof}

\begin{why}{30}
	Let $A = \{a,b\}, A' = \{b,a\}$, then $A = A'$.
\end{why}
\begin{proof}
	Let $x$ by any object. $x = a \vee x = b$ is a statement, so it is either exclusively true or false.
	
	\begin{itemize}
		\item If $x = a \vee x = b$ is true, then $x \in A \leftrightarrow x \in A'$ is true.
		\item If $x = a \vee x = b$ is not true, then $x \in A \leftrightarrow x \in A'$ is vacuously true.		
	\end{itemize}
	
	Therefore, $x \in A \leftrightarrow x \in A'$ is always true. By the definition of set equality, $A = A'$.
\end{proof}

\begin{why}{30}
	Let $A = \{a\}, A' = \{a,a\}$, then $A = A'$.
\end{why}
\begin{proof}
	Observe that $x \in A'$ iff $x = a$. Then, apply the proof of $\{a\} = \{a\}$ here.
\end{proof}

\begin{exercise}{3.1.1}
	Let $a,b,c,d$ be objects such that $\{a,b\} = \{c,d\}$, then at least one of
	\begin{enumerate}
		\item $a = c \wedge b = d$
		\item $a = d \wedge b = c$
	\end{enumerate}
	is true.
\end{exercise}
\begin{proof}
	I tried going straight to show all possibilities, but that yielded too many cases.
	
	Therefore, I prove by contradiction. Suppose for the sake of contradiction that neither of the two statements is true. In particular, $a \ne c \wedge a \ne d$.
	
	Then, we must have $\{a,b\} \ne \{c,d\}$, because $a \in \{a,b\}$ but $a \notin \{c, d\}$, which means $a \in \{a,b\} \to a \in \{c,d\}$ is false. This gives a contradiction.
\end{proof}

\begin{exercise}{3.1.2}
	$\varnothing, \{\varnothing\}, \{\{\varnothing\}\}$, and $\{\varnothing,\{\varnothing\}\}$ are all distinct.
\end{exercise}
\begin{proof}
	First, because all except $\varnothing$ contain some element, they are all non-empty, by Axiom 3.2 and 3.3.
	
	Now, we show that $\{\varnothing\}, \{\{\varnothing\}\}$, and $\{\varnothing,\{\varnothing\}\}$ are distinct. 
	
	Observe that $\{\varnothing\} \notin \{\varnothing\}$, but it is in both $\{\{\varnothing\}\}$, and $\{\varnothing,\{\varnothing\}\}$. Therefore, $\{\varnothing\}$ is distinct from the rest two.
	
	Now, we only need to show that $\{\{\varnothing\}\} \ne \{\varnothing,\{\varnothing\}\}$. This is done by noticing that $\varnothing$ is in the latter, but not in the former.
\end{proof}

\begin{why}{30}
	Let $A,B,A'$ be sets. If $A = A'$, then $A \cup B = A' = B$.
\end{why}
\begin{proof}\leavevmode
	\begin{enumerate}
		\item Let $x$ be any element in $A \cup B$. According to Axiom 3.5, $x \in A \vee x \in B$. 
		\begin{enumerate}
			\item If $x \in A$, then by Axiom 3.2, $x \in A'$, then $x \in A' \vee x \in B$, which, according to Axiom 3.5 again, means $x \in A' \cup B$. 
			\item If $x \in B$, then $x \in A' \vee x \in B$, which, according to Axiom 3.5 again, means $x \in A' \cup B$.
		\end{enumerate}
		
		\item Let $x$ be any element in $A' \cup B$. According to Axiom 3.5, $x \in A' \vee x \in B$. 
		\begin{enumerate}
			\item If $x \in A'$, then by Axiom 3.2, $x \in A$, then $x \in A \vee x \in B$, which, according to Axiom 3.5 again, means $x \in A \cup B$. 
			\item If $x \in B$, then $x \in A \vee x \in B$, which, according to Axiom 3.5 again, means $x \in A \cup B$.
		\end{enumerate}
	\end{enumerate}
\end{proof}

\begin{exercise}{3.1.3}
\begin{enumerate}
	\item $A \cup B = B \cup A$.
	\item $\{a,b\} = \{a\} \cup \{b\}$
\end{enumerate}
\end{exercise}
\begin{proof}\leavevmode
	\begin{enumerate}
		\item This follows from the commutativity of logical or.
		\item By Axiom 3.4, $x \in \{a,b\}$ iff $x = a$ or $x = b$. By Axiom 3.5, $x \in \{a\} \cup \{b\}$ iff $x \in \{a\}$ or $x \in \{b\}$.
		
		However, by Axiom 3.4 again, $x \in \{a\}$ iff $x = a$; $x \in \{b\}$ iff $x = b$. Hence, the two logical statements are equivalent.
	\end{enumerate}
\end{proof}

\begin{why}{32}
	For any set $A$,
	\begin{itemize}
		\item $A \subseteq A$.
		\item $\varnothing \subseteq A$.
	\end{itemize}
\end{why}
\begin{proof}\leavevmode
	\begin{itemize}
		\item For all $x$, $x \in A \to x \in A$ is true.
		\item For all $x$, $x \in \varnothing \to x \in A$ is vacuously true.
	\end{itemize}
\end{proof}

\begin{exercise}{3.1.4}
	\begin{enumerate}
		\item {\bf Antisymmetry.} If $A \subseteq B$ and $B \subseteq A$, then $A = B$. 
		\item If $A \subsetneq B$ and $B \subsetneq C$, then $A \subsetneq C$.
	\end{enumerate}
\end{exercise}
\begin{proof}\leavevmode
	\begin{enumerate}
		\item $A \subseteq B$ means that every element in $A$ is an element in $B$; $B \subseteq A$ means vice versa. Combining the two, we have the definition of $A = B$.		
		\item Prof.~Tao already showed that $A \subseteq C$. Now we only need to show that $A \ne C$. 
		
		Because $A \subsetneq B$, $A \ne B$. Hence, there exists such an element $x$ that is not in both. However, as everyone in $A$ is in $B$, we can only have $x \in B \wedge x \notin A$.
		
		Because $B \subseteq C$, we must also have $x \in C$. Therefore, $A \ne C$, as $x \in A$.
	\end{enumerate}
\end{proof}

\begin{why}{33}
	$\{x \in A : P(x)\} \subseteq A$.
\end{why}
\begin{proof}
	By Axiom 3.6, for any $y \in \{x \in A : P(x)\}$, we have $y \in A$.
\end{proof}
\begin{why}{33}
	If $A = A'$, then $\{x \in A : P(x)\} = \{x \in A' : P(x)\}$.
\end{why}
\begin{proof}
	By Axiom 3.6,
	\begin{enumerate}
		\item For any $y \in \{x \in A : P(x)\}$, $y \in A \wedge P(y)$. 
		
		Because $A = A'$, $y \in A' \wedge P(y)$. By Axiom 3.6 again, $y \in \{x \in A' : P(x)\}$.
		\item For any $y \in \{x \in A' : P(x)\}$, $y \in A' \wedge P(y)$. 
		
		Because $A = A'$, $y \in A \wedge P(y)$. By Axiom 3.6 again, $y \in \{x \in A : P(x)\}$.
	\end{enumerate}
\end{proof}

\begin{why}{34}
	$\varnothing$ and $\varnothing$ are disjoint, but not distinct.
\end{why}
\begin{proof}\leavevmode
	\begin{itemize}
		\item They are disjoint, because no element $x$ can satisfy $x \in \varnothing \wedge x \in \varnothing$.
		\item They are not distinct, because we showed earlier that any empty set is the same as $\varnothing$.
	\end{itemize}
\end{proof}

\begin{exercise}{3.1.5}
These are logically equivalent:
\begin{enumerate}
	\item $A \subseteq B$.
	\item $A \cup B = B$.
	\item $A \cap B = A$.
\end{enumerate}
\end{exercise}
\begin{proof}\leavevmode
	\begin{enumerate}
		\item If $A \subseteq B$, then
		\begin{enumerate}
			\item For every $x \in B$, by Axiom 3.5, $x \in A \cup B$. For every $y \in A \cup B$, by Axiom 3.5, $y$ in $A$ or $B$. But when $y \in A$, $y$ also $\in B$, as $A \subseteq B$. 
			
			Hence, $A \cup B = B$.
			\item For every $x \in A \cap B$, by definition, $x \in A$ and $B$, then $x \in A$. For every $y \in A$, by the definition of subsets, $y \in B$. Hence $y \in A \wedge y \in B$, and $y \in A \cap B$ follows. 
			
			Therefore, $A \cap B = A$.
		\end{enumerate}
	
		\item If $A \cup B = B$, then
		\begin{enumerate}
			\item For every $x \in A$, by Axiom 3.5, $x \in A \cup B$. By hypothesis, $x \in B$. 
			
			By the definition of subsets, $A \subseteq B$.
			
			\item For every $x \in A \cap B$, by definition, $x \in A$. For every $y \in A$, by Axiom 3.5, $y \in A \cup B$. By hypothesis, $y \in B$. Hence $y \in A \wedge y \in B$.
			
			Therefore, $A \cap B = A$.
		\end{enumerate}
	
		\item If $A \cap B = A$, then
		\begin{enumerate}
			\item For every $x \in A$, by hypothesis, $x \in A \cap B$. By the definition of intersection, $x \in B$. 
			
			By the definition of subsets, $A \subseteq B$.
			
			\item For every $x \in A \cup B$, by axiom 3.5, $x \in A$ or $B$. If $x \in A$, then by hypothesis, $x \in A \cap B$, and by the definition of intersection, $x \in B$. If $x \in B$, then $x \in B$.
			
			For every $y \in B$, by Axiom 3.5, $y \in A \cup B$.
			
			Therefore, $A \cup B = B$.
		\end{enumerate}
	\end{enumerate}
\end{proof}

\begin{exercise}{3.1.6}
	Let $A,B,C,X$ be any sets that $A,B,C \subseteq X$, then:
	\begin{enumabc}
		\item $A \cup \varnothing = A$, and $A \cap \varnothing = \varnothing$.
		\item $A \cup X = X$, and $A \cap X = A$.
		\item $A \cap A = A = A \cup A$.
		\item $A \cap B = B \cap A$, and $A \cup B = B \cup A$.
		\item $(A \cup B) \cup C = A \cup (B \cup C)$, and $(A \cap B) \cap C = A \cap (B \cap C)$.
		\item $A \cup (B \cap C) = (A \cup B) \cap (A \cup C)$, and $A \cap (B \cup C) = (A \cap B) \cup (A \cap C)$.
		\item $A \cup (X \setminus A) = X$, and $A \cap (X \setminus A) = \varnothing$. 
		\item $X \setminus (A \cup B) = (X \setminus A) \cap (X \setminus B)$, and $X \setminus (A \cap B) = (X \setminus A) \cup (X \setminus B)$.
	\end{enumabc}
\end{exercise}
\begin{proof}
	We might see that these properties of sets have a clear correspondence with some facts about logical statements, since many concepts in set theory are defined using logical primitives:
	\begin{enumabc}
		\item $p \vee \false \equiv p$, and $p \wedge \false \equiv \false$.
		
		\item $p \vee \true \equiv \true$, and $p \wedge \true \equiv p$.
		
		\item $p \wedge p \equiv p \equiv p \vee p$.
		
		\item $p \wedge q \equiv q \wedge p$, and $p \vee q \equiv q \vee p$.
		
		\item $(p \wedge q) \wedge r \equiv p \wedge (q \wedge r)$, and $(p \vee q) \vee r \equiv p \vee (q \vee r)$
		
		\item $p \wedge (q \vee r) \equiv (p \wedge q) \vee (p \wedge r)$, and $p \vee (q \wedge r) \equiv (p \vee q) \wedge (p \vee r)$.
		
		We see that the facts start getting not obvious. To prove them, one can use truth tables, but I will omit that here. In fact, the process of using truth tables, formally, corresponds to the way that we divide and conquer different situations to prove the corresponding statements about sets. The former approach just seems more systematic.
		
		For example, if we consider $x \in A$, we are also considering $p = \true$ and the lines in the truth table.
		
		\item $p \vee (\neg p) \equiv \true$, and $p \wedge (\neg p) \equiv \false$.
		
		\item $\neg(p \vee q) \equiv \neg p \wedge \neg q$, and $\neg (p \wedge q) \equiv \neg p \vee \neg q$.
	\end{enumabc}

	We will use these facts about logical statements to prove the proposition.
	\begin{enumabc}
		\item Given any $x$, let $p := x \in A$, and we have $\false \equiv x \in \varnothing$. We give an example of how to prove the statement and everything below is done similarly.
		
		For example, to prove $A \cup \varnothing = A$, we need to show that
		\[
			\forall x(x \in A \cup \varnothing \leftrightarrow x \in A)
		\]
		, which is, by Axiom 3.5,
		\[
			\forall x((x \in A \vee x \in \varnothing) \leftrightarrow x \in A)
		\]
		Substitute $p$ and $\false$ in, we have
		\[
			\forall x((p \vee \false) \leftrightarrow p)
		\]
		Which immediately follows from $p \vee \false \equiv p$.
		
		\item Given any $x$, 
		\begin{enumerate}
			\item If $x \in X$, then let $p := x \in A$, and $\true \equiv x \in X$.
			\item If $x \notin X$, then $x \in A \equiv x \in X \equiv \false$, and the two implication statements between them are all vacuously true.
		\end{enumerate}
	
		\item Given any $x$, let $p := x \in A$.
		
		\item Given any $x$, let $p := x \in A;\ q := x \in B$.
		
		\item Given any $x$, let $p := x \in A;\ q := x \in B;\ r := x \in C$.
		
		\item Given any $x$, let $p := x \in A;\ q := x \in B;\ r := x \in C$.
		
		\item There are two situations.
		\begin{enumerate}
			\item If $x \notin X$, then $x \in X \setminus \text{anything}$ is false, by the definition of the difference of sets. Therefore, all implication statements are vacuously true.
			
			\item If $x \in X$, then $x \in X \equiv \true; x \in \varnothing \equiv \false$. 
			
			In addition, for subset $A \subseteq X$, let $p := x \in A$. By definition of the difference of sets, $\neg p \equiv x \in X \setminus A$.
		\end{enumerate}
	
			\item There are two situations.
			\begin{enumerate}
				\item If $x \notin X$, then $x \in X \setminus \text{anything}$ is false, by the definition of the difference of sets. Therefore, all implication statements are vacuously true.
				
				\item If $x \in X$, then, let $p := x \in A$; $q := x \in B$. By the definition of the difference of sets, $\neg p \equiv x \in X \setminus A$; $\neg q \equiv x \in X \setminus B$.
			\end{enumerate}
	\end{enumabc}
\end{proof}

\begin{why}{35}
	Let $A = \{3,5,9\}$. Let $P(x,y)$ be $y = \successor{x}$. Then, $\{x:P(x,y)\ \text{for some}\ x \in A\} = \{4,6,10\}$.
\end{why}
\begin{proof}
	Clearly, $4,6,10$ are some $y$'s that satisfy $P(x,y)$ for some $x \in A$. Namely, $P(3,4), P(5,6), P(9,10)$ are true.
	
	Now we show that they are the only elements in $\{x:P(x,y)\ \text{for some}\ x \in A\}$. Let $e \in \{x:P(x,y)\ \text{for some}\ x \in A\}$, then, by Axiom 3.7, there exists $x \in A$ such that $P(x,e)$. $x$ can only be one of $3,5,9$, and because $P(x,y)$ is true for at most one $y$ for a $x$, we see that $4,6,10$ are the only possible numbers.
\end{proof}

\begin{why}{36}
	These two sets are exactly the same:
	\[
	\{8-n: n \in \setn, 0 \le n \le 5\}, \quad \{8-m: m \in \setn, 0 \le m \le 5\}
	\]
\end{why}
\begin{proof}
	We have used the Axiom of specification (first), and the Axiom of replacement (second) to create the two sets. In the progress, we see that $m$ and $n$ are merely placeholders for an element in $\setn$. So they are really irrelevant to the result.
\end{proof}

\begin{exercise}{3.1.7}
	\begin{enumerate}
		\item $A \cap B \subseteq A$ and $A \cap B \subseteq B$.
		\item $C \subseteq A \wedge C \subseteq B$ iff $C \subseteq A \cap B$.
		\item $A \subseteq A \cup B$ and $B \subseteq A \cup B$.
		\item $A \subseteq C \wedge B \subseteq C$ iff $A \cup B \subseteq C$.
	\end{enumerate}
\end{exercise}
\begin{proof}\leavevmode
	\begin{enumerate}
		\item Let $x$ by any object. If $x \in A \cap B$, then by definition $x \in A$ and $x \in B$. Hence, $A \cap B$ is a subset of both $A$ and $B$.
		
		\item If $C \subseteq A,B$, then for any element $x \in C$, $x \in A$ and $x \in B$, which by definition gives $x \in A \cap B$.
		
		If $C \subseteq A \cap B$, then for any element $x \in C$, $x \in A \cap B$. By definition, $x \in A$ and $x \in B$. $x \in A$ gives $C \subseteq A$, and $x \in B$ gives $C \subseteq B$.
		
		\item If $x \in A$, then $x \in A \vee x \in B$ is true. Then, $A \subseteq A \cup B$. Similarly, $B \subseteq A \cup B$.
		
		\item If $A,B \subseteq C$, then for any element $x \in A \cup B$, we have $x \in C$ either way.
		
		If $A \cup B \subseteq C$, then for any element $x \in A$, by the last item, $x \in A \cup B$. By the definition of subsets, $x \in C$. Similarly, $x \in B \to x \in C$.
	\end{enumerate}
\end{proof}

\begin{exercise}{3.1.8}
	\begin{enumerate}
		\item $A = A \cap (A \cup B)$.
		\item $A = A \cup (A \cap B)$.
	\end{enumerate}
\end{exercise}
\begin{proof}\leavevmode
	\begin{enumerate}
		\item Let $x \in A \cap (A \cup B)$. By definition, $x \in A$. 
		
		Let $x \in A$. By \exerciseref{3.1.7}, $x \in A \cup B$. Because we have both, by the definition of intersection, we have $x \in A \cap (A \cup B)$.
		
		\item Let $x \in A \cup (A \cap B)$. By definition, $x \in A \vee x \in (A \cap B)$. Therefore, if $x \notin A$, then $x$ must be in $A \cap B$, which shows that $x$ still $\in A$, a contradiction.
		
		Let $x \in A$, then by Axiom 3.5, $x \in A \cup (A \cap B)$. 
	\end{enumerate}
\end{proof}

\begin{exercise}{3.1.9}
	If $A \cup B = X$ and $A \cap B = \varnothing$, then
	\begin{enumerate}
		\item $A = X \setminus B$,
		\item and $B = X \setminus A$.
	\end{enumerate}
\end{exercise}
\begin{proof}\leavevmode
	\begin{enumerate}
		\item If $x \in A$, then $x \in A \cup B$, by Axiom 3.5. Then, $x \in X$. But $x \notin B$, because $A \cap B = \varnothing$ (so everyone in $A$ is not in $B$). Hence, $x \in X \setminus B$.
		
		If $x \in X \setminus B$, then $x \in X$, so $x \in A \cup B$, so $x \in A \vee X \in B$. But $x \notin B$ by the difference, then $x$ can only be in $A$.
		
		\item Using the commutativity of $\cap$ and $\cup$, the proof follows similarly.
	\end{enumerate}
\end{proof}

\begin{exercise}{3.1.10}
	\begin{enumerate}
		\item $A\setminus B$, $A\cap B$, and $B \setminus A$ are disjoint.
		\item The union of the three equals $A \cup B$.
	\end{enumerate}
\end{exercise}
\begin{proof}\leavevmode
	\begin{enumerate}
		\item If $x \in A \cap B$, then $x \in A \wedge x \in B$. This means, by definition, $x$ cannot be in any of the two differences between $A$ and $B$.
		
		If $x \in A \setminus B$, then $x \notin B$. But for $x$ to be in $B \setminus A$, $x$ must be in $B$. Hence, the two differences are also disjoint.
		
		\item Let $x \in A \cup B$. Consider $p := x \in A;\ q := x \in B$. Then,
		\begin{enumerate}
			\item If $p = \true, q = \true$, then $x \in A \cap B$, and is in the union of the three.
			\item If $p = \true, q = \false$, then $x \in A \setminus B$, and is in the union of the three.
			\item If $p = \false, q = \true$, then $x \in B \setminus A$, and is in the union of the three.
			\item If $p = \false, q = \false$, then it is a contradiction to our premise $x \in A \cup B$.
		\end{enumerate}
	
		Let $x$ be in the union of the three.
		\begin{enumerate}
			\item If $x \in A \setminus B$, then $x \in A$, then $x \in A \cup B$.
			\item If $x \in B \setminus A$, then $x \in B$, then $x \in A \cup B$.
			\item If $x \in A \cap B$, then $x \in A$, then $x \in A \cup B$.
		\end{enumerate}
	\end{enumerate}
\end{proof}

\begin{exercise}{3.1.11}
	The axiom of replacement implies the axiom of specification.
\end{exercise}
\begin{proof}
	Let $X$ be a set. Let $Q(x)$ be a property pertaining to any $x \in X$.
	
	Define $P(x,y)$ pertaining to any $x \in X$ and any $y$, such that
	\begin{enumerate}
		\item If $Q(x)$ is true, then $P(x,x)$ is true; and $P(x,y)$ is false for all $y \ne x$.
		\item If $Q(x)$ is false, then $P(x,y)$ is false for all $y$. In particular, $P(x,x)$ is false.
	\end{enumerate}

	Using the axiom of replacement, we see that the set $\{y: x \in X, P(x,y)\}$ exists.

	Now we show that it would be equal to what would be created by the axiom of specification, $\{x \in X: Q(x)\}$.
	\begin{enumerate}
		\item If $y \in \{y: x \in X, P(x,y)\}$, then by the axiom of replacement, there exists some $x \in X$ such that $P(x,y)$. If $Q(x)$, then we must have $x = y$. If $\neg Q(x)$, then a contradiction. Hence, $y = x$ for some $x \in X \wedge Q(x)$, which means $y \in \{x \in X: Q(x)\}$.
		
		\item If $y \in \{x \in X: Q(x)\}$, then $y \in X \wedge Q(y)$. By the definition of $P$, $P(y,y)$ is true, and $P(y,x)$ is false for all $x \ne y$. By the axiom of replacement, $y \in \{y: x \in X, P(x,y)\}$.
	\end{enumerate}
\end{proof}

\begin{exercise}{3.1.12}
	Suppose $A,B,A',B'$ are such sets that $A' \subseteq A, B' \subseteq B$. Then,
	\begin{enumerate}
		\item $A' \cup B' \subseteq A \cup B$. $A' \cap B' \subseteq A \cap B$.
		\item Show, using a counterexample, that $A' \setminus B' \subseteq A \setminus B$ is not the case. In addition, can you find a modification of this statement involving the set difference operation $\setminus$ that is true given the stated hypotheses? Justify your answer.
	\end{enumerate}
\end{exercise}
\begin{proof}\leavevmode
	\begin{enumerate}
		\item Let $x$ be any object. Let $p := x \in A,\ q := x \in B,\ p' := x \in A',\ q' := x \in B'$. Then, by the definition of subsets, we have $p' \to p$ and $q' \to q$.
		
		Hence, what we need to prove are immediate consequence of
		\[
		p' \vee q' \to p \vee q, \quad p' \wedge q' \to p \wedge q
		\]
		
		\item Let $A = \{1,2,3\}, B = \{2,3,4\}, A' = \{1,2\}, B' = \{3\}$. Then, $A \setminus B = \{1\}$, but $A' \setminus B' = \{1,2\}$.
		
		I discovered a good way to find a statement that works. Consider again the four statements $p,q,p',q'$ using the above definition. Then $A \setminus B$ and $A' \setminus B'$ correspond to, respectively,
		\[
		p \wedge \neg q,\ \text{and}\ p' \wedge \neg q'
		\]
		
		The problem here is, while $q' \to q$, it is not the case that $\neg q' \to \neg q$. So from $p' \wedge \neg q'$ we cannot imply $p \wedge \neg q$. (And neither can we imply the other way!)
		
		However, we do have $\neg p \to \neg p'$ and $\neg q \to \neg q'$. Therefore, to make an implication, we will have to swap one statement with another: now consider
		\[
		p \wedge \neg q',\ \text{and}\ p' \wedge \neg q
		\]
		Because $p' \to p$ and $\neg q \to \neg q'$, we have the latter implying the former. Hence, I propose
		\[
		A' \setminus B \subseteq A \setminus B'
		\]
		, whose proof follows the above reasoning.
	\end{enumerate}
\end{proof}

\begin{exercise}{3.1.13}
	A set $A$ does not have any non-empty proper subset iff $A$ is a singleton set.
\end{exercise}
\begin{proof}
	\fbox{\em If.} Suppose $A$ is a singleton set $\{x\}$. Let $B$ be a proper subset of $A$. Suppose for the sake of contradiction that $B$ is non-empty, that is, $\exists y \in B$. Then, by the definition of subsets, $y \in A$. Because $A$ is a singleton, we have $y = x$.
	
	This means that $x \in B$, or, in other words, $\{x\} \subseteq B$. Hence, we have $B \subseteq A \wedge A \subseteq B$, and it follows that $A = B$, a contradiction.
	
	\fbox{\em Only If.} Suppose that all proper subsets of $A$ are empty. By Why~\ref{why.unique.empty.set}, they are the same set: $\varnothing$. Because $\varnothing$ is a proper subset of $A$, we know that $A$ is non-empty.
	
	Hence, there exists some $x \in A$. We show that it is the only element in $A$. Actually, we have $\{x\} \subseteq A$. But $\{x\}$ cannot be a proper subset, as it is non-empty. Therefore, we must have $\{x\} = A$, as desired.
\end{proof}

\section{Russell's paradox}

\begin{exercise}{3.2.1}
	If we assumed the Axiom of universal specification, then we could have all the axioms 3.3--3.8, (for 3.8, assuming all natural numbers are objects).
\end{exercise}
\begin{proof}
\fbox{\em Axiom 3.3: The empty set.} To prove the existence of the empty set, simply choose a property that is false for all objects.

\fbox{\em Axiom 3.4: Singleton and pair sets.} Choose a property $P(x)$ that is true iff $x = a$ for the singleton set $\{a\}$, and iff $x = a \vee x = b$, for the pair set $\{a,b\}$.

\fbox{\em Axiom 3.5: Union sets.} Let $P(x) := x \in A \vee x \in B$ for $A \cup B$.

\fbox{\em Axiom 3.6: Specification.} Let the property be $P(x):= x \in A \wedge Q(x)$, for $\{x \in A : Q(x)\}$.

\fbox{\em Axiom 3.7: Replacement.} Let the property be $P(y):= \exists x(x \in A \wedge Q(x,y))$, for $\{y: x \in A: Q(x,y)\}$.

\fbox{\em Axiom 3.8: Infinity.} Let the property be $P(n):= (n \text{ is a natural number})$.
\end{proof}

\begin{exercise}{3.2.2}
	Let $A,B$ be sets. Then, (with the axiom of regularity)
	\begin{enumerate}
		\item $A \notin A$.
		\item Either $A \notin B$, or $B \notin A$.
	\end{enumerate}
\end{exercise}
\begin{proof}\leavevmode
	\begin{enumerate}
		\item Suppose for the sake of contradiction that $A \in A$ for some set $A$. This means that the set $\{A\} \subseteq A$. Now consider the set $\{A\}$.
		
		According to the axiom of regularity, its only element, is either not a set (excluded), or is disjoint from it, that is, $A \cap \{A\} = \varnothing$. But this is a contradiction, because we already have $\{A\} \subseteq A$, so there is some element (namely, $A$), that is contained in both $\{A\}$ and $A$.
		
		\item Suppose for the sake of contradiction that there exist such sets $A,B$ that $A \in B$ and $B \in A$. Now, we consider, somewhat similarly, the pair set $\{A,B\}$.
		
		Obviously, both elements in $\{A,B\}$ are sets. So one of them has to be disjoint from it. However,
		\begin{enumerate}
			\item $B \in A \cap \{A,B\}$
			\item $A \in B \cap \{A,B\}$ 	
		\end{enumerate}
		, a contradiction with the axiom of regularity.
	\end{enumerate}
\end{proof}

\begin{exercise}{3.2.3}
	(Assuming the other axioms of the set theory.) The axiom of universal specification is equivalent to the existence of a universal set $\Omega$ that contains all objects.
\end{exercise}
\begin{proof}
	\fbox{\em If.} If there exists such a set as $\Omega$, then for any property $P(x)$ pertaining to all objects $x$, we can construct the set for the axiom of universal specification using $\Omega$ and the axiom of specification.
	
	\fbox{\em Only If.} If the axiom of universal specification is assumed, then we can simply choose a property $P(x)$ which is true for all objects to have $\Omega$.
\end{proof}

\section{Functions}
\begin{why}{42}
	If $x'=x$, then $f(x')=f(x)$.
\end{why}
\begin{proof}
	The reason is, the property $P(x,y)$ obeys the axiom of substitution, and for the same $x$, the $y$ is unique.
\end{proof}

\begin{why}{43}
	For the same set $X$, all functions from $\varnothing$ to $X$ are equal.
\end{why}
\begin{proof}\leavevmode
	\begin{enumerate}
		\item The domains and codomains obviously match.
		\item $\forall x(x \in \varnothing \to (f(x) = f'(x)))$ is vacuously true.
	\end{enumerate}
\end{proof}

\begin{exercise}{3.3.1}
	\begin{enumerate}
		\item The equality of functions is reflexive, symmetric, and transitive.
		\item Let $f = \tilde{f}, g = \tilde{g}$ be functions. And $f,\tilde{f}: X \to Y$, $g,\tilde{g}: Y \to Z$. Then, $g \circ f = \tilde{g} \circ \tilde{f}$.
	\end{enumerate}
\end{exercise}
\begin{proof}\leavevmode
	\begin{enumerate}
		\item 
			\fbox{\em Reflexivity.} Let $f: X \to Y$ be any function. 
			
			Because of the reflexivity of the equality on sets, $X = X$ and $Y = Y$. Therefore, $f$ and $f$ have equal domain and codomains. Because of the reflexivity of objects, $f(x) = f(x)$, for all $x \in X$. By definition, $f = f$.
			
			\fbox{\em Symmetry.} Let $f: X \to Y,f': X' \to Y'$ be functions such that $f = f'$. 
			
			Then, by definition, $X = X', Y = Y'$, and $\forall x \in (X=X'), (f(x) = f'(x))$. Because of the symmetry of equality of objects (note that sets are also objects), $X' = X, Y' = Y$, and $\forall x \in (X=X'), (f'(x) = f(x))$. Therefore, by definition, $f' = f$.
			
			\fbox{\em Transitivity.} Let $f_0: X_0\to Y_0,\ f_1:X_1\to Y_1,\ f_2:X_2\to Y_2$ be such functions that $f_0 = f_1$ and $f_1 = f_2$.
			
			Then, by the definition of function equality, $X_0 = X_1 \wedge X_1 = X_2$, and $Y_0 = Y_1 \wedge Y_1 = Y_2$. By the transitivity of equality of objects, we have $X_0 = X_2$ and $Y_0 = Y_2$. Similarly, we can obtain $\forall x \in (X_0 = X_2), (f_0(x) = f_2(x))$. Therefore, by definition, we have $f_0 = f_2$.
		
		\item 
			By the reflexivity of the equality of sets, $X = X$ and $Z = Z$. Thus, the domains and codomains match.
			
			Now, let $x$ be any element in $X$. We show it is always true that $(g \circ f)(x) = (\tilde{g} \circ \tilde{f})(x)$.
			
			By the definition of composition, $(\tilde{g} \circ \tilde{f})(x) = \tilde{g}(\tilde{f}(x))$. By the definition of function equality, $\tilde{f}(x) = f(x)$. By the axiom of substitution on objects $\tilde{f}(x),f(x)$, we can then say $\tilde{g}(\tilde{f}(x)) = \tilde{g}(f(x))$. By the definition of function equality, we have $\tilde{g}(f(x)) = g(f(x))$. By the transitivity of object equality, we can then conclude $\tilde{g}(\tilde{f}(x)) = g(f(x))$, as desired.
	\end{enumerate}
\end{proof}

\begin{exercise}{3.3.2}
	Let $f: X \to Y, g: Y \to Z$ be functions.
	\begin{enumerate}
		\item If $f,g$ are both injective, then so is $g \circ f$.
		\item If $f,g$ are both surjective, then so is $g \circ f$.
	\end{enumerate}
\end{exercise}
\begin{proof}\leavevmode
	\begin{enumerate}
		\item Let $x, x'$ be any two elements in $X$. If $x \ne x'$, then by the definition of injectivity, $f(x) \ne f(x')$. By the definition again,
		$g(f(x)) \ne g(f(x'))$. This, by the definition of composition, means that $(g \circ f)(x) \ne (g \circ f)(x')$, as desired.
		
		\item Let $z$ be any element in $Z$. Because $g$ is surjective, there exists $y \in Y$ such that $g(y) = z$. Because $f$ is surjective, there exists $x \in X$ such that $f(x) = y$. 
		By the axiom of substitution, $z = g(y) = g(f(x))$, as desired.
	\end{enumerate}
\end{proof}

\begin{exercise}{3.3.3}
	When is the empty function into a given set $X$ injective, surjective, and bijective?
\end{exercise}
\begin{proof}
	Be careful how you intrepret the definition of injectivity. I didn't know how, so I had asked a \href{https://math.stackexchange.com/questions/3800240/how-to-interpret-the-definition-of-injectivity}{question} at Stack Exchange regarding this problem.

	\fbox{\em Injective.} For such a function $f: \varnothing \to X$ to be injective, we must have
	\[
	\forall x \forall x'\Big[(x \in \varnothing \wedge x' \in \varnothing) \to \big(x \ne x' \to f(x) \ne f(x')\big)\Big]
	\]
	, which is always vacuously true.
	
	\fbox{\em Surjective.} For such a function $f: \varnothing \to X$ to be surjective, we must have
	\[
	\forall y\Big[y \in X \to \big(\exists x(x \in \varnothing \wedge f(x) = y)\big)\Big]
	\]
	, which is false (because $x \in \varnothing$ is false), unless $y \in X$ is false. In that case, $X = \varnothing$.
	
	\fbox{\em Bijective.} Is true if $X = \varnothing$. Because only then can the function be surjective.
\end{proof}

I am tempted to discover and prove something stronger, because such boring situations involving these boring sets will be encountered a lot, later when we talk more about injectivity and surjectivity.
\begin{thm}\label{my.boring.inj.surj.thm}
	Let $f: X \to Y$ be a function, then
	\begin{enumerate}
		\item If $X = \varnothing$ or is a singleton $\{x\}$, then $f$ must be injective.
		\item If $Y = \varnothing$ or is a singleton $\{y\}$, then $f$ must be surjective.
	\end{enumerate}
\end{thm}
\begin{proof}\leavevmode
	\begin{enumerate}
		\item If $X = \varnothing$, then it's already proven in \exerciseref{3.3.3}.
		
		If $X$ is a singleton set, then for $x,x' \in X$, $x \ne x' \to f(x) \ne f(x')$ is always vacuously true. Thus, $f$ is injective.
		
		\item If $Y = \varnothing$, then $y \in Y \to \exists x(\dots)$ is always vacuously true.
		
		If $Y = \{y\}$, then we have two situations.
		\begin{enumerate}
			\item If $X = \varnothing$, then as shown in \exerciseref{3.3.3}, $f$ is surjective.
			\item If $X \ne \varnothing$, then there exists some $x \in X$. By the definition of functions, $f(x)$ must be in $Y$. Because $Y$ is a singleton, we must have $f(x) = y$. Because $Y$ is a singleton again, its only single element is mapped to by $f$, and thus $f$ is surjective.
		\end{enumerate}
	\end{enumerate}
\end{proof}

\begin{exercise}{3.3.4}
	Let $f,\tilde{f}: X \to Y$, $g,\tilde{g}: Y \to Z$ be functions. Then,
	\begin{enumerate}
		\item If $g \circ f = g \circ \tilde{f}$ and $g$ is injective, then $f = \tilde{f}$.
		\item If $g \circ f = \tilde{g} \circ f$ and $f$ is surjective, then $g = \tilde{g}$.
	\end{enumerate}
\end{exercise}
\begin{proof}\leavevmode
	The domains and codomains always match by the premise. We only need to prove that that function values equal.
	\begin{enumerate}
		\item By the definition of function equality, for all $x \in X$, $(g \circ f)(x) = (g \circ \tilde{f})(x)$. By the definition of composition, $g(f(x)) = g(\tilde{f}(x))$. Because $g$ is injective, we must have $f(x) = \tilde{f}(x)$ for all $x \in X$.
		
		\item To show that $g = \tilde{g}$, we have to show that $g(y) = \tilde{g}(y)$ for all $y \in Y$. Note that now the variable takes value in $Y$, instead of in $X$. 
		
		Suppose for the sake of contradiction that $g \ne \tilde{g}$, then, (because the domains and codomains already match,) there must exist at least one $y\in Y$ such that $g(y) \ne \tilde{g}(y)$. Because $f$ is surjective, we must also have some $x \in X$ such that $f(x) = y$. 
		
		Now consider $(g \circ f)(x)$ and $(\tilde{g}\circ f)(x)$, they equal to, by the definition of composition, $g(y)$ and $\tilde{g}(y)$, respectively. But they should be equal by the premise, a contradiction.
		
		Note that proof by contradiction is used here but not above. This is because we can apply injectivity nicely from the opposite direction of function composition above, but we cannot apply surjectivity  the same way down here, since it tells nothing about equality. Hence, if one way is hard to walk, then we tend to another.
	\end{enumerate}

	We cannot have each desired statement if we do not assume injectivity/\discretionary{}{}{}surjectivity:
	\begin{enumerate}
		\item If $g$ is not injective, then there can be some $y \ne y' \in Y$ such that $g(y) = g(y')$. If for some $x$, $f(x) = y$ while $\tilde{f}(x) = y'$, then we can still have the composition equal.
		
		\item If $f$ is not surjective, then there exists some $y \in Y$ that is not mapped to by $f$. For these $y$'s, $g(y)$ and $\tilde{g}(y)$ can be different without affecting its composition with $f$.
	\end{enumerate}
\end{proof}

\begin{exercise}{3.3.5}
	Let $f: X \to Y$, $g: Y \to Z$ be functions. Show that,
	\begin{enumerate}
		\item If $g \circ f$ is injective, then so must $f$.
		\item If $g \circ f$ is surjective, then so must $g$.
	\end{enumerate}
\end{exercise}
\begin{proof}\leavevmode
	\begin{enumerate}
		\item First, if $X = \varnothing$ or $\{x\}$, then $f$ must be injective, by My Theorem~\ref{my.boring.inj.surj.thm}.
		
		Suppose for the sake of contradiction that $f$ is not injective. Then, there must exist such $x,x' \in X$ that $x \ne x'$ but $f(x) = f(x')$. By the axiom of substitution, this implies that $g(f(x)) = g(f(x'))$, a contradiction with our premise.
		
		\item This time we don't have to exclude the situation, nor do we have to use consider the special situations. Just expand the definition of composition to get that $g(f(x))$ can reach every element in $Y$, so in particular $g$ has to be able to do that.
	\end{enumerate}
\end{proof}

\begin{exercise}{3.3.6}
	Let $f: X\to Y$ be a bijective function, and $f^{-1}: Y \to X$ be its inverse. Then,
	\begin{enumerate}
		\item For all $x \in X$, $f^{-1}(f(x)) = x$; for all $y \in Y$, $f(f^{-1}(y)) = y$.
		\item $f^{-1}$ is invertible. And $(f^{-1})^{-1} = f$.
	\end{enumerate}
\end{exercise}
\begin{proof}\leavevmode
	\begin{enumerate}
		\item For all $x \in X$, by the definition of functions, there exists a unique $y \in Y$ that $f(x) = y$. By the definition of $f^{-1}$, $f^{-1}(y) = x$. By the axiom of substitution, $f^{-1}(f(x)) = x$.
		
		For all $y \in Y$, by the bijectivity of $f$, there exists a unique $x$ such that $f(x) = y$. By the definition of $f^{-1}$, we have $f^{-1}(y) = x$. By the axiom of substitution, $f(f^{-1}(y)) = x$.
		
		\item To show that $f^{-1}$ is invertible, we need to show that it's bijective.
		
		Let $y,y' \in Y$ be two elements. (If $Y$ is empty then $f^{-1}$ is injective, by My Theorem~\ref{my.boring.inj.surj.thm}.) Denote $x := f^{-1}(y)$ and $x' = f^{-1}(y')$. Suppose they are equal. By the axiom of substitution, we then have $f(x) = f(x')$. By the definition of the inverse, we see that $f(x) = y$ and $f(x') = y'$, then $y = y'$. Hence, $f^{-1}$ is injective.
		
		Now we show that it is surjective. Let $x \in X$ be any element in $X$. (If $X$ is empty, then $f^{-1}$ is surjective, by My Theorem~\ref{my.boring.inj.surj.thm}.) Then $y := f(x) \in Y$, by the definition of functions. By the definition of $f^{-1}$, we must have $f^{-1}(y) = x$, as desired.
	\end{enumerate}
\end{proof}

\begin{exercise}{3.3.7}
	Let $f: X \to Y$, $g: Y \to Z$ be functions. If $f,g$ are both bijective, then
	\begin{enumerate}
		\item So is $g \circ f$.
		\item $(g \circ f)^{-1} = f^{-1} \circ g^{-1}$.
	\end{enumerate}
\end{exercise}
\begin{proof}\leavevmode
	\begin{enumerate}
		\item If $X$ is empty, then by My Theorem~\ref{my.boring.inj.surj.thm}, $g \circ f$ is injective. Otherwise, let $x,x'$ be any elements in $X$. Suppose $(g \circ f)(x) = (g \circ f)(x)$, then $g(f(x)) = g(f(x'))$, by the definition of composition. By the injectivity of $g$, we must have $f(x) = f(x')$. By the injectivity of $f$, we musth then have $x = x'$. Hence, $g \circ f$ is injective.
		
		If $Z$ is empty, then by My Theorem~\ref{my.boring.inj.surj.thm}, $g \circ f$ is surjective. Otherwise, let $z$ be any element in $Z$. By the surjectivity of $g$, there exists $y \in Y$ such that $g(y) = z$. By the surjectivity of $f$, there exists such $x \in X$ that $f(x) = y$. By the axiom of substitution, $g(f(x)) = z$. By the definition of composition, $g \circ f(x) = z$. Hence, $g \circ f$ is surjective.
		
		\item First, we have $g^{-1}: Z \to Y$ and $f^{-1}: Y \to X$, by the definition of inverses. By the definition of composition, $f^{-1} \circ g^{-1}$ is from $Z$ to $X$, the same as $(g \circ f)^{-1}$.
		
		Now, we show that their values are also equal. Let $z$ be any element in $Z$. Denote $g^{-1}(z) = y$ and $f^{-1}(y) = x$, and we have $(f^{-1} \circ g^{-1})(z) = f^{-1}(g^{-1}(z)) = x$.
		
		By the definition of inverses, we must have $f(x) = y$ and $g(y) = z$, which means, by the axiom of substitution, $g(f(x)) = z$, that is, $(g \circ f)(x) = z$. By the definition of inverses, this means that $(g \circ f)^{-1}(z) = x$. As $z$ is any element in $Z$, the two functions are equal.
	\end{enumerate}
\end{proof}

\begin{exercise}{3.3.8}
	If $X \subseteq Y$ are such sets, then let $\iota_{X \to Y}$ be defined as $X \to Y,\ x \mapsto x$. Then,
	\begin{enumerate}
		\item If $X \subseteq Y \subseteq Z$, then $\iota_{Y \to Z} \circ \iota_{X \to Y} = \iota_{X \to Z}$.
		\item Let $f$ be any function from $A$ to $B$. Then, $f = f \circ \iota_{A \to A} = \iota_{B \to B} \circ f$.
		\item If $f: A \to B$ is bijective, then $f \circ f^{-1} = \iota_{B \to B}$, and $f^{-1} \circ f = \iota_{A \to A}$.
		\item Let $X,Y$ be disjoint sets. Let $f: X \to Z$ and $g: Y \to Z$ be functions. Then, there exists a \emph{unique} function $h: X \cup Y \to Z$ such that $f = h \circ \iota_{X \to X \cup Y}$ and $g = h \circ \iota_{X \to X \cup Y}$.
		\item We can remove the limitation for $X,Y$ to be disjoint, if we have $f(x) = g(x)$ for all $x \in X \cap Y$.
	\end{enumerate}
\end{exercise}
\begin{proof}\leavevmode
	\begin{enumerate}
		\item First, it is clear that they have the same domain and codomain.
		
		To show that they are equal, consider all $x \in X$ and the value 
		\begin{align*}
			&(\iota_{Y \to Z} \circ \iota_{X \to Y})(x) \\
			&= \iota_{Y \to Z}(\iota_{X \to Y}(x)) 		&\text{def.~of composition}\\
			&= \iota_{Y \to Z}(x) 						&\text{def.~of}\ \iota_{X \to Y}\\
			&= x 										&\text{def.~of}\ \iota_{Y \to Z}\\
			&= \iota_{X \to X}(x)						&\text{def.~of}\ \iota_{X \to X}
		\end{align*}
		, as desired.
		
		\item First, it is clear that they have the same domain and codomain.
		
		To show that they are equal, consider all $a \in A$ and the value
		\begin{align*}
			&(f \circ \iota_{A \to A})(a) \\
			&= f(\iota_{A \to A}(a)) 		&\text{def.~of composition}\\
			&= f(a) 						&\text{def.~of}\ \iota_{A \to A}\\
			&= \iota_{B \to B}(f(a))		&\text{def.~of}\ \iota_{B \to B}\\
			&= (\iota_{B \to B} \circ f)(a)	&\text{def.~of composition}\\
		\end{align*}
		, as desired.
		
		\item First, it is clear that the domains and codomains match.
		
		To show that $f^{-1} \circ f = \iota_{A \to A}$, consider, for all $x \in A$, the value $(f^{-1} \circ f)(x)$. It equals to $x$ by \exerciseref{3.3.6}, and thus equals to $\iota_{A \to A}(x)$ by definition, as desired.
		
		To show that $f \circ f^{-1} = \iota_{B \to B}$, consider, for all $y \in B$, the value $(f \circ f^{-1})(y)$. It equals to $y$ by \exerciseref{3.3.6}, and thus equals to $\iota_{B \to B}(y)$ by definition, as desired.
		
		\item \fbox{\em Existence.} Define $h: X \cup Y \to Z$ as
		\[
		\begin{cases}
			a \mapsto f(a) &a \in X \\
			a \mapsto g(a) &a \in Y
		\end{cases}
		\]
		Now we show that it is a valid definition of a function. To show this, we need to show that for all $a \in X \cup Y$, there is a unique value $b \in Z$ such that $h(a) = b$.
		
		By the definition of $X \cup Y$, $a$ is either in $X$ or $Y$, but not both, since they are disjoint. If $a \in X$, then there is a value, namely $f(a)$. But because $f$ is a function, the value is unique. Similarly, we can show the statement when $a \in Y$.
		
		\fbox{\em Uniqueness.} Now we know that the definition gives a valid function, we show that it is unique. Suppose that we define another function $h'$ using the same definition. We will show that $h = h'$.
		
		Obviously, they have the same domain and codomain by definition. Now we consider, for all $a \in X \cup Y$, their values. By definition, if $a \in X$, then $h(a) = f(a) = h'(a)$. Similarly, if $a \in Y$, then $h(a) = g(a) = h'(a)$. Because $a$ is either in $X$ or $Y$, we have verified all the possible $a$'s, and hence, $h = h'$, as desired.
		
		\fbox{\em The two compositions.} Now we consider
		\begin{enumerate}
			\item $h \circ \iota_{X \to X \cup Y}$ and $f$. Obviously, they have the same domain and codomain.
			
			For all $x \in X$, 
			\begin{align*}
				&(h \circ \iota_{X \to X \cup Y})(x) \\
				&= h(\iota_{X \to X \cup Y}(x)) 	&\text{def.~of composition}\\
				&= h(x) 							&\text{def.~of}\ \iota_{X \to X \cup Y}\\
				&= f(x)								&\text{def.~of}\ h \wedge x \in X
			\end{align*}
			, as desired.
			
			\item $h \circ \iota_{Y \to X \cup Y}$ and $g$. Obviously, they have the same domain and codomain.
			
			For all $y \in Y$, 
			\begin{align*}
				&(h \circ \iota_{Y \to X \cup Y})(y) \\
				&= h(\iota_{Y \to X \cup Y}(y)) 	&\text{def.~of composition}\\
				&= h(y) 							&\text{def.~of}\ \iota_{Y \to X \cup Y}\\
				&= g(y)								&\text{def.~of}\ h \wedge y \in Y
			\end{align*}
			, as desired.
		\end{enumerate}
		
		\item \fbox{\em Existence.} Define $h: X \cup Y \to Z$ as
		\[
		\begin{cases}
			a \mapsto f(a), &a \in X \\
			a \mapsto f(a), &a \in X \cap Y \\
			a \mapsto g(a), &a \in Y
		\end{cases}
		\]
		Now we show that it is a valid definition of a function. To show this, we need to show that for all $a \in X \cup Y$, there is a unique value $b \in Z$ such that $h(a) = b$.
		
		This time, we have to consider different sets of values. We let $\Omega := X \cup Y$. By \exerciseref{3.1.7} (3), $X,Y \subseteq \Omega$. Hence, by \exerciseref{3.1.6} (g), $X \cup (\Omega \setminus X) = \Omega$.
		
		Now we consider two situations,
		\begin{enumerate}
			\item When $a \in X$. Then, by definition, $h(a) = f(a)$, whether $a \in X \cap Y$ or not.
			\item When $a \in \Omega \setminus X$. Then, $a$ has to be in $Y$, because $\Omega = X \cup Y$. By definition, $h(a) = g(a)$.
		\end{enumerate}
		Using the same reasoning as above, we see that for each $a$, $h(a)$ is unique. Hence, this is a valid definition of a function.
		
		\fbox{\em Uniqueness.} Now we know that the definition gives a valid function, we show that it is unique. Suppose that we define another function $h'$ using the same definition. We will show that $h = h'$.
		
		Obviously, they have the same domain and codomain by definition. Now we consider, for all $a \in X \cup Y$, their values. 
		
		By definition, if $a \in X$, then $h(a) = f(a) = h'(a)$. Similarly, if $a \in \Omega \setminus X$, then $h(a) = g(a) = h'(a)$. Because $a$ is either in $X$ or $\Omega \setminus X$, we have verified all the possible $a$'s, and hence, $h = h'$, as desired.
		
		\fbox{\em The two compositions.} Now we consider
		\begin{enumerate}
			\item $h \circ \iota_{X \to X \cup Y}$ and $f$. Obviously, they have the same domain and codomain.
			
			For all $x \in X$, 
			\begin{align*}
				&(h \circ \iota_{X \to X \cup Y})(x) \\
				&= h(\iota_{X \to X \cup Y}(x)) 	&\text{def.~of composition}\\
				&= h(x) 							&\text{def.~of}\ \iota_{X \to X \cup Y}\\
				&= f(x)								&\text{def.~of}\ h \wedge x \in X
			\end{align*}
			, as desired.
			
			\item $h \circ \iota_{Y \to X \cup Y}$ and $g$. Obviously, they have the same domain and codomain.
			
			For all $y \in Y$, 
			\begin{align*}
				&(h \circ \iota_{Y \to X \cup Y})(y) \\
				&= h(\iota_{Y \to X \cup Y}(y)) 	&\text{def.~of composition}\\
				&= h(y) 							&\text{def.~of}\ \iota_{Y \to X \cup Y}\\
				&= g(y) \text{ or } f(y)			&\text{def.~of}\ h \wedge y \in Y\\
				&= g(y)								&x \in X \cap Y \to h(x) = g(x)
			\end{align*}
			, as desired.
		\end{enumerate}
	\end{enumerate}
\end{proof}

\section{Images and inverse Images}
\begin{why}{47}
	The image of a function can be defined by using the axiom of specification, instead of the axiom of replacement.
\end{why}
\begin{proof}
	Let $f: X \to Y$ be any function. Let $P(x,y)$ be the property bound to $f$. Let $S$ be a subset of $X$. Define a property $Q(y)$ pertaining to elements in $Y$ this way:
	\[
	Q_S(y) := \exists x (x \in S \wedge P(x,y))
	\]
	And we define $f(S)_1 := \{y \in Y : Q_S(y)\}$.
	
	By the axiom of specification, the set $f(S)_1$ is well defined. Now we show that it equals to that set defined using the axiom of replacement: $f(S)_0 := \{x \in S : f(x)\}$.
	
	For any $y \in f(S)_0$, it must be replaced from some $x \in S$. By the definition of $f(S)_1$, we have $y \in f(S)_1$.
	
	For any $y \in f(S)_1$, by the definition of it, there exists a $x \in S$ such that $P(x,y)$ is true. By the definition of $f$, we must have $f(x) = y$. By the axiom of replacement, this $y$ is thus in $f(S)_0$.
\end{proof}

\begin{why}{48}
	Let $f: X \to Y$ be any function. Let $S$ be a subset of $X$. Then,
	\[
	y \in f(S) \leftrightarrow \exists x (x \in S \wedge y = f(x))
	\]
\end{why}
\begin{proof}
	Recall how we defined $f(S)$ using the axiom of replacement and showed that it is indeed the same set defined using the axiom of specification: $f(S) := \{y \in Y : Q_S(y)\}$, where
	\[
		Q_S(y) := \exists x (x \in S \wedge P(x,y))
	\]
	
	By the definition of functions, $P(x,y)$ iff $y = f(x)$. Thus,
	\[
		Q_S(y) \leftrightarrow \exists x (x \in S \wedge y = f(x))
	\]
	, as desired.
\end{proof}

\begin{why}{48}\label{why.func.image.surj}
	Let $f: \setn \to \setn, x \mapsto 2x$. We have
	\[
	f(f^{-1}(\{1,2,3\})) \ne \{1,2,3\}
	\]
\end{why}
\begin{proof}
	According to Prof.~Tao, $f^{-1}(\{1,2,3\}) = \{1\}$. But $f(\{1\}) = \{2\} \ne \{1,2,3\}$. 
\end{proof}

The problem is that this $f$ is not surjective, and $\{1,2,3\} \subsetneq f(\setn)$. But what if I restrict a function $f: X \to Y$ to be surjective --- will $f(f^{-1}(T)) = T$ for all $T \subseteq Y$? Let's see.

\begin{prop}\label{my.surj.image.prop}
	Let $f: X \to Y$ be any surjective function. Let $T$ be a subset of $Y$, then we have
	\[
		f(f^{-1}(T)) = T
	\]
\end{prop}
\begin{proof}
	Suppose for the sake of contradiction that $f(f^{-1}(T)) \ne T$, then we have two situations:
	\begin{enumerate}
		\item There exists $t \in T$ such that $t \notin f(f^{-1}(T))$. By the definition of inverse images, $f^{-1}(T)$ contains this set $S = \{x \in X: f(x) = t\}$. Because $f$ is surjective, and $X$ is the whole domain, $S$ is non-empty. 
		
		By the definition of forward images, we then must have $t$ in $f(f^{-1}(T))$, mapped to by some element in $S$, a contradiction.
		
		\item There exists $t \in f(f^{-1}(T))$ such that $t \notin T$. By the definition of forward images, there exists some element $s \in f^{-1}(T)$ such that $f(s) = t$. By the definition of inverse images, there must exist some element $t' \in T$, such that $t' = f(s) = t$, which means $t \in T$, a contradiction.
	\end{enumerate}
	Therefore, $f(f^{-1}(T)) = T$, as desired.
	
	Note that surjectivity is not used at all in the second situation. Therefore, in all cases, we must have $f(f^{-1}(T)) \subseteq T$.
\end{proof}

Surprisingly, for $f(f^{-1}(T)) = T$ we generally don't need $f$ to be bijective. It is not the case for a function values, though --- $f$ must be bijective for its inverse function to exist!

Symmetrical to the above Why~\ref{why.func.image.surj}, Prof.~Tao asked this:
\begin{why}{49}
	Let $f: \setz \to \setz$ by defined as $x \mapsto x^2$. We have
	\[
		f^{-1}(f(\{-1,0,1,2\})) \ne \{-1,0,1,2\}
	\]
\end{why}
\begin{proof}
This is because 
\[
	f^{-1}(f(\{-1,0,1,2\})) = f^{-1}(\{1,0,4\}) = \{-1,1,0,2,-2\}
\].
\end{proof}

The problem here is that, $f$ is not injective, so an element in $f(S)$ may correspond to multiple elements in $f^{-1}(f(S))$.

Symmetrical to the above My Proposition~\ref{my.surj.image.prop}, we can show that $f^{-1}(f(S)) = S$, if $f$ is injective.
\begin{prop}\label{my.inj.image.prop}
	Let $f: X \to Y$ be any injective function. Let $S$ be a subset of $X$, then we have
	\[
		f^{-1}(f(S)) = S
	\]
\end{prop}
\begin{proof}
	Suppose for the sake of contradiction that $f^{-1}(f(S)) \ne S$. Then we have two situations,
	\begin{enumerate}
		\item $\exists s \in S$ such that $s \notin f^{-1}(f(S))$. By the definition of forward images, we have $f(s) \in f(S)$. By the definition of inverse images, we must have $s \in f^{-1}(f(S))$, because $f(s)$ results in an element in $f(S)$, whether $f$ is injective or not. This is a contradiction.
		
		\item $\exists s \in f^{-1}(f(S))$ such that $s \notin S$. This is where we will need the injectivity of $f$. By the definition of inverse images, we have $f(s) \in f(S)$. By the injectivity, $s$ is the only element in $X$ for which $f$ equals $f(s)$. By our assumption, $s \notin S$, then we cannot have $f(s) \in f(S)$, a contradiction.
	\end{enumerate}
	Therefore, $f^{-1}(f(S)) = S$, as desired.

	Note that injectivity is not used at all in the second situation. Therefore, in all cases, we must have $S \subseteq f^{-1}(f(S))$.
\end{proof}

Putting My Proposition~\ref{my.surj.image.prop} and My Proposition~\ref{my.inj.image.prop} together, we have the following theorem.
\begin{thm}\label{my.inj.surj.image.thm}
	Let $f: X \to Y$ be any function, $S$ be a subset of $X$, and $T$ be a subset of $Y$, then we have
	\begin{enumerate}
		\item $S \subseteq f^{-1}(f(S))$.
		\item $f(f^{-1}(T)) \subseteq T$.
	\end{enumerate}

	If $f$ is additionally injective, then we also have $S = f^{-1}(f(S))$. If $f$ is additionally surjective, then we also have $f(f^{-1}(T)) = T$. So, if $f$ is bijective, then we have both, a result similar to that in \exerciseref{3.3.6}.
\end{thm}

\begin{exercise}{3.4.1}
	Let $f: X \to Y$ be function, and let $V$ be a subset of $Y$. If $f$ is bijective, then we can treat $f^{-1}(V)$ as both the forward image of $f^{-1}$ and the inverse image of $f$. Show that they are the same.
\end{exercise}
\begin{proof}
	Let $S$ denote the forward image, that is, $S = \{y \in V: f^{-1}(y)\}$, by the axiom of replacement. Let $T$ denote the inverse image, that is, $T = \{x \in X: \exists y (y \in V \wedge f(x) = y)\}$, by the axiom of specification. We show that $S = T$.
	
	For all $s \in S$, by its definition, there is a $y \in V$ that is replaced to it: $s = f^{-1}(y)$. By the definition of inverse functions, $f(s) = y$. This matches the condition in the definition of $T$, so we must have $s \in T$.
	
	For all $t \in T$, by its definition, there exists a $y \in V$ such that $f(t) = y$. By the definition of inverse functions, $f^{-1}(y) = t$. Because $y \in V$, it has to be replaced in the definition of $S$, to $t$. Therefore, $t \in S$.
\end{proof}

\begin{exercise}{3.4.2}
	Let $f: X \to Y$ be a function. Let $S \subseteq X$, and $U \subseteq Y$.
	\begin{enumerate}
		\item What can we say about $f^{-1}(f(S))$ and $S$?
		\item What can we say about $f(f^{-1}(U))$ and $U$?
		\item What about $f^{-1}(f(f^{-1}(U)))$ and $f^{-1}(U)$?
	\end{enumerate}
\end{exercise}
\begin{proof}\leavevmode
	\begin{enumerate}
		\item According to My Theorem~\ref{my.inj.surj.image.thm}, generally we have $S \subseteq f^{-1}(f(S))$. And if $f$ is injective, then the $\subseteq$ becomes $=$.
		
		\item According to My Theorem~\ref{my.inj.surj.image.thm}, generally we have $f(f^{-1}(U)) \subseteq U$. And if $f$ is surjective, then the $\subseteq$ becomes $=$.
		
		\item According to (1), generally, at least we should have $f^{-1}(U) \subseteq f^{-1}(f(f^{-1}(U)))$, if we regard $f^{-1}(U)$ as a subset of $X$. However, since Prof.~Tao mentioned this in particular, I wonder if I can prove something stronger, that is, if they are always equal in general.
		
		Now I only need to prove that $f^{-1}(f(f^{-1}(U))) \subseteq f^{-1}(U)$.
		
		For all $x \in f^{-1}(f(f^{-1}(U))$, by the definition of inverse images, there exists a $y \in f(f^{-1}(U))$ such that $f(x) = y$. Because $f(f^{-1}(U)) \subseteq U$, according to (2), we must also have $y \in U$. By the definition of inverse images again, we must have $x \in f^{-1}(U)$, since $f(x) = y \in U$.
		
		Therefore, in general we can also say $f^{-1}(f(f^{-1}(U))) = f^{-1}(U)$.
	\end{enumerate}
\end{proof}

\begin{exercise}{3.4.3}
	Let $A,B \subseteq X$ be sets. Let $f: X \to Y$ be a function. Then,
	\begin{enumerate}
		\item $f(A \cap B) \subseteq f(A) \cap f(B)$.
		\item $f(A) \setminus f(B) \subseteq f(A \setminus B)$.
		\item But $f(A \cup B) = f(A) \cup f(B)$.
	\end{enumerate}
\end{exercise}
\begin{proof}\leavevmode
	\begin{enumerate}
		\item For all $y \in f(A \cap B)$, we have $\exists x(x \in A \cap B \wedge f(x) = y)$. For such $x$, we have $x \in A$, thus $f(x) \in f(A)$, and $x \in B$, thus $f(x) \in f(B)$. Hence $y = f(x) \in f(A) \cap f(B)$, as desired.
		
		Note that the converse isn't generally true. If there exists $x \in A \setminus B, x' \in B \setminus A$ such that $x \ne x'$ and $f(x) = f(x')$, then we see that $x,x' \notin A \cap B$. Suppose that in addition that no other elements in $X$ can result in the value of $f = f(x)$, then we see that $f(A) \cap f(B)$ will contain $f(x)$, but $f(A \cap B)$ will not.
		
		\item For all $y \in f(A) \setminus f(B)$, we must have $y$ satisfy
		\[
		\exists x(x \in A \wedge f(x) = y) \wedge \forall x(x \in B \to f(x) \ne y)
		\]
		
		Let $x_0$ be some element that satisfies the $\exists$ statement. We have two situations:
		\begin{enumerate}
			\item $x_0 \in B$. Then, according to the $\forall$ statement, we have $f(x_0) \ne y$, a contradiction. Therefore, this situation is not possible.
			\item $x_0 \notin B$. Then, $x \in A \setminus B$, and by definition, $y = f(x) \in f(A \setminus B)$.
		\end{enumerate}
	
		The converse, generally is still not true. For a $x \in A \setminus B$, $f(x)$ might still equal to $f(x')$ for some $x' \in B$, if $f$ is not injective.
	
		\item For all $y \in f(A \cup B)$, we have
		\begin{equation}
			\exists x(x \in A \cup B \wedge f(x) = y) \label{eq.1.exer.3.4.3}
		\end{equation}
	
		For all $y \in f(A) \cup f(B)$, we have
		\begin{equation}
			\exists x(x \in A \wedge f(x) = y) \vee \exists x(x \in B \wedge f(x) = y) \label{eq.2.exer.3.4.3}
		\end{equation}
		
		We have to show the two logical statements are equivalent.
		\begin{enumerate}
			\item If $x$ satisfies \eqref{eq.1.exer.3.4.3}, then $x \in A \vee x \in B$, and $f(x) = y$. Therefore, we see that \eqref{eq.2.exer.3.4.3} can be satisfied.
			
			\item If $x$ satisfies \eqref{eq.2.exer.3.4.3}, then we have $x \in A \wedge f(x) = y$ or $x \in B \wedge f(x) = y$. Either case, we have $(x \in A \vee x \in B) \wedge f(x) = y$, that is, $x \in A \cup B \wedge f(x) = y$, satisfying \eqref{eq.1.exer.3.4.3}.
		\end{enumerate}
	\end{enumerate}

	In fact, let $P(x), Q(x)$ be predicates indicating that $x \in A$ and $x \in B$, respectively. And let $R(x)$ indicate $f(x) = y$. Then, we are essentially proving
	\begin{enumerate}
		\item $\exists x(P(x) \wedge Q(x) \wedge R(x)) \to \big[\exists x(P(x) \wedge Q(x)) \wedge \exists x(P(x) \wedge R(x))\big]$.
		\item $\big[\exists x(P(x) \wedge R(x)) \wedge \nexists x(Q(x) \wedge R(x))\big] \to \exists x(P(x) \wedge \neg Q(x) \wedge R(x))$.
		\item $\exists x((P(x) \vee Q(x)) \wedge R(x)) \equiv \big[\exists x(P(x) \wedge R(x)) \vee \exists x (P(x) \wedge R(x))\big]$.
	\end{enumerate}
\end{proof}

\begin{exercise}{3.4.4}
	Let $f: X \to Y$ be a function. Let $U,V$ be subsets of $Y$. Then,
	\begin{enumerate}
		\item $f^{-1}(U \cup V) = f^{-1}(U) \cup f^{-1}(V)$.
		\item $f^{-1}(U \cap V) = f^{-1}(U) \cap f^{-1}(V)$.
		\item $f^{-1}(U \setminus V) = f^{-1}(U) \setminus f^{-1}(V)$.
	\end{enumerate}
\end{exercise}
\begin{proof}
	For all $x \in X$, by definition $x \in f^{-1}(T)$ iff $f(x) \in T$. Therefore, we can translate them as:
	\begin{enumerate}
		\item $f(x) \in U \cup V$ versus $f(x) \in U \vee f(x) \in V$, which by the definition of pairwise union, are the same.
		\item $f(x) \in U \cap V$ versus $f(x) \in U \wedge f(x) \in V$, which by the definition of intersection, are the same.
		\item $f(x) \in U \setminus V$ versus $f(x) \in U \wedge \neg(f(x) \in V)$, which by the definition of set difference, are the same.
	\end{enumerate}
\end{proof}

\begin{exercise}{3.4.5}
	Let $f: X \to Y$ be a function. Then,
	\begin{enumerate}
		\item $\forall S(S \subseteq Y \to f(f^{-1}(S)) = S)$ iff $f$ is surjective.
		\item $\forall S(S \subseteq X \to f^{-1}(f(S)) = S)$ iff $f$ is injective.
	\end{enumerate}
\end{exercise}
\begin{proof}\leavevmode
	\begin{enumerate}
		\item \fbox{\em If.} An immediate consequence of My Theorem~\ref{my.inj.surj.image.thm}.
		
		\fbox{\em Only If.} Suppose for the sake of contradiction that $f$ is not surjective. Then, the set 
		$S := \{y \in Y: \forall x(x \in X \to f(x) \ne y)\}$ is non-empty. Clearly, $S \subseteq Y$, then by premise we must have $f(f^{-1}(S)) = S$. However, by the definition of $S$ and the definition of inverse images, we have $f^{-1}(S) = \varnothing$, and thus $f(f^{-1}(S)) = \varnothing$, a contradiction (as $S$ is non-empty).
		
		\item \fbox{\em If.} An immediate consequence of My Theorem~\ref{my.inj.surj.image.thm}.
		
		\fbox{\em Only If.} Suppose for the sake of contradiction that $f$ is not injective. Then, there exists such $x,x' \in X$ that $x \ne x'$ but $f(x) = f(x')$. Let $S := \{x\}$. Then, by the definition of forward images, $f(S) = \{f(x)\}$. However, by the definition of inverse images, $\{x,x'\} \subseteq f^{-1}(\{f(x)\})$, so it clearly is not equal to $S$, a contradiction.
	\end{enumerate}
\end{proof}

\begin{exercise}{3.4.6}
	\begin{enumerate}
		\item For all set $X$, there exists such a set $2^X$ that $Y \in 2^X$ iff $Y \subseteq X$.
		\item If we accept the above statement as an axiom, then we can prove the power set axiom, assuming the rest axioms of the set theory.
	\end{enumerate}
\end{exercise}
The proofs for the two are quite tricky. Let me divide them into parts.

\fbox{\em 1.} We use Prof.~Tao's hint.
\begin{proof}
	For any subset $Y \in X$, and any element $x \in X$, $x \in Y$ is either $\true$ or $\false$. If we regard $0 = \false$ and $1 = \true$, then for each $x$ we will assign either $0$ or $1$ to it, which looks like what a function does.

	Hence,  consider the set $\{0,1\}^X$, obtained by the axiom of power set. For all $f \in \{0,1\}^X$, we can replace $f$ with $f^{-1}(\{1\})$, because the inverse image is well defined. (That is, for each function, it has a unique inverse image on a given set.)
	
	First, be the definition of inverse images, $f^{-1}(\{1\})$ is a subset of $X$.
	
	Second, we need to show that every subset of $X$ equals $f^{-1}(\{1\})$ for some $f \in \{0,1\}^X$. Let $S \subseteq X$ be any subset of $X$. Define
	\[
	f_S: X \to \{0,1\},\ x \mapsto
	\begin{cases}
		0 & \text{if } x \notin S \\
		1 & \text{if } x \in S
	\end{cases}
	\]
	Clearly, $f_S$ is well-defined, because $x \in S$ is either true or false. By the power set axiom, $f_S \in \{0,1\}^X$. Now we only need to show that $f_S^{-1}(\{1\}) = S$.	By the definition of inverse images, $x \in f_S^{-1}(\{1\})$ iff $f(x) = 1$. Therefore, $x \in f_S^{-1}(\{1\})$ iff $x \in S$, as desired.
\end{proof}

\fbox{\em 2.} Now we assume the existence of $2^X$ for any set $X$. For any set $X$ and $Y$, we want to prove the existence of the set $Y^X$, which contains all functions from $X$ to $Y$.
\begin{proof}
	I tried many times and think this is quite hard. Let me try to break it down into smaller pieces. Then general idea is, for each element $x \in X$, it can be mapped to any element in $Y$. Then for each such mapping, another element in $X$ can be mapped to any element in $Y$, too. Repeat this until we have mapped all elements in $X$ to some element in $Y$.
	
	First, for any element $x \in X$, we would like to create a set that contains all possible mappings from it to an element in $Y$. To do this, we have to use the axiom of replacement on $Y$, replacing each element $y \in Y$ with $f_{xy}: \{x\} \to \{y\}$. This obeys the conditions of the axiom, because any function from a singleton to a singleton is unique. (Obviously the domain and codomain match; and obviously the function has only one possible value.) We denote the set created as $M_{x}$.
	
	Because $M_{x}$ solely depends on the choice of $x$, we can then replace each element $x \in X$ with $M_x$. Let's denote this big set as $\Omega_1$.
	
	This is already a breakthrough. Each $M_{x}$ has the same amount of elements as $Y$, and there are the same amount of $M_{x}$'s in $\Omega_1$ as there are elements in $X$. Prof.~Tao pointed out early that the power set $Y^X$ would have $n^m$ elements, if $Y$ has $n$ elements and $X$ has $m$. We already have exactly this number of functions in our $\Omega_1$, although
	\begin{enumerate}
		\item first we will have to extract them out from the layer of $M_{x}$'s.
		\item second they are still all singleton to singleton functions. We have to somehow manage to create ``larger'' functions from them.
	\end{enumerate}

	Anyway, let's do it step by step. Using the axiom of union, we can create
	\[
		\Omega_2 := \bigcup_{M_x \in \Omega_1} M_x
	\]
	This set contains all the singleton to singleton functions in $\Omega_1$ directly. Now we need to somehow make ``larger'' functions from them.
	
	It is where we need the assumed axiom of subsets, which (is the only axiom that) can combine the possibilities of each of these singleton to singleton functions. We then denote
	\[
		\Omega_3 := 2^{\Omega_2}
	\]
	
	Now, we select all subsets in $\Omega_3$ that can cover all $x \in X$, but only once for each, using the axiom of specification. Let $Q(S)$ be a property pertaining to all $S \in \Omega_3$ such that $Q(S)$ is true iff
	\begin{enumerate}
		\item For all $x \in X$, there exists a function $f \in S$ whose domain equals $\{x\}$.
		\item No two functions in $S$ have the same domain.
	\end{enumerate}
	, and denote the result set as
	\[
		\Omega_4 := \{S \in \Omega_3 : Q(S)\}
	\]
	
	Finally, we can turn each set in $\Omega_4$ into a corresponding function. For any $S \in \Omega_4$, define $f_S: X \to Y$ this way: For any $x \in S$, there exists, by the restriction applied with $Q(S)$, a unique $f_{xy}$ that maps $\{x\}$ to some $\{y\}$, where $y \in Y$. Define $f_S(x) := y$. Because the $f_{xy}$ is unique for any $x$, this function is well defined. We replace each $S \in \Omega_4$ with such a $f_S$, to get $\Omega_5$.
	
	Now we show that $\Omega_5$ is indeed the set that contains all functions from $X$ to $Y$.
	\begin{enumerate}
		\item First, we show that each function in $\Omega_5$ is indeed from $X$ to $Y$. By the definition of $Q(S)$, we see that every $x$ is covered. And from the construction of $f_{xy}$ earlier, we see that indeed $y \in Y$, as desired.
		
		\item Then, we show that every function from $X$ to $Y$ lies in $\Omega_5$. Let $f$ be any function in $\Omega_5$. We try to construct a $S \in \Omega_3$ that can satisfy $Q(S)$. If this is possible, then we can subsequently make $f$ again from $S$ with the above steps.
		
		With this $f$, replace each element $x$ in $X$ with the function $\{x\} \to \{f(x)\}$, to get the set $X_f$. One can easily verify that this
		\begin{enumerate}
			\item Is indeed a subset of $\Omega_2$, because each such function $\{x\} \to \{f(x)\}$ falls into $\Omega_2$, by its definition. Therefore, $X_f \in \Omega_3$.
			
			\item Satisfies the definition of $Q(S)$, because $f$ is a function from $X \to Y$ --- it covers every $x \in X$ uniquely. Therefore, $X_f \in \Omega_4$.
		\end{enumerate}
	
		Hence, we apply the replacement procedure that we used on $\Omega_4$ to get $\Omega_5$. And by the definition of the procedural, we can easily see that it preserves all the values of $f$. Hence the result must be equal to $f$.
	\end{enumerate}
	
	We have finally finished the proof. Phew.
\end{proof}

\declareexercise{3.4.7}
\begin{proof}
As stated by the previous exercise, there exists a set $\mathbb{X}$ whose elements are all subsets of 
$X$, and a set $\mathbb{Y}$ whose elements are all subsets of $Y$.

For every element $x \in \mathbb{X}$, apply the axiom of replacement to $\mathbb{Y}$, to obtain a set 
$S_x := \{y^x\}$ for every element $y \in \mathbb{Y}$. 

According to the axiom of union, using $\mathbb{X}$ as the index set, we have the set
\[
Z = \bigcup_{x \in \mathbb{X}} S_x
\]

Apply again the axiom of union to $Z$ to obtain $R$, which contains all elements of elements of $Z$. Now 
we show that $R$ is the set we want.

On one hand, let $f$ be an arbitrary function with the domain of $X' \subseteq X$, and the range of $Y' 
\subseteq Y$. We can see that $f \in {Y'}^{X'} \in S_{X'}$. ${Y'}^{X'}$ becomes an element of $Z$. And 
thus $f$ becomes an element in $R$. 

On the other hand, from the construction of $R$, we can see that $R$ contains only these elements.

\end{proof}

\declareexercise{3.4.8}
\begin{proof}
Let $A,B$ be two arbitrary sets. They are also objects as stated by Axiom 3.1. So according to 
Axiom 3.3, there exists a set $S=\{A,B\}$. By Axiom 3.11, we have a set $Z$ such that 
\[
\forall x(x \in Z \equiv \exists X(X \in S \wedge x \in X))
\]

Now we show that $Z$ is the set we want. If $x \in A \vee x \in B$, then 
$\exists X(X \in S \wedge x \in X)$ is true. So $x \in Z$.

If $x \notin A \wedge x \notin B$, then $\forall X(X \in S \Longrightarrow x \notin X)$, that is, 
$\exists X(X \in S \wedge x \in X)$ is false. So $x \notin Z$.

$Z$ is therefore the set we want. 
\end{proof}

\paragraph{Example 3.4.11}
In (3.3), why do Tao choose some element $\beta$ of $I$? This is because we need to apply the axiom of 
specification to $A_\beta$ with the restriction $x \in A_\alpha$ for all $\alpha \in I$.

\declareexercise{3.4.9}
\begin{proof}
This is quiet easy to prove. Let the left-handed side set be $S$, the RHS set be $S'$. For any 
$x \in S$, $x \in A_\alpha$ for all $\alpha \in I$. So $x \in A_{\beta'}$. And $x \in A_\alpha$ for all 
$\alpha \in I$. Therefore $x \in S'$. 

It is nearly the same the prove $x \in S' \Longrightarrow x \in S$.
\end{proof}

\paragraph{Exercise 3.4.10} \label{exercise3.4.10}
\begin{proof}
For the sake of convenience, let 
$(\bigcup_{\alpha \in I} A_{\alpha})\cup(\bigcup_{\alpha \in J}A_{\alpha})$ be $S$, \\
$\bigcup_{\alpha \in I \cup J} A_{\alpha}$ be $S'$, 
$(\bigcap_{\alpha \in I} A_{\alpha})\cap(\bigcap_{\alpha \in J}A_{\alpha})$ be $Z$,
$\bigcap_{\alpha \in I \cup J} A_{\alpha}$ be $Z'$.

(1) When $I,J \neq \varnothing$: 
On one hand, 
\[
x \in S \Longrightarrow (x \in \bigcup_{\alpha \in I} A_{\alpha} \vee 
x \in \bigcup_{\alpha \in J}A_{\alpha})
\]
If $x \in \bigcup_{\alpha \in I} A_{\alpha}$, then $x \in \bigcup_{\alpha \in I \cup J} A_{\alpha}$.
If $x \in \bigcup_{\alpha \in J} A_{\alpha}$, then $x \in \bigcup_{\alpha \in I \cup J} A_{\alpha}$.

On the other hand, if $x \in S'$, then there exists an object $a \in I \cup J$ such that $x \in A_a$.
If $a \in I$ then $x \in x \in \bigcup_{\alpha \in I} A_{\alpha} \Longrightarrow x \in S$.
If $a \in J$ then $x \in x \in \bigcup_{\alpha \in J} A_{\alpha} \Longrightarrow x \in S$.

When $I,J$ are both empty, $S,S'$ are all empty.

When there is only one of $I,J$ is empty, say it is $I$, then 
$S = \varnothing \cup \bigcup_{\alpha \in J} = \bigcup_{\alpha \in J}$. And 
$S' = \bigcup_{\alpha \in \varnothing \cup J} A_{\alpha} = \bigcup_{\alpha \in J}$.

(2)
\[
x \in Z \equiv \forall a(a \in I \Longrightarrow x \in A_a) \wedge 
\forall b(b \in J \Longrightarrow x \in A_b)
\], which is equal to $\forall a(a \in I \cup J \Longrightarrow x \in A_a) \equiv x \in Z'$.
\end{proof}

\paragraph{Exercise 3.4.11} \label{exercise3.4.11}
\begin{proof}
(1) Let the LHS be $S$, the RHS be $S'$. 
\begin{align*}
x \in S &\equiv \\
x \in X \wedge x \notin \bigcup_{\alpha \in I} A_{\alpha} &\equiv \\
x \in X \wedge \forall a(a \in I \Longrightarrow x \notin A_{a})
\end{align*}
\begin{align*}
x \in S' &\equiv \\
\forall a(a \in I \Longrightarrow x \in X \setminus A_a) &\equiv \\
x \in X \wedge \forall a(a \in I \Longrightarrow x \notin A_a)
\end{align*}.

So $S=S'$.

(2) Let the LHS be $Z$, the RHS be $Z'$. 
\begin{align*}
x \in Z &\equiv \\
x \in X \wedge x \notin \bigcap_{\alpha \in I} A_{\alpha} &\equiv \\
x \in X \wedge \neg(\forall a(a \in I \Longrightarrow x \in A_a)) &\equiv \\
x \in X \wedge \exists a(a \in I \Longrightarrow x \notin A_a)
\end{align*}
\begin{align*}
x \in Z' &\equiv \\
x \in X \wedge \bigvee_{\alpha \in I}(x \notin A_{\alpha}) &\equiv \\
x \in X \wedge \exists a(a \in I \Longrightarrow x \notin A_a)
\end{align*}

Thus, $Z = Z'$
\end{proof}

\section{Cartesian products}
\declareexercise{3.5.1}
\begin{proof}
First we show that $(x,y) = \{\{x\},\{x,y\}\}$ is a good definition. 
Let $S_1$ denote $(x_1,y_1) = \{\{x_1\},\{x_1,y_1\}\}$, $S_2$ denote 
$(x_2,y_2) = \{\{x_2\},\{x_2,y_2\}\}$.

On one hand, if $x_1=x_2\wedge y_1=y_2$, then obviously $S_1=S_2$ for they have the same elements.

On the other hand, if $S_1 = S_2$, then 
\[
\{x_1\} \in S_2 \wedge \{x_1,y_1\} \in S_2 \wedge
\{x_2\} \in S_1 \wedge \{x_2,y_2\} \in S_1 
\].
We have that 
\begin{align*}
\{x_1\} \in S_2 &\equiv \{x_1\} = \{x_2\} \vee \{x_1\} = \{x_2,y_2\} \\
&\equiv x_1=x_2 \vee x_1=x_2=y_2 \\
&\Longrightarrow x_1=x_2
\end{align*}
\begin{align*}
\{x_1,y_1\} \in S_2 &\equiv \{x_1,y_1\} = \{x_2\} \vee \{x_1,y_1\} = \{x_2,y_2\} \\
&\equiv x_1=x_2=y_1 \\
&\vee \textcolor{red}{((x_1=x_2\wedge y_1=y_2)\vee(x_1=y_2\wedge y_1=x_2))} 
\end{align*}
Similarly we have that
\begin{align*}
\{x_2,y_2\} \in S_1 
&\equiv x_2=x_1=y_2 \\
&\vee \textcolor{red}{((x_2=x_1\wedge y_2=y_1)\vee(x_2=y_1\wedge y_2=x_1))}
\end{align*}
We may notice that the red-colored text are two same statements. Thus from $\{x_1,y_1\} \in S_2$ and 
$\{x_2,y_2\} \in S_1$ we can always conclude that $y_1=y_2$. Therefore, 
$S_1 = S_2 \Longrightarrow x_1=x_2\wedge y_1=y_2$.

Then we show that if $X,Y$ are two sets, then $X \times Y$ is also a set. For each element $x \in X$, 
construct a set $S_x$, where we replace each element $y \in Y$ with $(x,y)$. Then construct the set 
$\bigcup_{x \in X}S_x$.
\end{proof}

\declareexercise{3.5.2}
\begin{proof}
Since $x,y$ are two functions, they are equal means that $\forall 1\leq i \leq n$, $x(i) = y(i)$. That 
is, $x_i = y_i, 1\leq i \leq n$.

Now we show that $\displaystyle \prod_{1\leq i\leq n}X_i$ is a set. Let set $F$ be the set that contains 
all partial functions from $N = \{i \in \mathbb{N}:1\leq i\leq n\}$ to 
$\displaystyle X = \bigcup_{1\leq i\leq n}X_i$ (Exercise 3.4.7). Use the axiom of specification, select 
such elements $f$ from $F$ that:
\begin{enumerate}
\item the element is surjective, and
\item its domain is $N$, and 
\item $f(i) \in X_i$
\end{enumerate}, 
and use all of them to construct a set $Z$, which is the set we want.
\end{proof}

\declareexercise{3.5.3}
\begin{proof}
The definition is entirely based on the equality of objects (e.g. $x = x'$). The proof is immediately 
done since this equality is reflective ($x = x$), symmetric ($x = x' \equiv x' = x$), and transitive 
($x_0 = x_1 \wedge x_1 = x_2 \Longrightarrow x_0 = x_2$).
\end{proof}

\declareexercise{3.5.4}
\begin{proof}
(1)
\begin{align*}
(x,y) \in A \times (B \cup C) &\equiv x \in A \wedge y \in (B \cup C) \\
&\equiv x \in A \wedge (y \in B \vee y \in C) \\
&\equiv (x \in A \wedge y \in B) \vee (x \in A \wedge y \in C) \\
&\equiv ((x,y) \in A \times B) \vee ((x,y) \in A \times C) \\
&\equiv (x,y) \in (A \times B) \cup (A \times C)
\end{align*}

(2)
\begin{align*}
(x,y) \in A \times (B \cap C) &\equiv x \in A \wedge y \in (B \cap C) \\
&\equiv x \in A \wedge (y \in B \wedge y \in C) \\
&\equiv (x \in A \wedge y \in B) \wedge (x \in A \wedge y \in C) \\
&\equiv ((x,y) \in A \times B) \wedge ((x,y) \in A \times C) \\
&\equiv (x,y) \in (A \times B) \cap (A \times C)
\end{align*}

(3)
\begin{align*}
(x,y) \in A \times (B \setminus C) &\equiv x \in A \wedge y \in (B \setminus C) \\
&\equiv x \in A \wedge (y \in B \wedge \neg (y \in C)) \\
&\equiv (x \in A \wedge y \in B) \wedge \neg (x \in A \wedge y \in C) \\
\tag{The statement $x \in A$ implies $\neg (x \in A \wedge y \in C) 
\Longrightarrow \neg (y \in C)$}\\
&\equiv ((x,y) \in A \times B) \wedge \neg((x,y) \in A \times C) \\
&\equiv (x,y) \in (A \times B) \setminus (A \times C)
\end{align*}
\end{proof}

\declareexercise{3.5.5}
\begin{proof}
(1)
\begin{align*}
(x,y) \in (A \times B) \cap (C \times D) 
&\equiv (x,y) \in (A \times B) \wedge (x,y) \in (C \times D) \\
&\equiv (x \in A \wedge y \in B) \wedge (x \in C \wedge y \in D) \\
&\equiv (x \in A \wedge x \in C) \wedge (y \in B \wedge y \in D) \\
&\equiv x \in A \cap C \wedge y \in B \cap D \\
&\equiv (x,y) \in (A \cap C) \times (B \cap D)
\end{align*}

(2) It is not true since 
\begin{align*}
(x,y) \in (A \times B) \cup (C \times D) 
&\equiv (x,y) \in (A \times B) \vee (x,y) \in (C \times D) \\
&\equiv (x \in A \wedge y \in B) \vee (x \in C \wedge y \in D) \\
&\nLeftrightarrow (x \in A \vee x \in C) \wedge (y \in B \vee y \in D)
\end{align*}
Generally 
\[
(x \in A \wedge y \in B) \vee (x \in C \wedge y \in D) \Longrightarrow 
(x \in A \vee x \in C) \wedge (y \in B \vee y \in D)
\], 
but
\[
(x \in A \vee x \in C) \wedge (y \in B \vee y \in D) \nRightarrow
(x \in A \wedge y \in B) \vee (x \in C \wedge y \in D)
\].

(3) It is not true since
\begin{align*}
(x,y) \in (A \times B) \setminus (C \times D) 
&\equiv (x,y) \in (A \times B) \wedge (x,y) \notin (C \times D) \\
&\equiv (x \in A \wedge y \in B) \wedge (x \notin C \vee y \notin D) \\
&\nLeftrightarrow (x \in A \wedge x \notin C) \wedge (y \in B \wedge y \notin D)
\end{align*}
\end{proof}

\declareexercise{3.5.6}
\begin{proof}
(1) On one hand, if $A \subseteq C$ and $B \subseteq D$, then 
\begin{align*}
(x,y) \in A \times B &\equiv x \in A \wedge y \in B \\
&\Longrightarrow x \in C \wedge y \in D \\
&\Longrightarrow (x,y) \in C \times D
\end{align*}, 
which means $A \times B \subseteq C \times D$.

On the other hand, if $A \times B \subseteq C \times D$, but we suppose that 
\[
\neg(A \subseteq C \wedge B \subseteq D)
\]. 
We only consider that $A \nsubseteq C$, the other situations are similar. Then 
$\exists x(x \in A \wedge x \notin C)$. Let $p = (x,y)$, where $y \in B$, then $p \in A \times B$. 
But $x \notin C$, so $p \notin C \times D$, a contradiction. Therefore, 
\[
A \times B \subseteq C \times D \Longrightarrow A \subseteq C \wedge B \subseteq D
\]

(2) On one hand, if $A = C \wedge B = D$, then
\begin{align*}
(x,y) \in A \times B &\equiv x \in A \wedge y \in B \\
&\equiv x \in C \wedge y \in D \\
&\equiv (x,y) \in C \times D
\end{align*}.

On the other hand, if $A \times B = C \times D$, but we suppose that $\neg(A = C \wedge B = D)$. 
We only consider that $A \neq C$, the other situations are similar. Then we only consider 
$\exists x(x \in A \wedge x \notin C)$, for the other situations are similar. 

(3) It is easy to prove that $X \times \varnothing = \varnothing$ and 
$\varnothing \times X = \varnothing$. Let $A = \varnothing$, we can see that even if $B \nsubseteq D$, 
$A \times B \subseteq C \times D$. 

Let $A = D = \varnothing$, then even if $A \neq C$, $A \times B = C \times D$.
\end{proof}

\declareexercise{3.5.7}
\begin{proof}
Existence: Let $h(t):=(f(t),y(t))$. It is easy to verify that $h(t) \in X \times Y$, and that given a 
$t \in Z$, $h(t)$ is unique. Therefore, $h$ is a function. And it is obvious that 
$\pi_{X\times Y \rightarrow X} \circ h = f$ and that $\pi_{X\times Y \rightarrow Y} \circ h = g$.

Uniqueness: $\pi_{X\times Y \rightarrow X} \circ h = f$ and $\pi_{X\times Y \rightarrow Y} \circ h = g$ 
imply that if there is another function $h'$ that satisfies the requirements, then $h'(t) = h(t)$. So $h$ 
is unique. 
\end{proof}

\declareexercise{3.5.8}
\begin{proof}
On one hand, if for some $i, X_i = \varnothing$, then 
\[
\forall (x_i)_{1\leq i \leq n}(\bigwedge^{n}_{i =1}x_i \in X_i \equiv (x_i)_{1\leq i \leq n} \in 
\varnothing)
\], 
which means that $\varnothing = \prod_{i=1}^{n}X_i$.

On the other hand, if $\prod_{i=1}^{n}X_i = \varnothing$ but we suppose that $X_i \neq \varnothing$. Then 
for each $i$, $\exists x_i \in X_i$. We thus have a tuple $(x_i)_{1\leq i \leq n}$, which should be an 
element of $\prod_{i=1}^{n}X_i$. Therefore we have a contradiction.
\end{proof}

\declareexercise{3.5.9}
\begin{proof}
On one hand, let $x \in (\bigcup_{\alpha \in I}A_{\alpha})\cap(\bigcup_{\beta \in J}B_{\beta})$. Then 
\[
\exists a(a \in I \wedge x \in A_a) \wedge \exists b(b \in J \wedge x \in B_b)
\]
It is obvious that $x \in A_a \cap B_b$ and that $(a,b) \in I \times J$. Therefore 
\[
x \in \bigcup_{(\alpha,\beta) \in I \times J}(A_\alpha \cap B_\beta)
\].

On the other hand, let $x \in \bigcup_{(\alpha,\beta) \in I \times J}(A_\alpha \cap B_\beta)$. Then 
\begin{align*}
\exists (a,b) \in I \times J(x \in A_a \cap B_b) 
&\Longrightarrow  x \in A_a \wedge x \in B_b \\
&\Longrightarrow x \in \bigcup_{\alpha \in I}A_{\alpha} \wedge x \in \bigcup_{\beta \in J}B_{\beta} \\
&\Longrightarrow x \in (\bigcup_{\alpha \in I}A_{\alpha})\cap(\bigcup_{\beta \in J}B_{\beta})
\end{align*}
\end{proof}

\paragraph{Exercise 3.5.10} \label{exercise3.5.10}
\begin{proof}
We denote $\overset{\sim}{f}$ as $f'$, the graph of $f$ as $G$, and the graph of $f'$ as $G'$ for the 
sake of simplification.

(1) On one hand, if $f = f'$, then for every $(x,f(x)) \in G$, we can find $(x,f'(x)) \in G'$, and 
obviously $(x,f(x)) = (x,f'(x))$, and vice versa.

On the other hand, if $G = G'$, then for each $(x,f(x)) \in G$, $(x,f(x)) \in G'$. Note that each 
element of $G'$ obeys the form $(x,f'(x))$, so $f(x) = f'(x)$ for every $x \in X$, that is, $f=f'$.

(2) Existence: Let $f(x)$ be such a value that $(x,f(x)) \in G$. Thus the value is unique, so $f$ is a 
function. According to its definition, the graph of $f$ is $G$.

Uniqueness: As proven in (1), if $f,f'$ have the same graph, then they are equal.
\end{proof}

\paragraph{Exercise 3.5.11} \label{exercise3.5.11}
I think this exercise is meaningless. Lemma 3.4.6 is proven by the fact that $X^Y$ exists, which depends 
on Axiom 3.10. Then the exercise asks us to prove Axiom 3.10 using Lemma 3.4.6. So I looked up some books 
about set theory and found out that the power set axiom is essentially Lemma 3.4.6, not Axiom 3.10.

Nevertheless, here is the proof:
\begin{proof}
Let set $Z$ contains all subsets of $X\times Y$. The specify such element in $Z$ that obey the vertical 
line test, and let them form the set $S$. According to the previous exercise, for each element in $S$, 
there exists an unique function whose graph is the element. Then we replace all elements in $S$ with 
these functions to construct the set $F$. Obviously, each element in $F$ is a function with the domain 
$X$ and the range $Y$.

Now we show that every function $f$ from $X$ to $Y$ is in $F$. Denote the graph of $f$ as $G$. We know 
that $G$ obeys the vertical line test and $G \subseteq X \times Y$, so $G \in S$. Since $G$ is the graph 
of $f$, $f \in F$.
\end{proof}

\paragraph{Exercise 3.5.12} \label{exercise3.5.12}
I am confused by this exercise. It seems that simply applying induction to $a$ can solve the 
problem, just like what we did in Proposition 2.1.16. What is wrong?

By the way, according to the \href{https://terrytao.wordpress.com/books/analysis-i/}{corrections}, 
edit the exercise as the following:
\begin{quotation}
Let $X$ be an arbitrary set containing at least an element $c$ and obeys the Peano axioms. Let $f$ be a 
function from $N \times X$ to $X$. ... 

Show that there exists an unique function $a$ from $X$ to $X$ such that 
\[
a(0) = c
\] 
and 
\[
a(n++) = f(n,a(n)), \forall n \in X
\]
...

such that $a_N(0) = c$ and $a_{N}(n++) = f(n,a_N(n))$ ... 
\end{quotation}

Note that all properties (e.g. orders, addition) in section 2 are deduced from the Peano axioms and their 
definitions. Since $X$ obeys these rules, we use such properties on elements of $X$ without proof.

The proof is now reserved for further research.
\begin{proof}
\end{proof}

\paragraph{Exercise 3.5.13} \label{exercise3.5.13}
\begin{proof}
Use induction.

Existence: We need to prove that for all $n \in \mathbb{N}$, $f(n)$ is defined. Use induction:
$f(0) = 0'$ is define. And the definition is unique for $0$ is not the successor of any natural number. 
Now suppose that $f(n) = n'$ is defined, then $f(S(n)) = S'(f(n)) = S'(n')$ is also defined. The 
definition is also unique. So we know that $f$ exists.

Injectivity: We need to prove that $f(m) = f(n) \Longrightarrow m = n$. If $f(m) = f(n)$, then $m' = n'$, 
and thus $m=n$. 

Surjectivity: Use induction: 
The basic case is, for $0' \in \mathbb{N}'$, $f(0) = 0'$. 

Now suppose that for $n' \in \mathbb{N}'$, we can find $n \in \mathbb{N}$ such that $f(n) = n'$, then 
for $S'(n')$, we have $f(S(n)) = S'(n')$. We can close the induction now.
\end{proof}

\section{Cardinality of Sets}

\declareexercise{3.6.1}
\begin{proof}
Reflexivity: Let $f(x):= x, X \rightarrow X$. $f$ is bijective since $f^{-1}(x) = x$ exists.

Symmetry: If $X,Y$ have the same cardinality, then $\exists f:X\rightarrow Y$ which is bijective. So 
$f^{-1}$ exists, and is also a bijection. Thus $Y,X$ have the same cardinality. Since then, we can say 
that two sets have the same cardinality without caring about the order.

Transitivity: If $X,Y$ have the same cardinality, and $Y,Z$ also have the same cardinality, then there 
exist two bijections: $f:X \rightarrow Y$ and $g:Y \rightarrow Z$. It is easy to verify that $g \circ f$ 
is also a bijection and is from $X$ to $Z$ (See \exerciseref{3.3.7}).
\end{proof}

\paragraph{Remark 3.6.6}
It is $f(n) := S(n)$. We are now proving something stronger
\begin{lem} \label{lem3.6.6}
For any natural number $m,n$, $\{i \in \mathbb{N}:0\leq i\leq n\}$ and 
$\{i \in \mathbb{N}:m\leq i\leq n+m\}$ have the same cardinality.
\end{lem}
\begin{proof}
Use induction on $m$. When $m=0$, the statement is obviously true. Simply give the function $f(n):=n$.

Suppose that for some $m$, we have proven the statement. Then there exists a bijection: 
\[
f:\{i \in \mathbb{N}:0\leq i\leq n\} \rightarrow \{i \in \mathbb{N}:m\leq i\leq n+m\}
\].
Let $g$ be a function from $\{i \in \mathbb{N}:0\leq i\leq n\}$ to $\mathbb{N}$ such that 
$g(x) = S(f(x))$. We prove that $g$ is a bijection from $\{i \in \mathbb{N}:0\leq i\leq n\}$ to 
$\{i \in \mathbb{N}:S(m)\leq i\leq n+S(m)\}$.

First we prove that $g(n)$ always in $\{i \in \mathbb{N}:S(m)\leq i\leq n+S(m)\}$, which is immediately 
given by the fact that addition preserves order. 

Surjectivity: For any $a \in \{i \in \mathbb{N}:S(m)\leq i\leq n+S(m)\}$, $a$ is positive. Then $a$ is 
always some number's successor, that is $a = S(b) = b+1$ for some natural number $b$. Since addition 
preserves order, $b \in \{i \in \mathbb{N}:m\leq i\leq n+m\}$. $f$ being surjective implies that there is 
some $x$ in the domain such that $f(x) = b$, and $g(x) = f(x) + 1 = a$.

Injectivity: By cancellation law, $f(x) + 1 \neq f(x') + 1 \equiv f(x) \neq f(x') \equiv x \neq x'$.

We can now close the induction.
\end{proof}

\paragraph{Lemma 3.6.9}
Empty functions are not injective when the range is not empty (See \exerciseref{3.3.3}). 

Now we show that $g$ is bijective:
\begin{proof}
Injectivity: $f$ being injective implies that 
\[
\forall x \forall x'((x \in X \wedge x' \in X) \Longrightarrow (f(x) = f(x') \Rightarrow x = x'))
\]
For $a,a' \in X - \{x\}$, they also $\in X$. If $g(a) = g(a')$, then either directly $f(a) = f(a')$ or 
$f(a) - 1 = f(a') - 1$, which gives $f(a) = f(a')$. Thus $a = a'$. (Note that subtraction is not defined 
yet, see the footnote about this in the book).

Surjectivity: The surjectivity of $f$ gives 
\[
(\forall 1 \leq i \leq n)(\exists a(a \in X \wedge f(a) = i))
\].

If $f(x) = n$, then $g(a) = f(a)$ for all meaningful $a$. Then for $1 \leq i \leq n-1$, we can find $a$ 
such that $a \in X \wedge a \neq x$, that is, $x \in X - \{x\}$. So $g(a)$ is meaningful, then $g$ is 
surjective.

If $f(x) \neq n$, then $f(x) < n$. For those $1 \leq i < f(x)$, $g$ is obviously surjective. For 
$n-1 \geq i \geq f(x)$, since $S(i) \leq n$, $\exists a(a \in X \wedge f(a) = S(i))$. And we know that 
$S(i) \neq f(x)$, then $a \in X - \{x\}$. So $g(a) = f(a) - 1 = i$.
\end{proof}

\declareexercise{3.6.2}
\begin{proof}
On one hand, if $X$ is empty, then we know that the empty function whose range is also empty is injective, 
(See \exerciseref{3.3.3}) so its cardinality is $0$. 

On the other hand, if $\# X = 0$ but $X \neq \varnothing$, then there exists an bijection 
$f:X \rightarrow \varnothing$, which is impossible.
\end{proof}

\declareexercise{3.6.3}
\begin{proof}
When $n = 0$, this is vacuously true. The base case then becomes $n=1$. We simply let $M = f(1)$.

Suppose that the statement for $n$ is true. And for $1\leq i\leq n$ we have the number $M$. Then $f(S(n))$ 
either $\geq$ or $<$ $M$. On the former case, let $f(S(n))$ be $M'$, and on the latter case, let $M' = M$. 
It is east to verify that $M'$ is the number we want.
\end{proof}

From now on we will denote $\{i\in \mathbb{N}:1\leq i \leq n\}$ as $\mathbb{N}_n$

\declareexercise{3.6.4}
\begin{proof}
(a) 
Let $n = \#X$. There is an injective  $f$ from $X$ to $\{i\in \mathbb{N}:1\leq i\leq n\}$. Let $g$ be a 
function from $X \cup \{x\}$ to $\{i\in \mathbb{N}:1\leq i\leq n+1\}$ such that $g(a) = f(a)$ if 
$a\neq x$, and $g(x) = n+1$. Now we show that $g$ is bijective.

Injectivity: We know that $\forall x \in X$, $g$ is already injective. Since that $g(x) = n+1 \neq g(a)$ 
for all $a \in X$, so $g$ is injective on $X \cup \{x\}$.

Surjectivity: We know that $\forall i \in \{i\in \mathbb{N}:1\leq i\leq n\}$, we can find 
$a \in X \cup \{x\}$ such that $g(a) = i$. And we have $g(x) = n+1$, so 
$\forall a \in \{i\in \mathbb{N}:1\leq i\leq n+1\}$, we can find $a \in X \cup \{x\}$ such that 
$g(a) = i$. 

(b)
First we prove that if $X,Y$ are disjoint, then $\#X + \#Y = \#(X\cup Y)$. Let $f$ be a bijection from $X$ 
to $\mathbb{N}_{\#X}$, and $g$ be a bijection from $Y$ to $\mathbb{N}_{\#Y}$. According to 
\hyperref[lem3.6.6]{this Lemma}, there exists a bijection $h$ from $\mathbb{N}_{\#Y}$ to 
$\{i\in \mathbb{N}:\#X+1\leq i \leq \#X+\#Y\}$. Thus $h \circ g$ is also a bijection. Let $u$ be a 
function from $X \cup Y$ to $\mathbb{N}_{\#X} \cup\{i\in \mathbb{N}:\#X+1\leq i \leq \#X+\#Y\}$. Now we 
show that $u$ is bijective.

Injectivity: For $x \neq x'$ in the domain. If $x,x'$ are both in $X$ or $Y$, then $f(x)\neq f(x')$ is 
immediately given by the injectivity of $f$ and $h \circ g$. If one of them is in $X$, and the other is 
in $Y$, then they can also never be equal because the ranges of the two functions are disjoint. 

Surjectivity: It is easy to verify that the range is equal to $\mathbb{N}_{\#X + \#Y}$. For any $y$ in the 
range, if $y \in$ the range of $f$, then $u$ is surjective since $f$ is, and if $y \in$ the range of $h 
\circ g$, $u$ is surjective for the same reason. The range consists of only this two sets, so $u$ is 
surjective on the whole range.

The proof is over. This also implies that $X \cup Y$ is finite. Now we need only to show that 
$\#(X \cup Y) < \#X + \#Y$ when $X,Y$ are not disjoint. It is easy to see that
\begin{align*}
\#A + \#B 
&= \#(A - A \cap B) + \#(A \cap B) + \#(B - A \cap B) + \#(A \cap B) \\
&= (\#(A - A \cap B) + \#(A \cap B) + \#(B - A \cap B)) + \#(A \cap B) \\
&= \#(A \cup B) + \#(A \cap B) \\
&> \#(A \cup B)
\end{align*}

(c)
If $X \subseteq Y \wedge X \neq Y$, then $\#(Y \setminus X) \neq 0$. 
\[
\#Y = \#X + \#(Y \setminus X) > \#X
\].

If $X = Y$, then $\#(Y \setminus X) = 0$, and $\#Y$ becomes $\#X$.

(d)
$f: X \rightarrow f(X)$ is always surjective. If $f$ is also injective, then $f$ is bijective. On this 
occasion, $\#f(X) = \#X$. If $f$ is not injective, we can select a set $X' \subseteq X \wedge X' \neq X$, 
on which $f$ is bijective. Then $\#X' = \#f(X') = \#f(X)$. According to (c), $\#X' < \#X$, so 
$\#f(X) < \#X$.

(e)
Suppose that $\#Y = n$. Use induction on $n$. 

When $n=0$, $Y$ is empty, then $\#(X \times Y) = 0 = \#X \times 0$. Here we additionally prove that 
when $n=1$, this is also true for further usage. When $n=1$, let $Y = \{a\}$. Then the bijection is 
$f(x):=(x,a), X \rightarrow X \times \{a\}$.

Suppose that we have proven for some $n$, $\#(X \times Y) = \#X \times \#Y$. Then when $\#Y = S(n)$, 
let $Y = Y\setminus\{x\}\cup\{x\}$, where $x \in Y$. Lemma 3.6.9 tells us that 
$\#(Y\setminus\{x\}) = S(n)-1 = n$. And \exerciseref{3.5.4} tells us that 
$X \times Y = X \times (Y\setminus\{x\}) \cup X \times \{x\}$. 
\begin{align*}
\#(X \times Y) 
&= \#(X \times (Y\setminus\{x\}) \cup X \times \{x\}) \\
&= \#(X \times (Y\setminus\{x\})) + \#(X \times \{x\}) \\
&= \#X \times n + \#X \\
&= \#X \times S(n)
\end{align*}

We can now close the induction.

(f)
We should first define $m^n$ for natural numbers $m,n$. It has not been done yet. Exponentiation is 
defined for rational numbers at Definition 4.3.9.
\begin{defn}
\begin{itemize}
\item $m^0=1$,
\item $m^{S(n)} = m^n \times m$
\end{itemize}
\end{defn}

Suppose that $\#Y = m,\#X = n$. Use induction on $n$. 

When $n=0$, $X$ is empty, then $Y^X$ has one function $f:\varnothing \rightarrow Y$.

Suppose that we have proven the statement for some $n$. Before we proceed the proof, we need some lemmas.
\begin{lem}
If $X$ is not empty, 
\[
\#Y^{X\setminus\{x'\}\cup\{x'\}} = \#Y^{X\setminus\{x'\}} \times \#Y
\], 
where $x'$ is an element of $X$.
\end{lem}
\begin{proof}
By (e) we know that 
\[
\#Y^{X\setminus\{x'\}} \times \#Y = \#(Y^{X\setminus\{x'\}} \times Y)
\].

Try to build a bijection between $Y^{X\setminus\{x'\}} \times Y$ and $Y^X$. Let $f' \in Y^X$.

Let $h$ be a function from $Y^X$ to $Y^{X\setminus\{x'\}} \times Y$ such that
\[
h(f') = (f,f'(x')), 
\]
where $f(x):=f'(x)$ when $x \neq x'$. Now we show that $h$ is bijective.

Injectivity: 
If ${f_1}' \neq {f_2}'$, then 
\[
{f_1}'(x') \neq {f_2}'(x') \vee \exists x(x \neq x' \wedge {f_1}'(x) \neq {f_2}'(x))
\]
That is, 
\[
{f_1}'(x') \neq {f_2}'(x') \vee f_1 \neq f_2,
\]
which means 
\[
(f_1,{f_1}'(x')) \neq (f_2,{f_2}'(x')).
\]

Surjectivity:
For any $(f,a) \in Y^{X\setminus\{x'\}} \times Y$, let $f'$ be $f$ if $x\neq x'$, and $f'(x') = a$. Then 
$f' \in Y^X$ and $h(f') = (f,a)$.

So, 
\[
\#Y^X = \#(Y^{X\setminus\{x'\}} \times Y)
\], which gives the lemma.
\end{proof}

Now we proceed the proof. Suppose that $\#X = n+1$, then $\#(X \setminus\{x'\}) = n$. By induction 
hypothesis, $\#(Y^{X \setminus\{x'\}}) = m^n$. 

By the lemma, 
\[
\#Y^X = \#Y^{X\setminus\{x'\}\cup\{x'\}} = \#Y^{X\setminus\{x'\}} \times \#Y,
\]
which equals to $m^n \times m$.

Now we can close the induction.

We have proven that the cardinality of power sets obeys the definition of power. This ensures the 
exercise.
\end{proof}

\declareexercise{3.6.5}
\begin{proof}
Let $f((x,y)):= (y,x), A\times B \rightarrow B \times A$. The bijectivity is obvious. 

Now we are using set theory to prove the commutativity of multiplication of natural number. For any 
natural number $m,n$, construct two sets: $M = \mathbb{N}_{m}, N = \mathbb{N}_{n}$. According to (e) in 
Proposition 3.6.14, we have that $\#(M \times N) = \#M \times \#N$. Then by what we have just proven, 
\[
\#(M \times N) = \#(N \times M) \Longrightarrow \#M \times \#N = \#N \times \#M \Longrightarrow mn = nm
\]
\end{proof}

\declareexercise{3.6.6}
\begin{proof}
Let $c \in C$, $f \in (A^B)^C$. Then $f(c)$ is a function $B\rightarrow A$. Let 
$b \in B, h \in A^{B\times C}$. Let
\[
g:A^{B\times C} \rightarrow (A^B)^{C}
\]
be such a function that for all $b,c$,
\[
g(h) = f \equiv h(b,c) = (f(c))\,(b)
\]
. Now we show that $g$ is bijective.

Injectivity: 
If $h \neq h'$, then $\exists b_0,c_0(h(b_0,c_0) \neq h'(b_0,c_0))$. Let $g(h) =f, g(h') = f'$. Then we 
know that $(f(c_0))\,(b_0) \neq (f'(c_0))\,(b_0)$, so $f(c_0) \neq f'(c_0) \Longrightarrow f \neq f'$. 
That means, $g(h) \neq g(h')$.

Surjectivity:
For any $f \in (A^B)^{C}$, let $h$ be such a function $\in A^{B\times C}$ that for all $b \in B,c \in C$, 
$h(b,c) := (f(c))\,(b)$. It is easy to see that $h$ is well-defined. So $g(h) = f$.

Note that by Proposition 3.6.14 we have $\#M^N = m^n$ and $\#(M \times N) = mn$, where $\#M =m, \#N =n$. 
Suppose that $\#A = a, \#B = b, \#C = c$, then
\[
\#(A^B)^C = (\#A^B)^{\#C} = (a^b)^c
\]
\[
\#A^{B\times C} = \#A^{\#(B \times C)} = a^{bc}
\]
So we have proven that $(a^b)^c = a^{bc}$.

Now we try to prove $a^b \times a^c = a^{b+c}$. Let $B,C$ be disjoint sets with the cardinality $b,c$ 
respectively. What we need to show is that
\[
\#(A^B \times A^C) = \#(A^{B \cup C}).
\]

Similarly, let 
\[
f: (A^{B \cup C}) \rightarrow (A^B \times A^C)
\]
be such a function that 
\[
f(g) = (u,v) \equiv \forall x(x \in B \Rightarrow g(x) = u(x) \wedge x \in C \Rightarrow g(x) = v(x)),
\]
where $g\in A^{B \cup C}, (u,v) \in A^B \times A^C$.

We can verify the bijectivity of $f$ nearly in the same way as way did previously. So I won't write it 
down here.

Then, we know $B \cap C = \varnothing \Rightarrow \#(B \cup C) = \#B + \#C$. So we can conclude that 
\[
a^b \times a^c = a^{b+c}
\]
\end{proof}

\declareexercise{3.6.7}
\begin{proof}
On one hand, if $\#A = a \leq \#B = b$, we show that $A$ has lesser or equal cardinalty to $B$. Let $f$ 
be a bijection from $A$ to $\mathbb{N}_{a}$, $g$ be a bijection from $B$ to $\mathbb{N}_b$. Let 
$\iota(x):=x, \mathbb{N}_{a} \rightarrow \mathbb{N}_b$. Then $g^{-1} \circ \iota \circ f$ is an injection 
from $A$ to $B$.

On the other hand, suppose that there is an injection $f$ from $A$ to $B$. We know that 
$f:A\rightarrow f(A)$ is bijective. So $\#A = \#f(A)$. Since $f(A) \subseteq B$, $\#f(A) \leq B$ (See (c) 
in Proposition 3.6.14). That is, $\#A \leq \#B$
\end{proof}

\declareexercise{3.6.8}
\begin{proof}
$f:A \rightarrow f(A)$ is bijective. So $f^{-1}: f(A) \rightarrow A$ is surjective. Let $g$ be defined as:
\begin{itemize}
\item $b \in f(A) \Longrightarrow g(b) = f^{-1}(b)$
\item $b \in B\setminus f(A) \Longrightarrow g(b)$ is any element of $A$.
\end{itemize}

Then $g$ is surjective.
\end{proof}

\declareexercise{3.6.9}
\begin{proof}
\begin{align*}
\#A + \#B 
&= \#(A - A \cap B) + \#(A \cap B) + \#(B - A \cap B) + \#(A \cap B) \\
&= (\#(A - A \cap B) + \#(A \cap B) + \#(B - A \cap B)) + \#(A \cap B) \\
&= \#(A \cup B) + \#(A \cap B)
\end{align*}
\end{proof}

\paragraph{Exercise 3.6.10} \label{exercise3.6.10}
\begin{proof}
Presume the negation:
\[
\forall i(i \in \{1,\dots,n\} \Longrightarrow \#(A_i) < 2)
\]

Use mathematical induction for (b) in Proposition 3.6.14, we can easily get:
\[
\#\bigcup_{i \in \{1,\dots,n\}}A_i \leq \sum_{i \in \{1,\dots,n\}} \#A_i
\]
We can also use mathematical induction to furthermore enhance what we proved while dealing with natural 
numbers to:
\[
\bigwedge_{i} a_{i} \leq b_{i} \Longrightarrow \sum_{i} a_{i} \leq \sum_{i} b_i
\].

Then because $\# A_i \leq 1$, so 
\[
\sum_{i \in \{1,\dots,n\}} A_i \leq (\sum_{i \in \{1,\dots,n\}} 1 = n)
\], 
which is impossible.
\end{proof}

\newpage
%% Copyright (C) He Guanyuming 2020
% The file is licensed under the MIT license.

\section{Integers and Rationals}
Now we are going to extend natural numbers to integers and rationals.

\subsection{The Integers}

\declareexercise{4.1.1}
\begin{proof}
It is immediately given by the fact that 
\[
a+b = a+b \equiv a -- b = a -- b
\]
\end{proof}

\paragraph{Lemma 4.1.3}
\[
(m--0)+(n--0) = (m+n)--0
\]
\[
(m--0) \times (n--0) = (mn) -- 0
\]
ensures that the definition $m--0:=m$ is consistent with addition and multiplication.

\declareexercise{4.1.2}
\begin{proof}
\[
a--b = a'--b' \equiv a=b \wedge a'=b'
\]
Then, 
\[
(b--a) = (b'--a') \equiv -(a--b) = -(a'--b')
\]
\end{proof}

\declareexercise{4.1.3}
\begin{proof}
\begin{align*}
-1 \times a 
&= (0 -- 1) \times (a -- 0) \\
&= (0\times a + 1 \times 0) -- (0 \times 0 + 1 \times a) \\
&= 0 -- a \\
&= -a
\end{align*}
\end{proof}

\declareexercise{4.1.4}
\begin{proof}
Let $x=(a--b),y=(c--d),z=(e--f)$.

(1)
\begin{align*}
(a--b) + (c--d) 
&= (a+c) -- (b+d) \\
&= (c+a) -- (d+b) \\
&= (c--d) + (a--b)
\end{align*}

(2)
\begin{align*}
((a--b) + (c--d)) + (e--f)
&= ((a+c)+e) -- ((b+d)+f) \\
&= (a+(c+e)) -- (b+(d+f)) \\
&= (a--b) + ((c--d) + (e--f))
\end{align*}

(3)
First ,
\[
(a--b) + (0--0) = (a--b)
\].

Second, by (1) we have $0+x=x+0$.

(4)
First, 
\begin{align*}
(a--b) + (b--a) 
&= (a+b) -- (a+b) \\
&= 0 -- 0 \tag{$a+b+0=a+b+0$}
\end{align*}

Second, by (1) we have $x+(-x) = (-x) + x$.

(5)
\begin{align*}
(a--b)(c--d)
&= (ac + bd) -- (ad + bc) \\
&= (ca + db) -- (cb + da) \\
&= (c--d)(a--b)
\end{align*}

(6)
The book proved this.

(7)
First,
\[
(1--0)(a--b) = (1a + 0b) -- (1b+0a) = (a--b)
\]

Second, by (5) we have $1x=x1$.

(8)
\begin{align*}
&(a--b)((c--d)+(e--f)) \\
&= (a--b)((c+e)--(d+f)) \\
&= (a(c+e) + b(d+f)) -- (a(d+f) + b(c+e)) \\
&= ((ac + bd)+(ae + bf)) -- ((ad + bc)+(af + be)) \\
&= (ac+bd)--(ad+bc) + (ae+bf)--(af+be) \\
&= (a--b)(c--d) + (a--b)(e--f)
\end{align*}

(9)
This can be easily concluded from (5) and (8).
\end{proof}

\declareexercise{4.1.5}
\begin{proof}
We need to show that 
\[
a \neq 0 \wedge b \neq 0 \Longrightarrow ab \neq 0
\]

Since $a,b$ are not 0, they can be either positive or negative. If they are both positive, the case is 
already proven.

When at least one of them is negative, we can divide the $-1$ from the negative ones. That is, if $a=-m$, 
where $m$ is positive, then we substitute $a$ with $-1 \times m$. Then we may get $ab$ in either the form 
$(-1)(-1) mn$ or $(-1) mn$, where the former is a positive number because $(-1)(-1) =1$ and the latter is 
negative.
\end{proof}

\declareexercise{4.1.6}
\begin{proof}
We check the value of $ac-bc$. We know that $ac=bc$, so $ac - bc = 0 - 0 = 0$. According to (9) in 
Proposition 4.1.6, 
\[
ac - bc = ac+(-b)c = (a+(-b))c = 0
\]

As stated by Proposition 4.1.8, since that $c \neq 0$, $a+(-b) = 0$, which means $a-b=0$. Then we have 
$a=b$.
\end{proof}

\declareexercise{4.1.7}
In the following contents, $p$ stands for a positive natural number, $n$ stands for a natural number.

\begin{proof}
(a)
\begin{align*}
a>b 
&\equiv a = b+p \\
&\equiv a+(-b) = b + (-b) + p  \tag{See the following explanation} \\
&\equiv a-b = p
\end{align*}
We now explain why $a = b+p \equiv a+(-b) = b + (-b) + p$. Using the substitution law and the 
commutativity of addition, it is clear to see that $a = b+p \Longrightarrow a+(-b) = b + (-b) + p$. We now 
show the cancellation law of addition, that is,
\begin{lem}
\[
a+c = b+c \Longrightarrow a = b
\]
\end{lem}
\begin{proof}
\begin{align*}
a+c=b+c
&\Longrightarrow a+c+(-c) = b+c+(-c) \\
&\Longrightarrow a+(c+(-c)) = b + (c+(-c)) \\
&\Longrightarrow a=b
\end{align*}
\end{proof}

So we get the inverse result: $a = b+p \Longleftarrow a+(-b) = b + (-b) + p$.

Note that by the definition of integer and what we have know now, we can conclude that 
\begin{lem}
For every integer 
$i = a - b, j = c - d$, there exists exactly one integer $k$ such that $i = j+k$.
\end{lem}

(b)
\begin{align*}
a>b
&\equiv a = b + p \\
&\Longrightarrow a+c = b+c+p \\
&\Longrightarrow a+c>b+c
\end{align*}

(c)
\begin{align*}
a>b
&\equiv a=b+p \\
&\Longrightarrow ac = (b+p)c = bc + pc \\
&\Longrightarrow ac > bc \tag{$pc > 0$ by Lemma 2.3.3}
\end{align*}

(d)
\[
a>b \equiv a = b+p
\]
Then
\[
-a = -(b+p) = (-1)(b+p) = -b - p
\]
So
\[
-a+p=-b-p+p
\]
That is,
\[
-b=-a+p \equiv -b>-a
\]

(e)
Let
\[
a = b+p_1,b=c+p_2
\]
Then $a = c+(p_1+p_2)$. Obviously $p_1+p_2$ is positive, so $a>c$.

Note that $-a,-b$ are also integers, and plus that $-(-a)=a$, so we can give a stronger conclusion:
\[
a>b \equiv -a<-b
\]

(f)
If $a,b$ are all natural numbers, the statement was proven before. 

If one of them (say $a$) is negative, 
the other ($b$) is a natural number, then $a=-n$, and we know that $b>0$ and 
$0 = a+n \Longrightarrow 0 >-a$, so by (e) we have $b>a$.

If they are both negative, then their negations satisfy the statement. Then
\[
-a<-b \equiv a>b, -a=-b \equiv a=b, -a>-b \equiv a<b
\].
\end{proof}

\declareexercise{4.1.8}
An example: $P(i): i>=0$. It is obvious that $P(0)$ and $P(i) \Longrightarrow P(i+1)$ is true. But for any  
negative integer $n$, $P(n)$ is not true.

We additionally prove one more property:
\begin{lem}
For integers $a,b$,
\[
a-b=0\Longrightarrow a=b
\]
\end{lem}
\begin{proof}
We can add $b$ to both side to obtain $a=b$.
\end{proof}

\subsection{The Rationals}
\declareexercise{4.2.1}
\begin{proof}
Reflectivity:
\[
a//b = a//b \equiv ab=ab
\] 

Being Symmetric:
\begin{align*}
a//b = c//d 
&\equiv ad = bc \\
&\equiv cb = da \\
&\equiv c//d = a//d
\end{align*}

Transitivity:
\[
a//b = c//d \equiv ad = bc
\]
\[
c//d = e//f \equiv cf = de
\]
Thus,
\[
(ad)(cf) = (bc)(de)
\]
We then have
\[
afcd = becd
\]
We can cancel $d$ since $d \neq 0$ to obtain $afc=bec$. If $a=0$, we can conclude that $c,e$ also must be 
$0$. Under this occasion, $af=be$ is also true because they all equal to $0$. 
\end{proof}

\paragraph{Definition 4.2.2}
It is useful to prove that 
\begin{lem} \label{lem4.2.3}
\[
(-a)//b = a//(-b)
\]
,
\[
a//b = (-a)//(-b)
\]
\end{lem}
\begin{proof}
The first is immediately given since $(-a)(-b) = ab$. The latter is proven as $a(-b) = b(-a) = -ab$.
\end{proof}

We may notice that subtraction is not mentioned here. This is because that we can get $a-b$ by adding $a$ 
and $-b$, where addition $+$ and negation $-$ are mentioned.

\declareexercise{4.2.2}
\begin{proof}
(1) is deduced in the book.

(2)
I don't quiet understand why Tao used this $*$ sign instead of $\times$. I know it is a new definition, 
but the $\times$ sign is undefined for rationals (except for integers, but for which we can verify that 
the two definitions are the same). We will use the $\times $ sign or just leave it off here.

\[
(a'//b')(c//d) \equiv a'd = b'c \equiv ad=bc \equiv (a//b)(c//d)
\]
Similarly we can verify this for $c'//d'$.

(3)
\[
-ab'=-a'b \equiv (-a)//b = (-a')//b'
\]
\end{proof}

For the sake of simplification, we hereby introduce some useful lemmas:
\begin{lem} \label{lem4.2.1}
\[
b=d\neq 0 \Longrightarrow (a//b = c//d \equiv a=c)
\]
\end{lem}
\begin{proof}
Assume that $b=d\neq 0$.

On one hand, if $a//b=c//d$, then $ad=bc$. Since that $b=d \neq 0$, we can cancel them to obtain $a=c$.

On the other hand, if $a=c$, then if we multiply them by the same integer (namely $b=d$), and the 
results are still equal ($ad=bc$). So $a//b=c//d$.
\end{proof}
\begin{lem} \label{lem4.2.2}
\[
c \neq 0 \Longrightarrow a//b=ac//bc
\]
\end{lem}
\begin{proof}
Assume that $c\neq 0$.

First we know that $ab=ab$. Then we can further obtain $abc=abc$, which 
means $a//b=ac//bc$.
\end{proof}

\declareexercise{4.2.3}
\begin{proof}
(1)
We have
\[
a//b + c//d = (ad+bc)//(bd)
\]
\[
c//d + a//b = (cb+da)//(db)
\]
It is easy to see that they are equal.

(2)
It is proven in the book.

(3)
We just deduce $x+0=x$ here, for we have $0+x=x+0$ according to (1).
\[
a//b + 0//1 = (a1+b0)//(b1) = a//b
\]

(4)
We only prove $x+(-x)=0$ here, for we have $x+(-x)=(-x)+x$ according to (1).
\[
a//b+(-a)//b= (ab-ab)//bb = 0//b^2=0
\]

(5)
\begin{align*}
a//b \times c//d
&= ac//bd \\
&= ca//db \\
&= c//d \times a//b
\end{align*}

(6)
\begin{align*}
&(a//b \times c//d) \times e//f \\
&= ac//bd \times e//f \\
&= ace//bdf \\
&= a//b \times ce//df \\
&= a//b \times (c//d \times e//f)
\end{align*}

(7)
We only prove $x1=x$ here, for we have $x1=1x$ according to (4).
\[
a//b \times 1//1 = a1//b1 = a//b
\]

(8)
\begin{align*}
&a//b (c//d + e//f) \\
&= a//b ((cf+ed) // (df)) \\
&= a(cf+ed)//bdf \\
&= ab(cf+ed) // b^2df \tag{See Lemma \ref{lem4.2.2}} \\
&= ((ac)(bf) + (bd)(ae)) // (bd)(bf) \\
&= ac//bd + ae//bf \\
&= (a//b \times c//d) + (a//b + e//f)
\end{align*}

(9)
This can be deduced from (5) and (8).

(10)
We merely conclude $xx^{-1} = 1$ here, since we have $xx^{-1} = x^{-1}x$ from (5).

\[
a//b \times b//a  = ab//ba = (ab)1//(ab)1 = 1//1
\]
The last step is done by Lemma \ref{lem4.2.2}.
\end{proof}

\declareexercise{4.2.4}
\begin{proof}
For any rational $r = a/b$, $a,b$ are integers.
They are either positive, $0$, or negative (except that $b$ cannot be 0). When $a,b$ are both positive, 
then $r$ is also positive. When $a$ is positive but $b$ is negative, then let $b=-p$, where $p$ is 
positive, thus $a/b = a/(-p) = (-a)/p$ is negative. When $a=0$, $r=0$. When $a$ is negative, and $b$ is 
positive, then by definition $r$ is negative. When $a,b$ are both negative, according to Lemma 
\ref{lem4.2.3}, $r$ is positive.

Therefore, we have iterated through all possible situations and verified that there is and only is one 
statement for a rational is true.
\end{proof}

\declareexercise{4.2.5}
\begin{proof}
Let $x=a/b,y=c/d,z=e/f$. Before proving the following components, we will introduce some useful 
propositions here.
\begin{lem} \label{lem4.2.4}
\begin{enumerate}
\item $x>0$ is logically equivalent to $x$ being positive.
\item $x<0$ is logically equivalent to $x$ being negative.
\end{enumerate}
\end{lem}
\begin{proof}
\[
x-0 =x
\]
is itself, so whether $x$ is positive or negative, the same is $x-0$, then we can deduce $x>0$ or $x<0$, 
and vice versa.
\end{proof}

We can now use simplified notation $x>0$ to express the same meaning: $x$ is positive. 

(a)
We check the value of 
\[
\delta = x-y = a/b +(-c)/d = (ad-bc)/bd
\]
$\delta$ is also a rational number. According to the previous exercise, it is either positive, negative, 
or $0$. So $x$ either $>y$, $<y$, or $=y$ (We haven't yet proven $x-y=0 \Longrightarrow x = y$. Let's 
prove it now. We can add $y$ to both side of $x-y=0$ to obtain the result).

(b)
According to Lemma \ref{lem4.2.4}, $x<y \Longrightarrow x-y<0$. Then we multiply $-1/1$ with $x-y$ to 
obtain (It is easy to see that for rational number $r$, $-1r = -r$ and $-(-r)=r$)
\[
-1/1 \times (x + (-y)) = -x + -(-y) = y - x
\]
Since $x-y$ is negative, and the negation of a positive number is negative, so the negation of $x-y$, 
$y-x$, is positive, which means that $y>x$.

(c)
By the hypothesis, $x-y<0 \wedge y-z<0$. We are now proving that $i,j<0 \Longrightarrow i + j <0$. We can 
write $i,j$ as $o/p,q/s$ respectively. Let $p,s>0$, then $o,q<0$. Then $o/p+q/s = (os+pq)/ps$. We know 
that $os,pq<0$ (Write a negative integer as a negation of a positive integer to see that the product of a 
positive and a negative is also negative). 

Now we show that for two positive integers, their sum is still 
positive. Integers who are positive are also natural number, and their sum remains a natural number. So 
the sum itself equals to $0$ plus itself, which means it is positive. The negation of this sum, which is 
also $-m+(-n)$, is thus negative. Since that $-m,-n$ can present any negative integer, the fact means that 
the sum of two negative integers remains a negative integer.

So $os+pq<0$. But $ps>0$, so $i+j<0$. Thus, $(x-y) + (y-z) = x-z<0$, which means $x<z$.

(d)
\begin{align*}
x+z-(y+z) 
&= x+z + (-)(y+z) \\
&= x+z + (-1)(y+z) \\
&= x+z -z - y \\
&= x-y <0
\end{align*}

(e)
It is easy to verify that the product of two positive rationals is still positive (Writing them as 
$a/b,c/d$, where $a,b,c,d>0$, then $ac/bd$ also $>0$). Then $xz-yz=z(x-y)$, which is the product of a 
positive number and a negative number, and is thus a negative number.
\end{proof}

\declareexercise{4.2.6}
\begin{proof}
According to (e) of Proposition 4.2.9, we need only to show that $x<y \Longrightarrow -x>-y$. Then we can 
multiply $xz>yz$ with $-1$ to obtain what we want.

We know that the negation operation will turn a positive into negative and vice versa. Now we have 
$x-y<0$, so the negation $-(x-y) = -x+y = -x - (-y)>0$, which means that $-x>-y$.
\end{proof}

There are still many properties about rationals that we use for granted (e.g. $x^-1$ has the same sign as 
$x$; the two definitions of order are the same, that is, $x-y>0 \equiv x=y+p \equiv x>y$, where $p>0$). 
Although they need to be proven prior to being used, we can not cover all of them here. We will 
prove some of them in the future only if they are used. Also, most of them are not hard to prove. We need 
not to worry.

We will prove some important ones here:
\begin{prop} \label{prop.different.def.of.order}
For rational numbers $x,y,p>0$
\[
x-y>0 \equiv x=y+p \equiv x>y
\]
\end{prop}
\begin{proof}
$x-y>0 \equiv x>y$ is the definition of order. We merely need to prove $x-y>0 \equiv x=y+p$ here.

On one hand, if $x-y>0$, then $p$ is not others, but the very number $x-y$.

On the other hand, if there exists a $p>0$ such that $x=y+p$. Then add $-y$ to the both side of the 
equation, and we can get $x-y = p$, which means $x-y$ is positive. So $x-y>0$.
\end{proof}

\begin{prop} \label{prop.4.2.add.ineq}
We can add two inequalities together. That is,
\[
a<b \wedge c<d \Longrightarrow a+c<b+d
\]
\end{prop}
\begin{proof}
We know that
\[
a<b \Longrightarrow a+c<b+c 
\]
, and
\[
c<d \Longrightarrow b+c<d+b 
\]
According to the transitivity of order, we can derive that
\[
a+c<b+c<b+d
\]
\end{proof}

\begin{prop} \label{prop.4.2.multiply.ineq}
We can multiply two inequalities of positives or negatives together. That is,
\[
a,b,c,d >0 \wedge a<b \wedge c<d \Longrightarrow ac<bd
\]
, and
\[
a,b,c,d <0 \wedge a<b \wedge c<d \Longrightarrow ac>bd
\]
\end{prop}
\begin{proof}
When they are all positive,
we know that
\[
a<b \Longrightarrow ac<bc
\]
, and
\[
c<d \Longrightarrow bc<bd
\]
According to the transitivity of order, we can derive that
\[
ac<bc<bd
\]

When they are all negative,
we know that
\[
a<b \Longrightarrow ac>bc
\]
, and
\[
c<d \Longrightarrow bc>bd
\]
According to the transitivity of order, we can derive that
\[
ac>bc>bd
\]
\end{proof}

Note that we can already add or multiply equations because of the axiom of substitution, so we can change 
the $<$ in the inequalities to $\leq$ in the previous two propositions whenever needed.

\subsection{Absolute Value and Exponentiation}
\declareexercise{4.3.1}
\begin{proof}
(a)
$x>0 \Rightarrow |x| >0$, $x=0\Rightarrow |x|=0$, $x<0 \Rightarrow |x| >0$. So $|x| \geq 0$.

And we can see that only when $x=0$ can $|x| = 0$.

(b)
This one is very tedious to prove. Let's enumerate all conditions:
\begin{enumerate}
\item $x,y>0$. On this occasion, 
\[
|x+y| = x+y = |x|+|y|
\]

\item At least one of them is $0$. On this occasion, let's just let $x$ be 0, the other 
situations are similar.
\[
|x+y| = |0+y| = |y| = |0| + |y| = |x| + |y|
\]

\item $x=y>0$. On this occasion, $|x+y| = |2x| = 2x = |x|+|x|$.

\item $x=y<0$. On this occasion, $|x+y| = |2x| = -2x = |x|+|x|$.

\item $x,y<0$. On this occasion, $|x+y| = -(x+y) = -x -y = |x| + |y|$.

\item One of them is positive, the other is negative. We specify $x>0,y<0$ here. But the other conditions 
are similar. Under this condition, we further divide the situation into three occasions:
\begin{itemize}
\item $x+y>0$ On this occasion, $|x+y| = x+y$, $|x| + |y| = x-y$. Note that $x-y = x+y +2(-y)$, where 
$2(-y) >0$, so $|x+y| < |x|+|y|$ (See Proposition \ref{prop.different.def.of.order}).
\item $x+y<0$ On this occasion, $|x+y| = -x-y$, $|x|+|y|=x-y$. Note that $-x-y - (x-y) = 2(-x) < 0$, so 
$|x+y| < |x| + |y|$.
\item $x+y=0$ On this occasion, $|x+y| = 0 \leq |x| + |y|$ (Recall that the sum of two positive rationals 
remains positive).
\end{itemize}
\end{enumerate}

We have iterated through all conditions.

(c)
We shall prove that 
\[
-|x| \leq x \leq |x|
\]
first.

\begin{enumerate}
\item If $x>0$, then $x=|x|>0$. And $0>-|x|$, so by the transitivity of order, $-|x| < x$.
\item If $x=0$, then $|x|=-|x| = x=0$.
\item If $x<0$, then $x=-|x|<0$ And $0<|x|$, so $x<|x|$.
\end{enumerate}
This also means that $x$ either equals to $|x|$ or $-|x|$.

Then we prove that $-y \leq x \leq y \equiv y \geq |x|$.

On one hand, if $-y \leq x \leq y$, then when $x=|x|$, we have $|x| \leq y$; when $x=-|x|$, we have 
$-y \leq -|x| \equiv y \geq |x|$. As stated previously, we know that at least one of the two conditions 
are satisfied.

On the other hand, if $y \geq |x|$, then $-y \leq -|x|$. But since that $-|x| \leq x \leq |x|$, we can 
obtain what we want by the transitivity of order.

(d)
\begin{enumerate}
\item If $x=y=0$, then $|xy| = 0 = |x||y|$.
\item If $x,y>0$, then $|xy| = xy = |x||y|$.
\item If $x,y<0$, then $xy>0$, $|xy| = xy = (-x)(-y) = |x||y|$.
\item If one of them is positive, and the other is negative, (say $x>0,y<0$), then 
$|xy|=-xy=x(-y)=|x||y|$. The other conditions are similar.
\end{enumerate}

Thus, $|-x| = |-1||x| = 1|x| = |x|$.

(e)
This can be easily conclude from (a).

(f)
Since that $|-x|=|x|$, we have $|x-y| = |-(x-y)| = |y-x|$.

(g)
Note that $x-z = (x-y) = (y-z)$. Then from (b) we can deduce that $|x-z| \leq |x-y| + |y-z|$, which is 
$d(x,z) \leq d(x,y) + d(y,z)$.
\end{proof}

\declareexercise{4.3.2}
\begin{proof}
(a)
If $x=y$, then $|x-y| = 0$. And any positive rational $\varepsilon >0$, so $|x-y|\leq\varepsilon$.

The other statement is much better easier to prove after we have know the denseness of rationals. We 
essentially repeat some of the proof work that are done afterwards here.
On the other hand, suppose the negation, that is, $(\forall \varepsilon>0)(|x-y| \leq \varepsilon)$, but 
$x\neq y$. Then $x-y \neq 0$. Let $\delta = |x-y| \neq 0$. We know that $2^{-1} = 1/2 > 0$, so 
$\delta / 2 > 0$. Also we have $\delta /2 + \delta /2 = \delta \Longrightarrow \delta /2 < \delta$. Then let 
$\varepsilon = \delta /2$. So we have both $|x-y| < \delta/2$ and $|x-y| > \delta /2$, which is impossible. 

(b)
It is immediately derived from $|x-y| = |y-x|$.

(c)
\[
|x-z| \leq |x-y| + |y-z| \leq \varepsilon + \delta
\]

(d)
\[
|x+z - (y+w)| = |x-y + z-w| \leq |x-y| + |z-w| \leq \varepsilon + \delta
\]
\[
|x-z - (y-w)| = |x-y + w-z)| \leq |x-y| + |w-z| \leq \varepsilon + \delta
\]

(e)
\[
|x-y| \leq \varepsilon < \varepsilon'
\]

(f)
From (c) of Proposition 4.3.3, we can derive that
\[
|x-z| \leq \varepsilon \equiv -\varepsilon \leq x-z \leq \varepsilon 
\equiv z-\varepsilon \leq x \leq z+ \varepsilon
\]
, and that
\[
|x-y| \leq \varepsilon \equiv y-\varepsilon \leq x \leq y + \varepsilon
\]
Thus we have
\[
y-\varepsilon \leq x \leq z + \varepsilon
\]

We will only prove the statement when $z\leq w \leq y$, another one is similar. On this occasion, 
$-y \leq -w \leq -z$. Add this inequality to $y-\varepsilon \leq x \leq z + \varepsilon$ to obtain that 
\[
-\varepsilon \leq x-w \leq \varepsilon
\]

(g)
\[
|xz-yz| = |x-y||z| \leq \varepsilon|z|
\]

(h)
We will explain why $|a| \leq \varepsilon \wedge |b| \leq \delta$ implies 
$|a||z| + |b||x| + |a||b| \leq \varepsilon|z| + \delta|x| + \varepsilon\delta$.

First, multiply $|a| \leq \varepsilon$ with $|z|$ to obtain $|a||z| \leq \varepsilon |z|$. Then add both sides 
of the inequality with $|b||x| + |a||b|$ to gain 
\[
|a||z| + |b||x| + |a||b| \leq \varepsilon |z| + |b||x| + |a||b| \tag{1}
\]
Similarly,
\[
|a||z| + |b||x| + |a||b| \leq |a||z| + \delta |x| + |a||b| \tag{2}
\]
Finally, we can multiply $|a| \leq \varepsilon$ with $|b| \leq \delta$ as stated by Proposition 
\ref{prop.4.2.multiply.ineq} to derive $|a||b| \leq \varepsilon\delta$. So after some addition we have
\[
|a||z| + |b||x| + |a||b| \leq |a||z| + |b||x| + \varepsilon\delta \tag{3}
\]

Using Proposition \ref{prop.4.2.add.ineq}, we add (1),(2) and (3) together:
\[
3(|a||z| + |b||x| + |a||b|) \leq (\varepsilon|z| + \delta|x| + \varepsilon\delta) + 2(|a||z| + |b||x| + |a||b|)
\],
which can be simplified to
\[
|a||z| + |b||x| + |a||b| \leq \varepsilon|z| + \delta|x| + \varepsilon\delta
\]

Also note that if we use $x-y$ as $a$, $z-w$ as $b$, and derive $|xz-yw|$ from 
\[
xz=(y+a)(w+b),
\]
then what we will get is that $xz,yw$ are ($\delta|y| + \varepsilon|w| + \delta\varepsilon$) close.

This consequence may seem obvious, but in fact it isn't. And should we change some variables of them, the 
result may vary. This example tells us that we should be very cautious when dealing with inequalities. What we 
should do is to carefully derive conclusions from what we have proven instead of taking intuitive things for 
granted.
\end{proof}

\declareexercise{4.3.3}
\begin{proof}
(a)
\begin{enumerate}
\item Use induction. We induct on $m$. First, $x^nx^0 = x^n1 = x^{n+0}$.

Suppose that for $m$, the statement is already true. Then 
\begin{align*}
x^nx^{m+1} 
&= x^n(x^m\times x)\\
&= x^nx^m \times x \\
&= x^{n+m} \times x \tag{The induction hypothesis} \\
&= x^{n+m+1}
\end{align*}

\item Use induction. We induct on $m$. First, $(x^n)^0 = 1 = x^{n\times 0}$.

Suppose that for $m$, the statement is already true. Then 
\begin{align*}
(x^n)^{m+1} 
&= (x^n)^m \times x^n \\
&= x^{mn} \times x^n \tag{The induction hypothesis} \\
&= x^{mn + n} \tag{By the previous statement} \\
&= x^{n(m+1)}
\end{align*}

\item Use induction. We induct on $n$. First, $(xy)^0 = 1 = x^0y^0$.

Suppose that for $m$, the statement is already true. Then 
\begin{align*}
(xy)^{m+1} 
&= (xy)^m \times xy \\
&= x^my^m \times xy \tag{The induction hypothesis} \\
&= x^m \times x \times y^m \times y \\
&= x^{m+1} y^{m+1}
\end{align*}
\end{enumerate}

(b)
On one hand, if $x=0$, then for $n>0$, $0^n=0$.

On the other hand, if for $n>0$, $x^n=0$, we need to prove $x=0$. We try to show that 
$x \neq 0 \Longrightarrow x^n \neq 0$. Use induction. Since that $n\neq 0$, we start from $n=1$. 
$x^1 = x^0 \times x = x$. 

Suppose that for $n$, the statement is already true. Then 
\[
x^{n+1} = x^n \times x,
\]
which is the product of two positive rationals, and which is thus positive. 

(c)
(1)
Use induction: $x^0 = y^0 =1>0$.

Suppose that for $n$, the statement is already true. Then we have two inequalities here:
\[
x^n \geq y^n \geq 0
\]
and
\[
x \geq y \geq 0
\]
We can multiply the two because of Proposition \ref{prop.4.2.multiply.ineq}. Then we have 
\[
x^{n+1} \geq y^{n+1} \geq 0
\]

If $n>0$, then we induct from $1$. The process resembles to what we have just done, so I don't write it here.

(d)
Use induction: $|x^0| = |1| =1 =|1|^0$. 

Suppose that for $n$, the statement is already true. Then 
\[
|x^{n+1}| = |x^n \times x| = |x^n| |x| = |x|^n |x| = |x|^{n+1}
\]
\end{proof}

\paragraph{Definition 4.3.11}
We can see that there are now two versions of $x^{-1}$. Now we try to show that they express the same thing. 
Write $x$ as $a/b$. The first version is $x^{-1} = b/a$. 

The second version is $x^{-1} = 1/x = 1 \times x^{-1}\text{(version 1)} = x^{-1} = b/a$.

Note that only after we have known this can we say that for the second version of $x^{-1}$, 
$(x^{-1})^{-1} = x$.

Now we can also say that $x^{-n} = (x^n)^{-1}$

\declareexercise{4.3.4}
\begin{proof}
Except for (3),
we have already proven these properties when $m,n \in \mathbb{N}$. Then we will just write $m,n$ as $-m,-n$.

(a)
(1)
\[
x^{-m}x^{-n} = \frac{1}{x^m}\frac{1}{x^n} = \frac{1}{x^mx^n}=1/x^{m+n} = x^{-m-n}
\]

(2)
Before doing this, we must derive that for integers $a,b$ and natural number $n$, $(a/b)^n = a^n/b^n$. Use 
induction: $(a/b)^0 = 1 = 1/1 = a^0/b^0$.

Suppose that for $n$, the statement is already true. Then 
\[
(a/b)^{n+1} = (a^n/b^n)(a/b) = (a^{n+1}/b^{n+1})
\]

We can now close the induction.

Thus, 
\begin{align*}
&(x^{-n})^{-m} \\
&= ((1/(x^n))^m)^{-1} \tag{$x^{-n} = (x^n)^{-1}$} \\
&= (1/(x^n)^m)^{-1} \\
&= (x^n)^m \\
&= x^{mn}
\end{align*}

(3)
\begin{align*}
(xy)^{-n} 
&= 1/(xy)^n \\
&= 1/(x^ny^n) \\
&= (1/x^n) (1/y^n) \\
&= x^{-n}y^{-n}
\end{align*}

(b)
\begin{lem}
For $x = (a/b) >  y = (c/d) > 0$,
\[
0 < x^{-1} < y{-1}
\]
\end{lem}
\begin{proof}
Let $a,b,c,d>0$. We know that
\[
a \times b^{-1} > c \times d^{-1}
\]
Multiply it with $bd>0$, we have
\[
ad>bc
\].
Multiply it with $a^{-1}c^{-1} >0$, we have
\[
d/c>b/a
\]
That is,
\[
y^{-1} > x^{-1}
\]
And they are obviously bigger than $0$.
\end{proof}

According to the lemma,
\[
x^{n} \geq y^{n} \Longrightarrow ((x^{n})^{-1} \leq (y^{n})^{-1} \equiv x^{-n} \leq y^{-n})
\]

(c)
We first show that for positive integer $n$, the statement is true. We assume that $x>y$. Another situation is 
similar.
Use induction, we try to prove that $x\neq y \Longrightarrow x^n \neq y^n$. We need to start from $n=1$. First, 
$x^1=x > y^1=y$. 

Suppose that for $n$, the statement is already true. Then
Multiply $x>y$ with $x^n > y^n$, 
we have $x^{n+1} > y^{n+1}$.

We can now close the induction.

Then for negative ones, we know that $1/x = 1/y$ iff $x=y$, so $x^n \neq y^n \equiv x^{-n} \neq y^{-n}$.

(d)
It is immediately derived since $1/x = 1/y \equiv x=y$.
\end{proof}

\declareexercise{4.3.5}
\begin{proof}
Use induction from $1$.

$2^1 = 2 >1$.

Suppose that for $N$, the statement is already true. Then
\[
2^{N+1} = 2\times 2^N > 2N \geq N+1
\]
(Note that $N\geq 1 \Longrightarrow 2N \geq N+1$)
\end{proof}

\subsection{Gaps In The Rational Numbers}
\declareexercise{4.4.1}
\begin{proof}
Existence: We show that when $x \geq 0$, $(\exists n \in \mathbb{Z})(n \leq x < n+1)$. Write $x$ as $a/b$, 
where $a \geq 0,b>0$. If $a= 0$, then $x=0$, $n=0$. If $a\neq 0$, then according to Proposition 2.3.9, 
\[
\exists m \exists r(a = mb+r),
\]
where $m,r \in \mathbb{N}$ and $r < b$. Because of this, $mb+b >a$, so $(mb+b)/b>a/b=x$, which means $m+1>x$. 
On the other hand, $mb \leq a$, so $mb/b \leq a/b =x$, which means $m \leq x$.

Then when $x<0$, then $-x>0$, and $(\exists n \in \mathbb{Z})(n \leq -x < n+1)$, so 
\[
-n \geq x > -n -1
\]
$\geq$ means $>$ or $=$ (exclusive). When $-n > x >-n-1$, let $m=-n-1$, then $m \leq x < m+1$ is true. When 
$-n =x > -n-1$, let $m=-n$, then $m \leq x < m+1$ is also true. So $m$ is the integer we want if $x<0$.

Uniqueness:
For $n \leq x < n+1$, and $m \neq n$, we try to prove that $m \leq x<m+1$ is not possible. Before doing this, 
we need some lemmas:
\begin{lem}
(1) For integers $i,j$, $i<j \equiv i+1\leq j$.
(2) For integer $i$, there is no integer $j$ such that $i<j<i+1$.
\end{lem}
\begin{proof}
(1) It has already been proven for natural numbers. If $i,j<0$, then $-i,-j>0$.
\[
i<j \equiv -i>-j \equiv -i \geq -j+1 \equiv i \leq j-1 \equiv i+1 \leq j
\]

(2)
Suppose the negation, that there exists a integer $j$ such that $i<j<i+1$. We know that 
$i<j \equiv i+1 \leq j \equiv j \geq i+1$. But we also have $j<i+1$, which is impossible.
\end{proof}

$m$ either $<$ or $>n$. On the former case, $m+1\leq n\leq x$, so $m+1>x$ is not possible. One the latter case, 
$x<n+1\leq m$, so $m \leq x$ is impossible.
\end{proof}

\declareexercise{4.4.2}
\begin{proof}
(a) We will use a different approach from the hint the book provided here. After assuming the negation, we 
try to prove that 
$a_n \leq a_0 -n$. Note that subtraction may results in a overflow (that is, natural numbers flows to negative 
integers). So we will first define $a_n$ as integers. And we try to show that no such infinite descent 
sequences can only lie in  $\mathbb{N}$.

Use induction: $a_0 \leq a_0-0$.

Suppose the statement for $n$ is already true, then 
$a_{n+1} < a_n \equiv a_{n+1} +1 \leq a_n \equiv a_{n+1} \leq a_n - 1$. Then we have 
$a_{n+1} \leq (a_0 -n -1 = a_n -(n+1))$. We can now close the induction.

However, let $n = a_0$, then $a_{n+1} \leq n-n-1 = -1$, which means that $a_{n+1}$ does not lie in 
$\mathbb{N}$.

(b)
(1) Yes. For example, $a_n:=-n$ satisfies our restrictions.

(2) Yes. Because it is always possible to find a rational between $0$ and $a_0$.
\end{proof}

\declareexercise{4.4.3}
\begin{proof}
(1) Suppose the negation, that natural number $n=2k=2k'+1$, where $k,k'$ are also natural. Then $2k=2k'+1$
Then we have $2k > 2k' \Longrightarrow k>k'$. But $2k+1 = 2(k'+1)$, so $2(k'+1) > 2k \Longrightarrow k'+1 >k$. 
But we know that between $k',k'+1$ there exists no natural numbers. so it is impossible. (Note that we don't 
have a proposition saying $ac>bc \Longrightarrow a>b$ for natural numbers, but we can first deal with them with 
the range of rationals. Multiply them with $c^{-1}$, and we will see that the result of the two sides are also 
natural numbers, so for natural numbers this is true.)

(2)
\[
p^2 = (2k+1)^2 = 4k^2+4k + 1 = 2(2k^2+2k) +1
\]

(3)
Treat $p,q$ as rationals.
$p^2/2=q^2 \Longrightarrow q^2<p^2$. We show that $q \geq p$ can not be true. It is obvious that $q \neq p$. 
And when $q>p$, multiply it with itself, $q^2>p^2$, which is impossible.
\end{proof}

\newpage
%\section{The Real Numbers}

\subsection{Cauchy Sequences}
\declareexercise{5.1.1}
\begin{proof}
Since $(a_n)^\infty_{n=1}$ is a Cauchy sequence, it is 1-steady for some $N$. That implies,
\[
\forall n \geq N(|a_n-a_N| \leq 1)
\]
And we know that $a_1,a_2,\dots,a_N$ is bounded by some number $M$, which means $|a_N| \leq M$. Expand $|a_n-a_N| \leq 1$ to obtain 
\[
a_N-1 \leq a_n \leq a_N+1
\]
If $a_N \geq 0$, then $a_n \geq a_N-1 \geq -a_N-1$, then 
\[
-(a_N+1) \leq a_n \leq a_N+1 \equiv |a_n| \leq |a_N+1|
\]
So 
\[
|a_n| \leq |a_N+1| \leq |a_N| + |1| \leq M+1
\]
If $a_N < 0$, then $a_n \leq a_N +1 < -a_N+1$, then
\[
a_N - 1 \leq a_n \leq -(a_N-1) \equiv |a_n| \leq |a_N-1|
\]
We can also get 
\[
|a_n| \leq |a_N-1| \leq |a_N| + |1| \leq M+1
\]
Therefore, we know that $(a_n)^\infty_1$ is bounded by $M+1$.
\end{proof}

\subsection{Equivalent Cauchy Sequences}
\declareexercise{5.2.1}
\begin{proof}
Although we need to prove that $a_n$ being a Cauchy sequence is logically equivalent to $b_n$ being a Cauchy sequence, showing that one 
implies another is enough due to the structure of these statements.

Now we show that $a_n$ being a Cauchy sequence implies that $b_n$ being a Cauchy sequence. We need to show that for any $\varepsilon >0$, 
there exists a $N$ such that for all $m,n\geq N$, $|a_m-a_n| \leq \varepsilon$. 

First we know that for any $\varepsilon >0$, there exists a $N$ such that $\forall m,j\geq N(|a_m-b_j|\leq \varepsilon)$. By substituting $m$ 
with $n$ we have both $|a_m-b_j|\leq \varepsilon$ and $|a_n-b_j|\leq \varepsilon$.
Thus, 
\[
|a_m-a_n| = |(a_m-b_j) - (a_n-b_j)| \leq |a_m-b_j| + |a_n-b_j| \leq 2\varepsilon
\]

Then we find the $N'$ such that $\forall m,n\geq N(|a_m-b_n| \leq \varepsilon/2)$, and then for $i,j \geq N$, $|a_i-a_j| \leq \varepsilon$.
\end{proof}

\declareexercise{5.2.2}
\begin{proof}
We need only to show that $a_n$ being bounded implies that $b_n$ being so because of the structure of these statements. 

Consider \exerciseref{5.1.1}. Since that $(a_n)^\infty_{n=1},(b_n)^\infty_{n=1}$ are eventually $\varepsilon$-close, the sequence 
$(a_n-b_n)^\infty_{n=1}$ is eventually $\varepsilon$-steady. Thus, according to the exercise mentioned, it is bounded by some number $M$. And 
we say that $a_n$ is bounded by some number $N$.

So $|a_n-b_n| \leq M$. Again, similar to the proof of $|a_n| \leq |a_N|+1$ in \exerciseref{5.1.1}, we can obtain that 
$|b_n| \leq |a_n| + M \leq N+M$. Therefore, $b_n$ is also bounded.
\end{proof}

\subsection{The Construction of the Real Numbers}
\declareexercise{5.3.1}
\begin{proof}
Reflectivity: It is immediately derived from $|a_n-a_n| = 0$.

Symmetry: It is immediately derived from $|a_n-b_n| = |b_n-a_n|$.

Transitivity: For any $\varepsilon >0$, we can find $M,N$ such that $\forall n\geq M(|a_n-b_n| \leq \varepsilon)$ and 
$\forall n\geq N(|b_n-c_n| \leq \varepsilon)$. Let $B = \max{(M,N)}$. Then for $n\geq B$,
\[
|a_n-c_n| \leq |a_n-b_n| + |b_n-c_n| \leq 2\varepsilon
\]
This can also be deduced by (c) in Proposition 4.3.7
So $a_n$ and $c_n$ are also equal.
\end{proof}

\declareexercise{5.3.2}
\begin{proof}
(1)
We need to show that $(a_nb_n)^\infty_{n=1}$ is a Cauchy sequence. For any $\varepsilon>0$, we can find $M,N$ such that 
$\forall i,j\geq M(|a_i-a_j| \leq \varepsilon)$ and $\forall i,j\geq N(|b_i-b_j| \leq \varepsilon)$. Let $B = \max{(M,N)}$. Then for 
$i,j \geq B$, (See (h) in Proposition 4.3.7)
\[
|a_ib_i - a_jb_j| \leq \varepsilon(|a_i| + |a_j|) + \varepsilon^2
\]
Note that $(a_n)^\infty_{n=1}$ is a Cauchy sequence, so it is bounded by some number $M$. Thus, 
$|a_i| + |a_j| \leq 2M$. 

For any $\varepsilon' >0$, we need to find a $\varepsilon >0$ such that 
$\varepsilon(|a_i| + |a_j|) + \varepsilon^2 \leq \varepsilon'$. First, if $\varepsilon' \geq 1$, then by 
setting $\varepsilon < 1$ we can obtain $\varepsilon^2 < 1 \leq \varepsilon'$; if $\varepsilon' < 1$, then we let $\varepsilon < \varepsilon'$, and multiply it with $\varepsilon < 1$, we then have 
$\varepsilon^2 < \varepsilon'$. After these steps, we can ensure that $\varepsilon' - \varepsilon^2>0$. 

Consider the number $t = \dfrac{\varepsilon' - \varepsilon^2}{2M}>0$. If $\varepsilon$ already satisfies 
$\varepsilon <t$, then it is the number we want. If it doesn't, then we can shrink it. That is, let
$\varepsilon'' < t \leq \varepsilon$. $\varepsilon'' < \varepsilon$ gives 
$(\varepsilon'')^2 < (\varepsilon)^2$, then $-(\varepsilon'')^2 > -(\varepsilon)^2$, and finally 
\[
t'' = \frac{\varepsilon' - (\varepsilon'')^2}{2M}>\frac{\varepsilon' - \varepsilon^2}{2M}
\]
So $\varepsilon'' < t < t''$. We can set $\varepsilon$ to this $\varepsilon''$. Then 
\begin{align*}
\varepsilon < \frac{\varepsilon' - \varepsilon^2}{2M} &\Longrightarrow \\
\varepsilon \times 2M < \varepsilon' - \varepsilon^2 &\Longrightarrow \\
\varepsilon \times 2M + \varepsilon^2 < \varepsilon'
\end{align*}

So no matter what $\varepsilon'>0$ is, we can always find $\varepsilon >0$ such that 
\[
\varepsilon(|a_i| + |a_j|) + \varepsilon^2 \leq \varepsilon'
\].
And for this $\varepsilon'$, there exists $N\geq 1$ such that 
\[
\forall i,j\geq N(|a_ib_i - a_jb_j| \leq \varepsilon(|a_i| + |a_j|) + \varepsilon^2 \leq \varepsilon')
\]
Then, $(a_nb_n)^\infty_{n=1}$ is a Cauchy sequence. So is $xy$ a real number.

(2)
For any $\varepsilon >0$, we can find $N$ such that $\forall n\geq N(|a_n-a'_n|\leq \varepsilon)$. Thus, for 
such $n$, 
\[
|a_nb_n - a'_nb_n| = |b_n||a_n - a'_n| \leq \varepsilon|b_n|
\]
Note that $(b_n)^\infty_{n=1}$ is bounded by some number $M$. So $|a_nb_n - a'_nb_n| \leq \varepsilon M$. 
Therefore, we find the $N'$ such that $\forall n\geq N'(|a_n-a'_n|\leq \varepsilon/M)$. Then for such $n$, 
$|a_nb_n - a'_nb_n| \leq \varepsilon$. Thus $\taoseq{a_nb_n}{1} = \taoseq{a'_nb_n}{1}$.
\end{proof}

\declareexercise{5.3.3}
\begin{proof}
On one hand, if $a=b$, then obviously $a,a,\cdots = b,b,\cdots$. 

On the other hand, if $a,a,\cdots \neq b,b,\cdots$, we try to show that $a=b$. Presume the negation, that is, 
$a \neq b$. Then, $|a_n-b_n| = |a-b| \geq |a-b|$. For any $0<\varepsilon< |a-b|$, the two Cauchy sequences cannot be 
$\varepsilon-$close, which is impossible.
\end{proof}

\paragraph{Lemma 5.3.14}
Here it is asked that why can we deduce $|b_n| \geq \varepsilon/2$ from $|b_{n0} - b_n| \leq \varepsilon/2$ and 
$|b_{n0} > \varepsilon$. The book says that the triangle inequality can be used. In fact, we use the fact 
\[
||b_{n0}| - |b_n|| \leq |b_{n0} - b_n|
\]
instead of $|b_{n0} - b_n| \leq |b_{n0}| + |b_n|$. Since that $|b_{n0}| - |b_n| \leq ||b_{n0}| - |b_n||$, 
we have 
\[
|b_{n0}| - |b_n| \leq \varepsilon/2 \equiv |b_{n0}| \leq \varepsilon/2 + |b_n|
\]
But $|b_{n0}| > \varepsilon$, so $\varepsilon/2 + |b_n| > \varepsilon \equiv |b_n| > \varepsilon/2$.

It is not mentioned in (b) of Proposition 4.3.3, but it can be easily proven if we divide conditions, though 
the process is indeed very tedious.

\declareexercise{5.3.4}
\begin{proof}
Since that the two Cauchy sequences are equivalent, they are eventually $\varepsilon-$steadiness for any $\varepsilon>0$. 
Then, according to \exerciseref{5.2.2}, $(a_n)^\infty_{n=1}$ being bounded implies that $(b_n)^\infty_{n=1}$ being so.
\end{proof}

\declareexercise{5.3.5}
\begin{proof}
We show that $(\frac{1}{n})^\infty_{n=1} = (0)^\infty_{n=1}$.

For each $\varepsilon>0$, we want to find $N \in \mathbb{N}$ such that 
$n\geq N \longrightarrow |\frac{1}{n}-0|\leq \varepsilon$. Note that 
\[
|\frac{1}{n}-0|\leq \varepsilon \equiv \frac{1}{n} \leq \varepsilon \equiv \frac{1}{\varepsilon} \leq n
\], 
which means that we need to find $N \geq \frac{1}{\varepsilon}$. This is always possible as stated by Proposition 4.4.1.

Then the two sequences are equivalent, which proves our proposition.
\end{proof}

\subsection{Ordering the Reals}
\declareexercise{5.4.1}
\begin{proof}
We try to show that if a real number $a$ is non-zero, then it must be either positive or negative (not both). We already know from 
Lemma 5.3.14 that $a$ can equal to $\LIM{a_n}$, where $\taoseq{a_n}{1}$ is bounded away from zero. Now we show that every Cauchy 
sequence that is bounded away from zero can always equal to either a positively bounded away from zero or a negatively bounded away 
from zero Cauchy sequence.

Let $\taoseq{a_n}{1}$ be a Cauchy sequence that is bounded away from zero. Then for every $n$, $|a_n| \geq c$. Choose $\varepsilon$ 
so that $0<\varepsilon<c$. We can find $N$ such that $m,n \geq N \longrightarrow |a_n-a_m| \leq \varepsilon$. We know that 
$|a_n| \geq c \longrightarrow a_n\leq -c \vee a_n \geq c$. We will only show that $\taoseq{a_n}{1}$ equals to some sequences 
positively bounded away from zero on the latter condition. It is easy to derive that $\taoseq{a_n}{1}$ equals to some sequences 
negatively bounded away from zero on the former condition.

We have
\[
|a_n-a_m| \leq \varepsilon \longrightarrow a_n -\varepsilon <a_m
\]
and hence $a_m >c-\varepsilon>0$ since $\varepsilon<c \wedge a_n \geq c$.

Let $\taoseq{b_n}{1}$ be such a sequence that
\begin{itemize}
\item $m \geq N \longrightarrow b_n = a_n$,
\item $0<n<N \longrightarrow b_n$ be any value bigger than $c-\varepsilon$.
\end{itemize}

Thus, $\taoseq{b_n}{1}$ is positively bounded away from zero, and is also equivalent to $\taoseq{a_n}{1}$ since that 
$m \geq N \longrightarrow b_n = a_n$.

Now we show that it cannot equal to both. Denote it by $a_n$. Presume the negation, that is, it is equal to both a sequence 
$x_n \geq c$ and $y_n \leq -c$ for $c>0$. Choose $\varepsilon$ so that $0<\varepsilon<c$. It equals to $x_n$ implies that we can always 
find $N_1$ such that 
\[
n \geq N \longrightarrow |a_n-x_n| \leq \varepsilon \longrightarrow a_n > x_n -\varepsilon >c-\varepsilon >0
\]
Similarly, we can find $N_2$ such that
\[
a_n < -(c-\varepsilon)<0
\]

This is impossible, as $a_n$ cannot be eventually $2(c-\varepsilon)$-close.

From what we have shown, we can easily derive that a real number is either positive, negative, or zero.

Now we show that $x$ is positive iff $-x$ is negative. We know $x = \LIM{a_n}$, where $a_n > c>0$. Then 
$-x = \LIM{-a_n}$. Since that $-a_n < -c <0$, $\taoseq{-a_n}{1}$ is negatively bounded away from zero. So $-x$ is 
negative, as desired.

Finally we show that if $x=\LIM{a_n},y=\LIM{b_n}$ are both positive, then $x+y=\LIM{a_n+b_n},xy = \LIM{a_nb_n}$ is also 
positive. It is immediately leaded to since 
\[
a_n >c>0 \wedge b_n >d>0 \Longrightarrow a_nb_n > cd>0 \wedge a_n+b_n > c+d>0
\]
\end{proof}

\declareexercise{5.4.2}
\begin{proof}
(a) It is immediately derived since $x-y$ satisfies Proposition 5.4.4.

(b)
Denote $x,y$ as $\LIM{a_n},\LIM{b_n}$, respectively. Note that by definition 
$y-x = \LIM{b_n-a_n} = \LIM{-(a_n-b_n)} = -(x-y)$. So from Proposition 5.4.4 we can see that
\[
x>y \equiv x-y>0 \equiv y-x<0 \equiv y<x
\]
One might notice that we use $x-y>0$ to represent $x-y$ being positive here. It is quite easy to prove. Just notice that 
$x$ being positive is logically equivalent to $x-0$ being so, and thus is to $x>0$. We can further prove that $x<0$ is 
equivalent to $x$ being negative.

(c)
We know by Proposition 5.4.4 that 
\[
x-y >0 \wedge y-z>0 \longrightarrow (x-y)+(y-z) = x-z >0
\]
So $x>y \wedge y>z \longrightarrow x>z$.

(d)
It is immediately deduced as $(x+z)-(y+z) = x-y$.

(e)
$x>y \equiv x-y>0$. As stated by Proposition 5.4.4, $z(x-y) >0$. So $xz-yz>0 \longrightarrow xz>yz$.

\end{proof}

\begin{prop} \label{prop.5.4.basicproperties}
Since that reals numbers possess the same basic algebraic properties as the properties possessed by rational numbers, we 
can ascertain that the corollaries of them are also right. For example,
\[
a<b\wedge c<d \longrightarrow a+c<b+d
\]
\[
a,b,c,d>0\wedge a<b\wedge c<d \longrightarrow ac<bd
\]
\end{prop}

\declareexercise{5.4.3}
\begin{proof}
Existence:
For a $\varepsilon>0$, there exists a $N$ such that $n\geq N \longrightarrow |a_n-a_N| \leq \varepsilon$. Choose an 
arbitrary $c>0$, and let a rational number $y$ be $\LIM{a_N-\varepsilon-c}$. On this occasion, the real number 
$y-x<0$, which means that $y<x$. 

Since $y$ is a rational number, there exists a natural number $M$ such that $M \leq y$. So $M<x$ the number $M+1$ may be 
bigger than $x$, and if it is, $M$ is the number we want.

If $M+1 \leq x$, then we check if $(M+1)+1$ is bigger than $x$, and we repeat the step until we find the first natural 
number $M'$ such that $M'+1>x$. Hence $M'$ is the number we want.

Uniqueness:
We have already shown that $\exists M(M\leq x<M+1)$.
Suppose that there exists another natural number $K$ such that $K\leq x<K+1$. We show that $K=M$. Suppose the negation. 
Then $K$ either $<M$ or $>M$. Under the former condition, $K+1\leq M\leq x$, which is impossible. Under the latter 
condition, $x<M+1\leq K$, which is also impossible.
\end{proof}

\declareexercise{5.4.4}
\begin{proof}
\[
x>0\rightarrow \frac{1}{x}>0\rightarrow \exists N(N>\frac{1}{x}>0)
\]
So, according to Proposition 5.4.8,
\[
0<\frac{1}{N}<x
\]
, as desired.
\end{proof}

\declareexercise{5.4.5}
\begin{proof}
Let $x=\LIM{x_n},y=\LIM{y_n}$. Since $x>y$, the sequence $\taoseq{x_n-y_n}{1}$ equals to a sequence that is positively 
bounded 
away from zero. We may just let $\taoseq{x_n-y_n}{1}$ be such a sequence. (This is always possible. Given a sequence 
$\taoseq{z_n}{1}$, which is positively bounded away from zero, and satisfies $\LIM{z_n}=x-y$, we can define $x_n$ as 
$y_n+z_n$) This way, for some $c>0$, $x_n-y_n>c\equiv x_n>c+y_n$. 

Moreover, we can find $N$ such that for $0<\varepsilon<\frac{c}{2}$ and $n\geq N$, 
$|x_n-x_N|\leq \varepsilon\wedge|y_n-y_N|\leq \varepsilon$, which means
\[
x_n\geq x_N -\varepsilon\wedge y_n \leq y_N+\varepsilon
\]
Adding $c>2\varepsilon$ to the inequality $x_N \geq y_N+c$ gives
\[
x_N>y_N+2\varepsilon \equiv x_N-\varepsilon>y_N+\varepsilon
\]

This simplifies our work because both $x_N-\varepsilon$ and $y_N+\varepsilon$ are rational numbers, and we know that 
there exists a rational number $q$ such that $x_N-\varepsilon<q<y_N+\varepsilon$. We now know that for $n\geq N$,
\[
x_n \geq x_N -\varepsilon \longrightarrow x_n-q \geq x_N-\varepsilon-q>0
\]
and
\[
y_n \leq y_N +\varepsilon \longrightarrow q-y_n \geq q-(y_N+\varepsilon)>0
\]
Define a new sequence as the following: $x_n'=x_n$ if $n\geq N$, and $x_n'$ be any rational number such that $
|x_n'-x_N|\leq \varepsilon$, and define $y_n'$ in the same way. Obviously $x=\LIM{x_n'},y=\LIM{y_n'}$. And the sequences
$\taoseq{x_n'-q}{1},\taoseq{q-y_n'}{1}$ are both positively bounded away from zero. Hence $x<q<y$, as desired.
\end{proof}

\declareexercise{5.4.6}
\begin{proof}
On one hand, suppose that $|x-y|<\varepsilon$. If $x=y$, then $y-\varepsilon<x<y+\varepsilon$ is obvious. If $x>y$, then 
$|x-y|=x-y<\varepsilon \rightarrow x<y+\varepsilon$, and $x>y \rightarrow x>y-\varepsilon$. If $x<y$, then 
$x<y+\varepsilon$ and $|x-y|=y-x<\varepsilon\rightarrow x>y-\varepsilon$.

On the other hand, conversely, suppose that 
\[
y-\varepsilon<x<y+\varepsilon \rightarrow x-y<\varepsilon \wedge y-x<\varepsilon
\]. If $x>y$, then $|x-y|=x-y<\varepsilon$. If $x=y$, then $|x-y|=0<\varepsilon$. If $x<y$, then 
$|x-y|=y-x<\varepsilon$.
\end{proof}

\declareexercise{5.4.7}
\begin{proof}
(a)
On one hand, if $x\geq y$, then add $0<\varepsilon$ to it (See Proposition \ref{prop.5.4.basicproperties}), we have 
$x \geq y+\varepsilon$.

On the other hand, conversely, if for all real number $\varepsilon>0$, $x \leq y+\varepsilon$, we show that $x\leq y$. 
Presume the negation, that is, $x>y$, then as stated by Proposition 5.4.14, we can find $q$ such that $x>q>y$, so 
$x>y+(q-y)$, a contradiction.

(b)
On one hand, suppose that $|x-y|\leq \varepsilon$ for all $\varepsilon>0$. If $x\geq y$, then 
$|x-y|=x-y \leq \varepsilon$, so we have $x\leq y$ by (a). If $x\leq y$, we can conclude that $x\geq y$ as well. So 
either way we will have $x=y$.

On the other hand, if $x=y$, then $|x-y|=0<\varepsilon$.
\end{proof}

\declareexercise{5.4.8}
\begin{proof}
We shall just prove the first one here.

Presume the negation, that is, $\LIM{a_n}>x$. Then we can find a rational $q$ such that $x<q<\LIM{a_n}$
(Proposition 5.4.14). Thus
$a_n\leq x<q$,
and then $\LIM{a_n}\leq\LIM{q}=q$ (Corollary 5.4.10). But we have $\LIM{a_n}>q$, a contradiction.
\end{proof}

\subsection{The Least Upper Bound Property}

\paragraph{Example 5.5.3}
This set has no upper bound because
\begin{itemize}
\item If $x$ is greater than an element in the set, then $x$ must also be an element in the set;
\item $\forall x \in \mathbb{R}^+(\exists y(y \in \mathbb{R}^+ \wedge y>x))$.
\end{itemize}
\begin{proof}
(1)
This is immediately derived from the fact that order is transitive (Proposition 5.4.7).

(2)
Consider the number $x+1$, which, according to (1), is also in the set, and which, is greater than $x$ as $y-x = 1>0$.
\end{proof}

\declareexercise{5.5.1}
\begin{proof}
First we show that every upper bound of $E$, $N$, is a lower bound for $-E$, and vice versa. This can be concluded from
\[
x \leq N \equiv -x \geq -N
\]
Then we show that $-M$ is the greatest lower bound. In fact,
\[
M \leq N \to -M \geq -N
\]
\end{proof}

\declareexercise{5.5.2}
\begin{proof}
We show that for all $L<m\leq K$, there at least exists one $m_0$ such that we can not have both $\frac{m_0}{n}$ and $\frac{m_0-1}{n}$ being upper bounds and they not being upper bounds. That is, for this $m_0$, either 
\[
\frac{m_0}{n} \text{ is an upper bound} \wedge \frac{m_0-1}{n} \text{ is not}
\]
, which is impossible as $\frac{m_0}{n}>\frac{m_0-1}{n}$, or
\[
\frac{m_0-1}{n} \text{ is an upper bound} \wedge \frac{m_0}{n} \text{ is not}.
\]

Presume the negation, that is, (Note that this statement includes the situation when $\frac{m_0}{n}$ and $\frac{m_0-1}{n}$ are both not upper bounds)
\[
(\forall L<m\leq K)(\frac{m_0}{n} \text{ is an upper bound} \equiv \frac{m_0-1}{n} \text{ is an upper bound})
\]
We know that when $m=K$, $m/n$ is an upper bound. Then $(m-1)/n$ also is. Repeat the step until $m=L+1$, then we can 
conclude that $\frac{m-1=L}{n}$ is an upper bound, a contradiction.
\end{proof}

\declareexercise{5.5.3}
\begin{proof}
\emph{This is a different approach.}

On one hand, for all $x<m_n \to x \leq m_n-1$, $x/n\leq (m_n-1)/n$, which means such $x/n$ cannot be upper bounds. On the other hand, 
for all $x>m_n \to x-1\geq m_n$, $(x-1)/n\geq m_n/n$.
\end{proof}

\declareexercise{5.5.4}
\begin{proof}
(1) Simply take $M \geq \varepsilon$, then $|m_n/n-m_{n'}/n'|\leq 1/M\leq \varepsilon$.

(2)
Since we want to use \exerciseref{5.4.8}, we need to prove that for some $N$, $(\forall n \geq N, q_n < q_M) \vee (\forall n \geq N, q_n > q_M)$. This thought introduces the following lemma:
\begin{lem}
For any Cauchy sequence $a_n$, if $a_M \neq \LIM{a_n}$, then there exists a $N$ such that for all $n$ greater than or equal to it, $a_n$ are either all lesser than $a_M$ or all greater than $a_M$.
\end{lem}
\begin{proof}
Since $q_M \neq \LIM{a_n}$,
\[
\neg (\forall \varepsilon>0(\exists N(n\geq N \to |a_n-a_M| \leq \varepsilon) ))
\]
, which is
\[
\exists \varepsilon >0 (\forall N(\exists n\geq N(|a_n-a_M| > \varepsilon)))
\]
For this $\varepsilon$, $\exists N_1(n,n'\geq N_1 \to |a_n-a_{n'}| \leq \varepsilon)$. But we know $\exists N_2 \geq N_1(|a_{N_2}-a_M| > \varepsilon)$. So for all $n \geq N_2$, we have
\[
|a_n-a_{N_2}| \leq \varepsilon \wedge |a_{N_2} - a_M| > \varepsilon
\]
This means either $a_n < q_{M}$ (when $a_M > a_{N_2}$), or $a_n > a_{M}$ (when $a_M < a_{N_2}$).
\end{proof}

If $q_M = S$, then $|q_M-S| = 0 \leq \frac{1}{M}$.

If $q_M \neq S$, then according to the previous lemma, for some $N$ and all $n\geq N$, $q_n > q_M$ or $q_n<q_M$. We suppose that $q_n > q_M$ here. We also know that for all $n\geq M$, $|q_M - q_n| \leq \frac{1}{M}$. Then for $n \geq \max(N,M)$, we can say that $\frac{1}{M} \geq q_n - q_M >0$. The limit of the sequence $(q_n-q_M)_1^n$ is $S-q_M$ (To really make it a sequence, we define its value to be any number between $\frac{1}{M}$ and 0 when $n < \max(N,M)$). We can now apply \exerciseref{5.4.8} to it to obtain $\frac{1}{M}\geq S-q_M >0$ (Note that $S \neq q_M$ is our hypothesis.) A similar process produces $\frac{1}{M}\geq q_M-S >0$ when $q_n<q_M$. 

We can now finish the proof.
\end{proof}

\declareexercise{5.5.5}
\begin{proof}
We will prove that there exists an irrational number $0<i<y-x$ and then $x+i$ automatically becomes the number we want.

Then we only need to prove that irrational numbers can be arbitrarily small. So we think of $\frac{\sqrt{2}}{N}$, which becomes smaller than any given $\varepsilon>0$ as $N$ approaches to infinite.

But first we need to prove that it is an irrational number. Presume the negation, that is, it is rational. We know that rationals are closed under multiplication, which means $\frac{\sqrt{2}}{N} \cdot N = \sqrt{2}$ is rational, a contradiction.
\end{proof}

\subsection{Real Exponentiation, Part I}
\paragraph{Lemma 5.6.5}
The set contains 0 because $0 \geq 0$ and for $n \geq 1$, $0^n = 0$.

If $y>1$, then according to Proposition 5.4.7 and Proposition 4.2.9, we know that we can multiply $y>1\wedge y>1$ to get $y^2>1^2=1$, 
and so on. In fact, $y>1 \to y^2 > y$ and $y>1\to y \times 1 > 1 \times 1 \equiv y >1$, so $y^2>y>1$. We can use induction to prove 
$y^n>1$.

Since that we can multiply such inequalities where both sides are positive, we can conclude from $y>x$ and $y>1$ that $y^n > x \times 1^{n-1} \equiv y^n > x$.

\declareexercise{5.6.1}
\begin{proof}
Let's denote the set $\{y \in \mathbb{R}:y \geq 0 \wedge y^n \leq x\}$ as $S$. Furthermore, we assert the following lemma without proof (the proof can be given by induction):
\begin{lem}
\[
(x+\varepsilon)^n = \sum_{i=0}^n{\binom{n}{i}n^i\varepsilon^{n-i}}
\]
\end{lem}


(a) If $y^n < x$, then there exists a real number $c$ such that $y^n < c < x$. By definition, $c \in$ the set that determines $x^{1/n}$. But $y$

(c) By definition, for any $y \in S$, $y \geq 0$.
\end{proof}


\newpage
%\section{Limits of Sequences}
\subsection{Convergence and limit laws}
\declareexercise{6.1.1}
Let $m=n+q$, then we may induct on $q$.

\declareexercise{6.1.2}
\noindent given any $\varepsilon>0 \equiv \forall \varepsilon>0$\\
eventually $\varepsilon$-close $\equiv \exists N\geq m(n\geq N \to |a_n - L| <\varepsilon)$

\declareexercise{6.1.3}
Base on \exerciseref{6.1.2}, if $\{a_n\}^\infty_{n=m}$ is convergent, then for any $\varepsilon>0$, we can find $N$ such that $\forall n\geq N( |a_n-L|<\varepsilon)$. Thus, $\forall n\geq \max(N,m') (|a_n-L|< \varepsilon)$, which means that $\{a_n\}^\infty_{n=m'}$ is convergent.

It is easy to prove the converse.

\declareexercise{6.1.4}
$\{a_{n+k}\}^\infty_{n=m}$ is essentially $\{a_{n}\}^\infty_{n=m+k}$. So it is immediately derived from the previous exercise.

\declareexercise{6.1.5}
\[
|a_p-a_q| \leq |a_p-L| + |L-a_q| \leq 2\varepsilon
\]

\declareexercise{6.1.6}
$\forall m \geq 1, a_m = \LIM a_m$. $a_m - L = \LIM{a_m} - \LIM{a_n} = \LIM{(a_m-a_n)}$. Suppose that $a_m$ is not eventually $\varepsilon$-close to $L$ for all $\varepsilon>0$, then $\exists \epsilon >0(\forall N \geq 1(\exists M \geq N(|\LIM{(a_m-a_n)}|>\epsilon)))$. Since that $\{a_n\}$ is a Cauchy sequence, $\exists P(\forall m,n\geq P(|a_m-a_n|\leq \epsilon))$. Given these $m,n$, we have
\[
|\LIM{(a_m-a_n)}| = \LIM{|a_m-a_n|} \leq \LIM \epsilon = \epsilon
\]
, a contradiction.

\subsection{The extended real number system}
\declareexercise{6.2.1}
These statements are already proven for real numbers. Now we only consider the statements when at least some of $x,y,z$ are $+,-\infty$.

(a) Obviously true for $+,-\infty$.

(b) If $y \ne x = - \infty$  then $x < y$; if $y \ne x = + \infty$, then $x > y$. If $x$ is real, then $x < y$ if $y = \infty$, $x > y$ if $y = -\infty$. If $x = y$, then $x = y$.

(c) If $x = y = z$ or at least two are equal, then the transitivity is obvious. Now let $x \ne y \ne z$, and we have $x,y \ne + \infty, y,z \ne - \infty$ (so $y$ is real now). If $x = - \infty$  then by definition we have $x < z$. If $x \ne - \infty$, then $z = + \infty$ (remember that we ignore when $x,y,z$ are all real), and we still have $x < z$.

(d) It is trivial when $x = y$. If $x = - \infty$, then $-x = +\infty$, and by definition we know $-y< -x$ whatever $-y$ is. If $x \ne -\infty$, then $y = +\infty$, and $-y = - \infty$, thus we have $-y < -x$ whatever $-x$ is.

\declareexercise{6.2.2}
These statements are already proven for real numbers. Now we only consider the statements when at least some of $E$ are not real. Also in the below three arguments for (a), (b), (c), respectively, we consider $E$ as non-empty.

(a) If $+ \infty \in E$, then $\sup E = + \infty$, and $\forall x \in E, x \le \infty$. For $\inf E = - \sup (-E)$, since that $- \infty \in -E$, $\inf E = - \sup (-E - \{-\infty\}) = \inf (E - \{+ \infty\})$. If $E - \{+ \infty\}$ consists of only reals, then we are done. Otherwise, $\inf E = - \infty$, and we are also done. 

If $- \infty \in E$, we have already shown the $\inf$ part, and we know that for all $x \in E - \{-\infty\}$ (a set doesn't contain $- \infty$), $x \le \sup (E - \{-\infty\})$. But by definition, $\sup E = \sup (E - \{-\infty\})$, so we are done.

(b) If $+ \infty \in E$, then it is the only upper bound of $E$, and it is $\sup E$. If $- \infty \in E$, then this, however, does not change $\sup E$. We can also show that it also does not change any upper bound of $E$, since $- \infty$ is less or equal to any extended real number. Therefore, we are done.

(c) The argument for (b) applies similarly here, if we exchange $-\infty$ with $\infty$.

If otherwise $E$ is empty, then (a) instead becomes vacuously true, and (b) and (c) are obvious.

\subsection{Suprema and Infima of sequences}
\declareexercise{6.3.1}
Since that 1 is an element of this set, then 1 must be smaller than or equal to all the upper bounds. In addition, 1 is the biggest number in the set, so for all elements in the set, it is less than or equal to 1. The two facts together make 1 the least upper bound of the set.

The infimum of the sequence is the supremum of the minus sequence. 0 is obviously an upper bound of the minus sequence. On the other hand, for any number < 0, there are elements of the minus sequence that can exceed beyond it. So no number smaller than 0 can be an supremum for the minus sequence, and thus 0 is the infimum of the sequence.

\declareexercise{6.3.2}



\newpage
%\section{Series}
\subsection{Finite Series}
\declareexercise{7.1.1}
\begin{proof}
(a)
Induct on $p$. When $p=m+1$, $n=m$, and 
\[
\sum_{i=m}^n{a_i} + \sum_{i=n+1}^p{a_i} = a_m + a_{m+1} = \sum_{i=m}^p{a_i}
\]

We suppose inductively that for some $p$ the property still holds, then for $p+1$,
\begin{align*}
&\sum_{i=m}^n{a_i} + \sum_{i=n+1}^{p+1}{a_i}\\
&= \sum_{i=m}^n{a_i} +\sum_{i=n+1}^{p}{a_i} + a_{p+1} \tag{By def.}\\
&= \sum_{i=m}^{p}{a_i} + a_{p+1} \tag{Induction Hypothesis} \\
&= \sum_{i=m}^{p+1}{a_i} \tag{By def.} 
\end{align*}

(b)
Induct on $n$. The inductive step is 
\[
\sum_{i=m}^{n+1}a_i = \sum_{i=m}^{n}a_i + a_{n+1} = \sum_{i=m+k}^{n+k}a_i + a_{n+k+1-k} = \sum_{i=m+k}^{n+k}a_i
\]

(c)
The inductive step is
\begin{align*}
\sum_{i=1}^{n+1}{(a_i+b_i)} 
&= \sum_{i=1}^{n}{(a_i+b_i)} + (a_{n+1}+b_{n+1}) \\
&= \sum_{i=1}^{n}{a_i}+\sum_{i=1}^{n}{b_i}+ (a_{n+1}+b_{n+1}) \\
&= \sum_{i=1}^{n+1}{a_i}+\sum_{i=1}^{n+1}{b_i}
\end{align*}

(d)
The inductive step is
\begin{align*}
\sum_{i=1}^{n+1}{ca_i} 
&= \sum_{i=1}^{n}{ca_i} + ca_{n+1} \\
&= c\sum_{i=1}^{n}{a_i}+ c(a_{n+1}) \\
&= c\sum_{i=1}^{n+1}{a_i}
\end{align*}

(e) 
The inductive step is 
\begin{align*}
\left|\sum_{i=1}^{n+1}{a_i}\right|
&= \left|\sum_{i=1}^{n}{a_i} + a_{n+1}\right| \\
&\leq \left|\sum_{i=1}^{n}{a_i}\right| + \left|a_{n+1}\right| \\
&\leq \sum_{i=1}^{n}{|a_i|} + |a_{n+1}| \\
&= \sum_{i=1}^{n+1}{|a_i|}
\end{align*}

(f)
The inductive step is
\begin{align*}
\sum_{i=1}^{n+1}{a_i}
&= \sum_{i=1}^{n}{a_i} + a_{n+1} \\
&\leq \sum_{i=1}^{n}{b_i} + b_{n+1} \\
&= \sum_{i=1}^{n+1}{b_i}
\end{align*}
\end{proof}

\declareexercise{7.1.2}
\begin{proof}
(a) 
Any function $g$ from the empty set ($\{i:1\leq i \leq 0\}$) to the empty set is a bijection. So $\sum_{x \in \varnothing}{f(x)} = \sum_{i=1}^0{f(g(i))} = 0$.

(b)
This time the bijection $g$ would be $\{1\} \to \{x_0\}$. And we have $\sum_{x \in \{x_0\}}{f(x)} = \sum_{i=1}^1{f(g(i))} = f(g(1)) = f(x_0)$.

(c)

\end{proof}

\subsection{Infinite Series}
\declareexercise{7.2.1}
It is divergent.
\begin{proof}
It is immediately derived from $((-1)^n)^\infty_n$ being divergent.
\end{proof}


\declareexercise{7.2.2}
\begin{proof}
According to Theorem 6.4.18, $(S_n)_n^\infty$ is convergent iff it is Cauchy. That is, iff 
\[
\forall \varepsilon>0(\exists N(\forall p,q\geq N(|S_p-S_q|\leq \varepsilon)))
\]
If $p\geq q$, then according to Lemma 7.1.4, (a), $|S_p-S_q| = |\sum_{i=q+1}^p{a_i}|$. If $p \leq q$, then 
\[
|S_p-S_q| = \left|\sum_{i=1}^p{a_i}-(\sum_{i=1}^p{a_i}+\sum_{i=q+1}^p{a_i})\right| = \left|-\sum_{i=q+1}^p{a_i}\right| = \left|\sum_{i=q+1}^p{a_i}\right|
\]
We nearly finished the proof except that here $i$ starts at $q+1$, not $q$. But this is an unimportant matter. In fact, on one hand, $\forall p,q \geq N(|\sum_{i=q}^p{a_i}|) \rightarrow \forall p,q \geq N(|\sum_{i=q+1}^p{a_i}|)$; on the other hand, $\forall p,q \geq N(|\sum_{i=q+1}^p{a_i}|) \rightarrow \forall p,q \geq N+1(|\sum_{i=q}^p{a_i}|)$. We only requires the existence of $N$ for any arbitrary $\varepsilon >0$, so the two statements are equivalent.
\end{proof}

\declareexercise{7.2.3}
\begin{proof}
Simply let $p=q$ in Proposition 7.2.5 to obtain $\forall \varepsilon >0 (\exists N(\forall p\geq N(|a_p|\leq \varepsilon)))$ and we are finished.
\end{proof}

\declareexercise{7.2.4}
\begin{proof}
According to Lemma 7.1.4, (e), we have $|\sum_{i=q}^p{a_i}| \leq \sum_{i=q}^p{|a_i|} = |\sum_{i=q}^p{|a_i|}|$ as $\sum_{i=q}^p{|a_i|}\geq 0$. So if $|\sum_{i=q}^p{|a_i|}| \leq \varepsilon$, then $|\sum_{i=q}^p{a_i}|$ must also satisfy it.
\end{proof}

\declareexercise{7.2.5}
\begin{proof}
(a)
According to Lemma 7.1.4, (c), the partial sum of $\sum_{n=m}^\infty{a_n+b_n}$ is $\sum_{n=m}^Na_n + \sum_{n=m}^Nb_n$. The limit of this sequence, according to Proposition 6.1.19, is the sum of the two limits, $\sum_{n=m}^\infty a_n+ \sum_{n=m}^\infty b_n$.

(b)
The partial sum can be seen as $c \sum_{n=m}^Na_n$.

(c)
This statement immediately follows from taking limits on both sides of the following equation derived from Lemma 7.1.4 (for $N \geq m+k$)
\[
\sum_{n=m}^Na_n = \sum_{n=m}^{m+k-1}a_n + \sum_{n=m+k}^Na_n
\]

(d)
This statement immediately follows from 
\[
\sum_{n=m}^Na_n = \sum_{n=m+k}^{N+k}a_{n-k}
\]
\end{proof}

\declareexercise{7.2.6}
\begin{proof}
By induction we can easily show that $\sum_{i=0}^N{a_n-a_{n+1}} = a_0-a_{N+1}$. Taking limits on both sides of this equation immediately gives 
$\sum_{n=0}^\infty = a_0-\lim_{n \to \infty}a_n$.
\end{proof}

\subsection{Sums of non-negative numbers}
\declareexercise{7.3.1}
\begin{proof}
By Lemma 7.1.4, we know that $\sum_{n=m}^N|a_n| \leq \sum_{n=m}^Nb_n$. Since that $\sum_{n=m}^\infty b_n$ converges, there is a $M$ such that 
$\sum_{n=m}^N|a_n| \leq \sum_{n=m}^Nb_n \leq M$. This fact plus that $|a_n| \geq 0$ immediately leads to the convergence of $\sum_{n=m}^\infty|a_n$. We can draw a similar conclusion for $|\sum_{n=m}^Na_{n}|$.

To show that the limits of them still follows the order, however, it is not so obvious. We must prove that limit preserves order.
\end{proof}

\declareexercise{7.3.2}
\begin{proof}
If $|x| \geq 1$, then $x^n \to$ either $\infty$ or $-\infty$. Thus the series is divergent.

If $|x| < 1$, then the partial sum of $|x|^n$ equals $\frac{1-|x|^{N+1}}{1-|x|}$. Since $|x|^{N+1} \to 0$, taking limit gives $\sum_{n=0}^\infty {|x|^n} = \frac{1}{1-|x|}$. This immediately implies that the original series is conditionally convergent. So taking limit on the not absolute partial sum gives what we want.
\end{proof}

\declareexercise{7.3.3}
\begin{proof}
Assume that negation, that is, at least for some $n \in X$, $a_n \neq 0$.
\end{proof}

\newpage
%\section{Infinite Sets}
\subsection{Countability}
Why $\setn$ being infinite implies $\setn - \{0\}$ being so?
\begin{proof}
\begin{align*}
&X \text{ being finite} \to X \cup \{x\} \text{ being finite} \tag{3.6.14(a)} \\
\equiv &X \cup \{x\} \text{ being infinite} \to X  \text{ being infinite}
\end{align*}
\end{proof}

\paragraph{Examples 8.1.3}
Why $f: n \mapsto 2n$ gives a bijection? Injectivity: $m \ne n \to 2m \ne 2n$; Surjectivity: $\forall n \in \setn, 2n \in f(\setn)$.

\declareexercise{8.1.1}
First we introduce a lemma.
\begin{lem}\label{lem.infseries.fromset}
Every infinite set contains a countable subset. In other words, countable sets are the smallest infinite sets.
\end{lem}
\begin{proof}
This is where the axiom of choice is required. The general idea is to select one element $x_0$ from the infinite set (say $X$), then select one element  $x_1$ from $X_1 = X \setminus \{x_0\}$, then select an element $x_2$ from $X_2 = X_1 \setminus \{x_1\}$, and so on...

I understand the requirement of the axiom of choice, since we are making infinite choices from sets of \emph{indistinguishable} elements. However, I still don't know how to apply the axiom to get the result, because every definition of $X_n$ requires the definition of the previous one.

That is, the entire collection of $X_n$ can't come in handy immediately...

But this seems to be too difficult for me to solve, so I will leave it unsolved here.
\end{proof}
\begin{proof}
A proper subset of a finite set $X$ cannot have the same cardinality as $X$. Thus, if a proper subset of $X$ has the same cardinality as $X$, then $X$ must be infinite.

We now show that every infinite set must have a proper set that has the same cardinality as it. According to lemma \ref{lem.infseries.fromset}, for an infinite set $X$, we have a subset 
$S = \{x_0,x_1,\cdots\} \subseteq X$. Now we define the function $f: X \to X \setminus \{x_0\}$ as follows:
\begin{align*}
n \in \setn &\to f(x_n) := x_{n+1} \\
x \notin S &\to f(x):=x
\end{align*}
It is easy to see that $f$ is a bijection. 
\end{proof}

\declareexercise{8.1.2}
Proof using induction:
\begin{proof}
Suppose that for $S \subseteq \setn$, there is no smallest element in $S$. We now prove that $S$ must be empty.

First, $0 \notin S$, otherwise, 0 would be the smallest element.

Now suppose that for some $N$, $\forall n \le N, n \not in S$, then $N+1$ must not be in $S$, otherwise it would be the smallest number.

We can now close the induction.
\end{proof}

Proof using the least upper bound property:
\begin{proof}
It is obvious that the nonempty set $S \subseteq \setn$ has a lower bound. For example, $0$ is one. Therefore, $\exists L \in \setr$ to be the greatest lower bound of $S$. We now show that $L$ is the smallest element of $S$.

$L$ can only be an integer. Otherwise, all real numbers between $L$ and $\ceiling{L}$ would be a lower bound bigger than $L$. In addition, $L \in S$, otherwise, $L+1$ would be a lower bound ($\forall x \in S, x>L \to x \ge L+1$). Therefore, $L$ is the smallest number.
\end{proof}

\declareexercise{8.1.3}
Gap Number
\begin{enumerate}
\item $X$ must be unbounded, for any $S \subseteq \setn$ bounded above by $M \in setn$ cannot have a cardinality bigger than $M+1$. Therefore $X\setminus\{a_m,m\le n\}$ is also unbounded, and is thus infinite (Remark 3.6.13).
\item Denote the set $\{x \in X: x \neq a_m \forall m < n\}$ as $X_n$, and we have $X_{n} \subseteq X_{n-1}$. Since that $a_n$ is the smallest element in $X_n$, all elements in $X_m$ for $m > n$ is bigger than $a_n$. Thus $a_{n-1} < a_n$ follows from there.
\item There is no equality in the previous relation.
\item This is nearly obvious because we are selecting elements from $X$ and its subsets.
\item $\forall n \forall m < n, a_n \ne x \to x \ne a_m$.
\item We have both $a_0 \ge 0$ and $a_n < a_{n+1} \to a_{n+1} \ge a_n + 1$. Thus $a_n > n$ can be easily shown with induction.
\item Otherwise $g$ could not be both bijective and increasing. If $g(m) \ne a_m$, then $g(m) > a_m$ since $a_m$ is the smallest element in the remaining set. $g$ is a bijection, so $a_m$ must be equaled by $g(n)$ with some $n > m$, a contradiction to the fact that $g$ is increasing.  
\end{enumerate}

\declareexercise{8.1.4}
\begin{proof}
It is obvious that when restricted to $A$, $f$ becomes injective. Now we show that $\forall x \in f(N), \exists n \in A, f(n) = x$. 

Suppose for sake of contradiction that there are elements $y \in f(N)$ such that $\forall x \in A, f(x) \ne y$. Note that we must also have $\exists n \in \setn \setminus A, f(n) = y$, which means the set $S := \setn \setminus A$ is non-empty. $S$ is also a subset of $\setn$, so it has a smallest element, and let's call it $m$. 

All numbers between 0 and $m$ thus become elements in $A$. By definition, $m$ cannot equal to any of them, that is, $\forall 0\le x \le m, f(x) \ne f(m)$. But this implies that $m$ is an element of $A$, a contradiction.

Therefore, $f: A \to f(N)$ is a bijection. $f(N)$ then is proved to be at most countable since $A$ is a subset of $\setn$.
\end{proof}

\declareexercise{8.1.5}
\begin{proof}
Since $X$ is countable, there is a bijection $g: \setn \to X$, which gives $X = g(\setn)$. So $f(X) = f(g(\setn)) = f \circ g (\setn)$. Therefore Corollary 8.1.9 follows from Proposition 8.1.8.
\end{proof}

\declareexercise{8.1.6}
\begin{proof}
If $A$ is finite, then there exists a bijection $f: A \to S = \{m\in \setn : 1 \le m \le \#A\}$. Since that $S$ is a subset of $\setn$, $f:A \to \setn$ is injective.

If $A$ is countable, then there is a bijection $f: A \to \setn$, which is itself injective.

We have proved that if $A$ is at most countable, then there is an injective function $f: A \to \setn$. 

On the other hand, if there is an injective map $f: A \to \setn$, then $f: A \to f(A)$ is a bijection. Since that $f(A)$ is a subset of $\setn$, it is at most countable, which implies that $A$ is at most countable.
\end{proof}

\declareexercise{8.1.7}
\begin{proof}
Since $f$ is bijective, for all $x \in X, \exists n \in \setn, f(n) = X$. But for all $n$, $h(2n) = f(n)$, so it means $h$ iterates every element of $X$. Similarly, $h$ iterates every element of $Y$. Thus, $X\cup Y \subseteq h(\setn)$. But by definition, $h(\setn) \subseteq X \cup Y$, so we must have $h(\setn) = X \cup Y$.

Both $X$ and $Y$ are infinite, so their union cannot be finite. According to Proposition 8.1.8, $X\cup Y$ thus can only be countable.
\end{proof}

\declareexercise{8.1.8}
\begin{proof}
Since that $X,Y$ are countable, there are two bijections: $f: \setn \to X, g: \setn \to Y$. Define $h:(m,n) \mapsto (f(m),g(n))$, and we can see that $h$ is a bijection from $\setn \times \setn$ to $X \times Y$.

Because $\setn \times \setn$ is countable, so is $X \times Y$.
\end{proof}

\declareexercise{8.1.9}
\begin{proof}
We might notice Corollary 8.1.13, which asserts that $\setn \times \setn$ is countable. So we may try to construct a bijection from $\setn \times \setn$ to $\bigcap_{\alpha \in I} A_\alpha$, which is plausible since that $\bigcap_{\alpha \in I}A_\alpha$ is \emph{like} $I \times A_{\alpha}$ (though not is).

For every $\alpha$, since that $A_\alpha$ is countable, there exists a bijection from $\setn$ to $A_\alpha$. However, the bijection may not be unique, so to select one bijection for each $\alpha$, we have to utilize the axiom of choice.

Now we have a sequence of bijections $\{f_n\}$. Define the function 
\[
g(m,n):=f_m(n);
\]
\end{proof}

\subsection{Summation on infinite sets}

\subsection{Uncountable sets}

\subsection{The axiom of choice}
\declareexercise{8.4.1}`
\begin{proof}
When $X = \varnothing$, $2^X = \{\varnothing\}$, whose cardinality is 1.
	
Let $Y = Y\setminus\{x_0\}\cup\{x_0\}$, where $\# Y = n+1 \wedge x_0 \in Y$. We have
\begin{align*}
&S \in 2^Y \\
&\equiv S \subseteq Y\\
&\equiv \forall x(x \in S \to x \in Y)\\
&\equiv \forall x\Big(x \notin S \vee \big(x \in S \wedge (x \in Y\setminus\{x\} \vee x = x_0)\big)\Big)
\end{align*}
For any $S$, $x_0 \in S$ is either true or false. If it is false, all $S$ that satisfy the above proposition are $S$ such that $\forall x (x \in S \to x \in Y\setminus\{x\})$, which means $S \in 2^{Y\setminus\{x\}}$. There are, by the induction hypothesis, $2^n$ distinct elements in $2^{Y\setminus\{x\}}$. If $x_0 \in S$, then $S$ can only be  
\end{proof}|

\subsection{Ordered sets}
\declareexercise{8.5.1}
It is both partially ordered and totally ordered, as the statement $\forall x \in \varnothing(P(x)) \equiv \forall x(x \in \varnothing \to P(x))$ is always a vacuous true statement.

But it is not well-ordered. It has only one subset, the $\varnothing$ itself, which obviously has no minimum element.

\declareexercise{8.5.2}

\declareexercise{8.5.8}
\begin{proof}
	We induct on the cardinality of the set. 
	
	Base case: if the set contains only one element $x$, then it is the maximum and minimum element.
	
	Inductive step: Assume that for a set of $n$ elements, the proposition is true. Let $S$ be a totally ordered set of $n+1$ elements, and let $x \in S$. Then $S = \{x\} \cup S \setminus \{x\}$. The cardinality of the set $\# (S \setminus \{x\}) = n$, so it has a maximum $M$ and a minimum element $m$ and $x \ne m \wedge x \ne M$.
	
	We must have $M \geq m$, otherwise we would have $M < m$, a contradiction. For $x$, we either have $x < m$, $m < x < M$ (if $m \neq M$), or $M < x$.
	
	On the first case, $x$ is the new minimal element ($\forall a \ne x, a \geq m > x$); on the second case, $m,M$ are still the minimal and maximal elements, respectively. on the third case, it is easy to prove that $x$ is the new maximal element. We can now close the induction.
\end{proof}

\declareexercise{8.5.12}
\begin{proof}
	Partial ordering: 
	\begin{enumerate}
		\item Reflexivity: if $(x,y) \leq_{X \times X} (x,y)$ because $x \leq x \wedge y \leq y$.
		\item Anti-symmetry: $(x,y) \leq (a,b) \wedge (a,b) \leq (x,y) \to x \leq a \wedge a \leq x \wedge y \leq b \wedge b \leq y \to x = y \wedge a = b$.
		\item Transitivity: $(x,y) \leq (a,b) \wedge (a,b) \leq (m,n) \to x \leq a \wedge a \leq m \wedge y \leq b \wedge b \leq n \to x\leq m \wedge y \leq n$.
	\end{enumerate}
	Total ordering:
\end{proof}




\newpage
%\section{Continuous Functions on $\setr$}
\subsection{Subsets of the real line}

\declareexercise{9.1.5}
\begin{description}
	\item[If] Suppose that, taking values entirely from $X$, there exists a sequence \taoseq{a_n} that converges to $x$, that is, it is eventually \eclose\ to every \pose. Now we show that $x$ is an adherent point of $X$, that is, $x$ is \eadherent\ to $X$ for every \pose.
	
	Given any \pose, we need to find $x_\varepsilon \in X$ such that $|x_\varepsilon - x| < \varepsilon$. Because \taoseq{a_n} converges to $x$, for this $\varepsilon$ we can find $N_\varepsilon > 0$ such that $|a_n - x| < \varepsilon$ for all $n \ge N$. Since \taoseq{a_n} takes value from $X$ entirely, any such $a_n$ suffices.
	
	\item[Only If] Now suppose that $x$ is an adherent point of $X$. Therefore, for every \pose, the set $X_\varepsilon = \{a \in X\colon |a-x| < \varepsilon\}$ is non-empty. 
	
	Consider the sequence $\taoseq{a_n}[1]\colon a_n \in X_{1/n}$. The existence of this sequence is obtained by using the axiom of choice. Then, it is easy to verify that \taoseq{a_n}[1] is indeed convergent to $x$.
\end{description}

\newpage
%% Copyright (C) He Guanyuming 2020
% The file is licensed under the MIT license.

\section{Mathematical Logic}

\subsection{Mathematical Statements}

\declareexercise{a.1.1}
It is ( (both $X, Y$ are false) or (both $X, Y$ are true) ).

\declareexercise{a.1.2}
It is ( ($Y$ can be true even if $X$ is false) or ($Y$ can be false even if $X$ is true) ).

\declareexercise{a.1.3}
Yes. That's the definition of logical equivalent.

\declareexercise{a.1.4}
No. It is still possible that (even if $X$ is false, $Y$ is still true).

Consider a statement $Y$ that satisfies:
\begin{enumerate}
\item If $X$, then $Y$.
\item If $X$ is false, then $Y$ or (exclusively) $Y$ is false.
\end{enumerate}

$X,Y$ satisfy the description in the exercise, but they are not logical equivalent.

\declareexercise{a.1.5}
Yes. (Now I'm using the symbols defined in the A.2 for the sake of simplification)
$X \Longleftrightarrow Y$ means $X \Longrightarrow Y \wedge \neg X \Longrightarrow \neg Y$. So does $Y$ and 
$Z$. So 
\begin{align*}
(X \Longrightarrow Y \Longrightarrow Z \wedge \neg X \Longrightarrow \neg Y \Longrightarrow \neg Z)
&\Longrightarrow \\
(X \Longrightarrow Z \wedge \neg X \Longrightarrow \neg Z)
\end{align*}
, which means $X$ and $Z$ are logical equivalent.

(Note that $A \Longrightarrow B$ can also be interpreted as a statement, meaning ``If $A$ is true, then $B$ 
is true'', just like we did in this example.)

\declareexercise{a.1.6}
Yes. $(X \Longrightarrow Y \Longrightarrow Z) \Longrightarrow (X \Longrightarrow Z)$. 

Now we are proving that 
$Z \Longrightarrow X \equiv \neg X \Longrightarrow \neg Z$. Assume that $\neg X \wedge Z$. Since 
$Z \Longrightarrow X$, we have a contradiction: $X \wedge \neg X$.

So $X \Longrightarrow Z \wedge \neg X \Longrightarrow \neg Z$. Therefore, $X,Z$ are logical equivalent. 
Besides, we can conclude that $Y \Longrightarrow X$. Thus $X,Y$ are also logical equivalent.

\subsection{Implication}
Why did Tao say
\begin{quotation}
If $X$, then $Y$ can also be written as ``$X$ can only be true when $Y$ is true''
\end{quotation}?

Assume the $X \wedge \neg Y$, but $X \Longrightarrow Y$. So we have a contradiction 
$Y \wedge \neg Y$.

Define ``when $x \neq 2$, $X:x=2 \Longrightarrow x^2=4$ is vacuously true'' to ensure that $X$ is 
always true regardless of the value of $x$.

\subsection{Nested Quantifiers}
\declareexercise{a.5.1}
\textbf{(a)} Let $P$ be $y^2=x$ is true for each positive number $y$. And this statement means $P$ is 
true for each positive number $x$. 

\emph{Gaming metaphor}: Me and my friend each randomly pick up a positive, say $x$ and $y$, and check 
if $y^2=x$.

The statement is false.

\textbf{(b)} There is at least one positive number $x$ such that for every positive number $y$, 
$y^2=x$.

\emph{Gaming metaphor}: I have to pick up a positive number $x$ such that whatever positive number $y$ 
my friend picks up, $y^2=x$ is always true.

The statement is false.

\textbf{(c)} There is at least two positive numbers $x,y$ such that $y^2=x$.

\emph{Gaming metaphor}: Me and my friend each have to pick up a positive number, say $x$ and $y$, such 
that $y^2=x$.

The statement is true. For example, $1^2=1$.

\textbf{(d)} The statement $\exists x > 0, y^2=x$ is true for every $y>0$.

\emph{Gaming metaphor}: For each positive number $y$ my friend picks up, I have to pick up a positive 
number $x$ such that $y^2=x$.

The statement is true, because $y^2$ is also positive.

\textbf{(e)} There is at least one positive number $y$ such that for every positive number $x$, $y^2=x$ 
is always true.

\emph{Gaming metaphor}: I have to find a number $y>0$ such that regardless of what number $x$ my friend 
picks up, $y^2=x$ is always true.

The statement is false.

\subsection{Equality}
\declareexercise{a.7.1}
\begin{proof}
Let $F(x) := x+c$. By axiom 4, $F(a)=F(b)$. That is, $a+c=b+c$. Similarly, by letting $G(x) := a+x$, 
we have $a+c=a+d$, which, according to axiom 2, becomes $a+d=a+c$. Now we have $a+d=a+c, a+c=b+c$. 
According to axiom 3, we can conclude that $a+d=b+c$.
\end{proof}

%\section{Metric spaces}
\declareexercise{ii.1.2.3}

%\begin{proof}
\paragraph{(a)}
If $E = \interior(E)$, then $E \cap \partial E = \varnothing$, then $E$ is open by definition. Conversely, if $E$ is open, then $E \cap \partial E = \varnothing$. But any set, including $E$, cannot contain its exterior points, so $E$ can only contain points that are neither boundary nor exterior to it, which means only interior points. Therefore, $E \subseteq \interior(E)$. Additionally, an interior point of $E$ must be in $E$, therefore, $\interior(E) \subseteq E$ and thus $E = \interior(E)$.

\paragraph{(b)}
Suppose $E$ is closed. By definition of interior points, $\interior(E) \subseteq E$. By definition of closed sets, $\partial E \subseteq E$. Therefore, $\interior(E) \cup \partial E \subseteq E$.
By definition of exterior points, $\exterior(E) \cap E = \varnothing$. If a point is not exterior, then it's either interior or boundary by definition, so $E \subseteq \interior(E) \cup \partial E$. Putting the two parts together, $E = \interior(E) \cup \partial E$. Proposition 1.2.10 (a) and (b) tell that the set of adherent points of $E$ equals $\interior(E) \cup \partial E$, so $E$ equals the set of its adherent points if $E$ is closed.

If $E = \interior(E) \cup \partial E$, then $E$ contains all of its boundary points and is closed by definition.

\paragraph{(c)}
I will use (a) to prove an open ball is open, so I need to show that $\forall x \in B(x_0, r), \exists p > 0, B(x,p) \subseteq B(x_0, r)$. It immediately follows from the triangular inequality. Let $p$ be any positive real number that is smaller than $r - d(x,x_0)$, then for any $y \in B(x,p)$, $d(y,x_0) \le d(y,x) + d(x,x_0) < r$, which means $B(x,p) \subseteq B(x_0, r)$ and thus the latter is open.

Now I will prove a closed ball is closed. I will use Proposition 1.2.10 (c). Suppose there exists a sequence $\taoseq{x}[1]$ in $E$ that converges to $y \notin E$. Therefore, $d(y,x_0) > r$. Let $q = d(x_0, y) - r$ and consider the ball $B(y, q)$. For any point $p$ in it, $d(p,y) < q$. Using the triangular inequality, we have
\begin{align*}
d(x_0, y)	&\le d(x_0, p) + d(p, y) \to \\
d(x_0, p)	&\ge d(x_0, y) - d(p, y) \\
			&>   d(x_0, y) - (q = d(x_0, y) - r) \\
			&=   r
\end{align*}
Thus, $B(y, q)$ is completely not contained within the closed ball. But as the sequence $\taoseqone{x}[1]$ converges to $y$, we can find $n \ge N$ such that $x_n \in B(y, q)$, which is both in $E$ and not in $E$, a contradiction. Therefore, all such sequences must converge inside $E$, and $E$ is closed.

\paragraph{(d)}
I will also use Proposition 1.2.10 (c) to prove it. The only possible sequence $\taoseqone{x}[1]$ that lies in a singleton set is a sequence where every $x_n = x$. That sequence obviously converges to $x$, and because a sequence cannot converge to two different elements, this means all sequences in $\{x\}$ converges to $x$, thus $\{x\}$ is closed.

\paragraph{(e)} I only need to prove $(E = \interior(E)) = \exterior(E^C)$, because what's left from $\exterior(E^C)$ are its interior and boundary points. It's quite easy to show, just note that $B(x, r) \subseteq E$ implies $B(x, r) \cap E^C = \varnothing$ by the definition of complements.

\paragraph{(f)} 
Let $x \in E_1 \cap \dots \cap E_n$, then $B(x, r_1) \subseteq E_1, B(x, r_2) \subseteq E_2, \dots, B(x, r_n) \subseteq E_n$. Let $R = \min(r_1, r_2, \dots, r_n)$, which always exists and is one of the $r$'s because the set is finite. Then $B(x, R) \subseteq E_1, E_2, \dots, E_n$, which means $B(x, R) \subseteq E_1 \cap E_2 \cap \dots \cap E_n$. Therefore, all points in the intersection are interior to it, so it's open. Note: \emph{An infinite intersection of open sets can be closed: intersecting $(0, \frac{1}{n})$ for $n$ from 1 to $\infty$ results in $\{0\}$, which is obviously closed.}

Note: \emph{Infinite union of closed sets can be open: let each point in any infinite open set (e.g.\ $(0,1)$) form a set, which is obviously closed, then their union is obviously open.}

\paragraph{(g)}
Suppose for the sake of contradiction that $\bigcup_{\alpha \in I} E_\alpha$ is not open, then $\partial \bigcup_{\alpha \in I}E_\alpha \cap \bigcup_{\alpha \in I}E_\alpha \ne \varnothing$. Let $x \in \partial \bigcup_{\alpha \in I}E_\alpha$, then $x \in E_a$ for some $a \in I$. Because $E_a$ is open, $x$ is interior in $E_a$, which means $B(x, r) \subset E_a$ for some $r > 0$, contradicting the fact that $x$ is the boundary of the union.

TBD the next part.
%\end{proof}

%\section{Uniform convergence}
\subsection{Limiting value of functions}
\declareexercise{ii.3.1.1}
\begin{description}
	\item[If] Suppose that $\lim_{x\to x_0;x\in E\setminus\{x_0\}} f(x)$ exists and equals to $f(x_0)$. This means that, for every $\varepsilon > 0$, there exists $\delta > 0$ such that $x \in \{x \in E\setminus\{x_0\} : d_X(x,x_0) < \delta\} \to d_Y(f(x),f(x_0)) < \varepsilon$.
	
	However, considering that $d_Y(f(x_0),f(x_0)) = 0 < \varepsilon$ for every $\varepsilon > 0$, we may conclude that $x \in \{x \in E : d_X(x,x_0) < \delta\} \to d_Y(f(x),f(x_0)) < \varepsilon$. By definition, this means that $\lim_{x\to x_0;x\in E} f(x) = f(x_0)$.
	
	\item[Only If] Suppose that $\lim_{x\to x_0;x\in E} f(x)$ exists. Say it equals to $L$. Then, for every $\varepsilon > 0$, there exists $\delta > 0$ such that $x \in \{x \in E : d_X(x,x_0) < \delta\} \to d_Y(f(x),L) < \varepsilon$. 
	
	In particular, since $d_X(x_0,x_0) = 0$, which is smaller than any $\delta > 0$, I conclude that $d_Y(f(x_0), L)$ is smaller than any $\varepsilon > 0$, which means it equals to $0$.
	
	Given any $\delta > 0$ and any $x \in \{x \in E\setminus\{x_0\} : d_X(x,x_0) < \delta\}$, consider the triangular in equality:
	$$
	d_Y(f(x),f(x_0)) \le d_Y(f(x),L) + d_Y(L, f(x_0)) = d_Y(f(x),L)
	$$
	Hence, if $d_Y(f(x),L) < \varepsilon$, then $d_Y(f(x),f(x_0)) < \varepsilon$. Because of the existence of $\lim_{x\to x_0;x\in E} f(x)$, we can always find $\delta > 0$ to get some $x$ to satisfy the $\varepsilon$. It means that $\lim_{x\to x_0;x\in E\setminus\{x_0\}} f(x) = f(x_0)$, as desired.
\end{description}

Note that $f(x_0) = L$ immediately follows from $d_Y(f(x_0),L) = 0$. Then, I have finished all the proofs.

One might be eager to arrive at the conclusion that the two definitions of limit are the same, as we are able to say now
\begin{quotation}
	$\lim_{x\to x_0;x\in E\setminus\{x_0\}} = f(x_0)$ if and only if $\lim_{x\to x_0;x\in E} = f(x_0)$.
\end{quotation}
However, it is not exactly the case, because when saying $\lim \cdots = L$, we are saying two things:
\begin{enumerate}
	\item $\lim \cdots$ exists.
	\item It equals to $L$.
\end{enumerate}
Hence, when $\lim_{x\to x_0;x\in E\setminus\{x_0\}} = L \ne f(x_0)$, $\lim_{x\to x_0;x\in E}$ does NOT exist!


%\foreach \x in {1,2,...,6}
{
\addanswer{2.2.\x}
}

\foreach \x in {1,2,...,5}
{
\addanswer{2.3.\x}
}

\foreach \x in {1,2,...,11}
{
\addanswer{3.1.\x}
}

\foreach \x in {1,2,3}
{
\addanswer{3.2.\x}
}

\foreach \x in {1,2,...,8}
{
\addanswer{3.3.\x}
}

\foreach \x in {1,2,...,11}
{
\addanswer{3.4.\x}
}

\foreach \x in {1,2,...,13}
{
\addanswer{3.5.\x}
}

\foreach \x in {1,2,...,10}
{
\addanswer{3.6.\x}
}

\foreach \x in {1,2,...,8}
{
\addanswer{4.1.\x}
}

\foreach \x in {1,2,...,6}
{
\addanswer{4.2.\x}
}

\foreach \x in {1,2,...,5}
{
\addanswer{4.3.\x}
}

\foreach \x in {1,2,...,3}
{
\addanswer{4.4.\x}
}

\foreach \x in {1}
{
\addanswer{5.1.\x}
}

\foreach \x in {1,2}
{
\addanswer{5.2.\x}
}

\foreach \x in {1,2,...,5}
{
\addanswer{5.3.\x}
}

\foreach \x in {1,2,...,8}
{
\addanswer{5.4.\x}
}

\foreach \x in {1,2,...,6}
{
\addanswer{6.1.\x}
}

\addanswer{7.1.1}
\addanswer{7.1.2}

\foreach \x in {1,2,...,6}
{
\addanswer{7.2.\x}
}

\foreach \x in {1,2,...,3}
{
\addanswer{7.3.\x}
}

\foreach \x in {1,2,...,8}
{
\addanswer{8.1.\x}
}

\foreach \x in {1,2,...,6}
{
\addanswer{a.1.\x}
}

\addanswer{a.5.1}

\addanswer{a.7.1}

\end{document}