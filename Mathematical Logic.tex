% Copyright (C) He Guanyuming 2020
% The file is licensed under the MIT license.

\section{Mathematical Logic}

\subsection{Mathematical Statements}

\declareexercise{a.1.1}
It is ( (both $X, Y$ are false) or (both $X, Y$ are true) ).

\declareexercise{a.1.2}
It is ( ($Y$ can be true even if $X$ is false) or ($Y$ can be false even if $X$ is true) ).

\declareexercise{a.1.3}
Yes. That's the definition of logical equivalent.

\declareexercise{a.1.4}
No. It is still possible that (even if $X$ is false, $Y$ is still true).

Consider a statement $Y$ that satisfies:
\begin{enumerate}
\item If $X$, then $Y$.
\item If $X$ is false, then $Y$ or (exclusively) $Y$ is false.
\end{enumerate}

$X,Y$ satisfy the description in the exercise, but they are not logical equivalent.

\declareexercise{a.1.5}
Yes. (Now I'm using the symbols defined in the A.2 for the sake of simplification)
$X \Longleftrightarrow Y$ means $X \Longrightarrow Y \wedge \neg X \Longrightarrow \neg Y$. So does $Y$ and 
$Z$. So 
\begin{align*}
(X \Longrightarrow Y \Longrightarrow Z \wedge \neg X \Longrightarrow \neg Y \Longrightarrow \neg Z)
&\Longrightarrow \\
(X \Longrightarrow Z \wedge \neg X \Longrightarrow \neg Z)
\end{align*}
, which means $X$ and $Z$ are logical equivalent.

(Note that $A \Longrightarrow B$ can also be interpreted as a statement, meaning ``If $A$ is true, then $B$ 
is true'', just like we did in this example.)

\declareexercise{a.1.6}
Yes. $(X \Longrightarrow Y \Longrightarrow Z) \Longrightarrow (X \Longrightarrow Z)$. 

Now we are proving that 
$Z \Longrightarrow X \equiv \neg X \Longrightarrow \neg Z$. Assume that $\neg X \wedge Z$. Since 
$Z \Longrightarrow X$, we have a contradiction: $X \wedge \neg X$.

So $X \Longrightarrow Z \wedge \neg X \Longrightarrow \neg Z$. Therefore, $X,Z$ are logical equivalent. 
Besides, we can conclude that $Y \Longrightarrow X$. Thus $X,Y$ are also logical equivalent.

\subsection{Implication}
Why did Tao say
\begin{quotation}
If $X$, then $Y$ can also be written as ``$X$ can only be true when $Y$ is true''
\end{quotation}?

Assume the $X \wedge \neg Y$, but $X \Longrightarrow Y$. So we have a contradiction 
$Y \wedge \neg Y$.

Define ``when $x \neq 2$, $X:x=2 \Longrightarrow x^2=4$ is vacuously true'' to ensure that $X$ is 
always true regardless of the value of $x$.

\subsection{Nested Quantifiers}
\declareexercise{a.5.1}
\textbf{(a)} Let $P$ be $y^2=x$ is true for each positive number $y$. And this statement means $P$ is 
true for each positive number $x$. 

\emph{Gaming metaphor}: Me and my friend each randomly pick up a positive, say $x$ and $y$, and check 
if $y^2=x$.

The statement is false.

\textbf{(b)} There is at least one positive number $x$ such that for every positive number $y$, 
$y^2=x$.

\emph{Gaming metaphor}: I have to pick up a positive number $x$ such that whatever positive number $y$ 
my friend picks up, $y^2=x$ is always true.

The statement is false.

\textbf{(c)} There is at least two positive numbers $x,y$ such that $y^2=x$.

\emph{Gaming metaphor}: Me and my friend each have to pick up a positive number, say $x$ and $y$, such 
that $y^2=x$.

The statement is true. For example, $1^2=1$.

\textbf{(d)} The statement $\exists x > 0, y^2=x$ is true for every $y>0$.

\emph{Gaming metaphor}: For each positive number $y$ my friend picks up, I have to pick up a positive 
number $x$ such that $y^2=x$.

The statement is true, because $y^2$ is also positive.

\textbf{(e)} There is at least one positive number $y$ such that for every positive number $x$, $y^2=x$ 
is always true.

\emph{Gaming metaphor}: I have to find a number $y>0$ such that regardless of what number $x$ my friend 
picks up, $y^2=x$ is always true.

The statement is false.

\subsection{Equality}
\declareexercise{a.7.1}
\begin{proof}
Let $F(x) := x+c$. By axiom 4, $F(a)=F(b)$. That is, $a+c=b+c$. Similarly, by letting $G(x) := a+x$, 
we have $a+c=a+d$, which, according to axiom 2, becomes $a+d=a+c$. Now we have $a+d=a+c, a+c=b+c$. 
According to axiom 3, we can conclude that $a+d=b+c$.
\end{proof}