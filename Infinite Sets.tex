\section{Infinite Sets}
\subsection{Countability}
Why $\setn$ being infinite implies $\setn - \{0\}$ being so?
\begin{proof}
\begin{align*}
&X \text{ being finite} \to X \cup \{x\} \text{ being finite} \tag{3.6.14(a)} \\
\equiv &X \cup \{x\} \text{ being infinite} \to X  \text{ being infinite}
\end{align*}
\end{proof}

\paragraph{Examples 8.1.3}
Why $f: n \mapsto 2n$ gives a bijection? Injectivity: $m \ne n \to 2m \ne 2n$; Surjectivity: $\forall n \in \setn, 2n \in f(\setn)$.

\declareexercise{8.1.1}
First we introduce a lemma.
\begin{lem}\label{lem.infseries.fromset}
Every infinite set contains a countable subset. In other words, countable sets are the smallest infinite sets.
\end{lem}
\begin{proof}
This is where the axiom of choice is required. The general idea is to select one element $x_0$ from the infinite set (say $X$), then select one element  $x_1$ from $X_1 = X \setminus \{x_0\}$, then select an element $x_2$ from $X_2 = X_1 \setminus \{x_1\}$, and so on...

I understand the requirement of the axiom of choice, since we are making infinite choices from sets of \emph{indistinguishable} elements. However, I still don't know how to apply the axiom to get the result, because every definition of $X_n$ requires the definition of the previous one.

That is, the entire collection of $X_n$ can't come in handy immediately...

But this seems to be too difficult for me to solve, so I will leave it unsolved here.
\end{proof}
\begin{proof}
A proper subset of a finite set $X$ cannot have the same cardinality as $X$. Thus, if a proper subset of $X$ has the same cardinality as $X$, then $X$ must be infinite.

We now show that every infinite set must have a proper set that has the same cardinality as it. According to lemma \ref{lem.infseries.fromset}, for an infinite set $X$, we have a subset 
$S = \{x_0,x_1,\cdots\} \subseteq X$. Now we define the function $f: X \to X \setminus \{x_0\}$ as follows:
\begin{align*}
n \in \setn &\to f(x_n) := x_{n+1} \\
x \notin S &\to f(x):=x
\end{align*}
It is easy to see that $f$ is a bijection. 
\end{proof}

\declareexercise{8.1.2}
Proof using induction:
\begin{proof}
Suppose that for $S \subseteq \setn$, there is no smallest element in $S$. We now prove that $S$ must be empty.

First, $0 \notin S$, otherwise, 0 would be the smallest element.

Now suppose that for some $N$, $\forall n \le N, n \not in S$, then $N+1$ must not be in $S$, otherwise it would be the smallest number.

We can now close the induction.
\end{proof}

Proof using the least upper bound property:
\begin{proof}
It is obvious that the nonempty set $S \subseteq \setn$ has a lower bound. For example, $0$ is one. Therefore, $\exists L \in \setr$ to be the greatest lower bound of $S$. We now show that $L$ is the smallest element of $S$.

$L$ can only be an integer. Otherwise, all real numbers between $L$ and $\ceiling{L}$ would be a lower bound bigger than $L$. In addition, $L \in S$, otherwise, $L+1$ would be a lower bound ($\forall x \in S, x>L \to x \ge L+1$). Therefore, $L$ is the smallest number.
\end{proof}

\declareexercise{8.1.3}
Gap Number
\begin{enumerate}
\item $X$ must be unbounded, for any $S \subseteq \setn$ bounded above by $M \in setn$ cannot have a cardinality bigger than $M+1$. Therefore $X\setminus\{a_m,m\le n\}$ is also unbounded, and is thus infinite (Remark 3.6.13).
\item Denote the set $\{x \in X: x \neq a_m \forall m < n\}$ as $X_n$, and we have $X_{n} \subseteq X_{n-1}$. Since that $a_n$ is the smallest element in $X_n$, all elements in $X_m$ for $m > n$ is bigger than $a_n$. Thus $a_{n-1} < a_n$ follows from there.
\item There is no equality in the previous relation.
\item This is nearly obvious because we are selecting elements from $X$ and its subsets.
\item $\forall n \forall m < n, a_n \ne x \to x \ne a_m$.
\item We have both $a_0 \ge 0$ and $a_n < a_{n+1} \to a_{n+1} \ge a_n + 1$. Thus $a_n > n$ can be easily shown with induction.
\item Otherwise $g$ could not be both bijective and increasing. If $g(m) \ne a_m$, then $g(m) > a_m$ since $a_m$ is the smallest element in the remaining set. $g$ is a bijection, so $a_m$ must be equaled by $g(n)$ with some $n > m$, a contradiction to the fact that $g$ is increasing.  
\end{enumerate}

\declareexercise{8.1.4}
\begin{proof}
It is obvious that when restricted to $A$, $f$ becomes injective. Now we show that $\forall x \in f(N), \exists n \in A, f(n) = x$. 

Suppose for sake of contradiction that there are elements $y \in f(N)$ such that $\forall x \in A, f(x) \ne y$. Note that we must also have $\exists n \in \setn \setminus A, f(n) = y$, which means the set $S := \setn \setminus A$ is non-empty. $S$ is also a subset of $\setn$, so it has a smallest element, and let's call it $m$. 

All numbers between 0 and $m$ thus become elements in $A$. By definition, $m$ cannot equal to any of them, that is, $\forall 0\le x \le m, f(x) \ne f(m)$. But this implies that $m$ is an element of $A$, a contradiction.

Therefore, $f: A \to f(N)$ is a bijection. $f(N)$ then is proved to be at most countable since $A$ is a subset of $\setn$.
\end{proof}

\declareexercise{8.1.5}
\begin{proof}
Since $X$ is countable, there is a bijection $g: \setn \to X$, which gives $X = g(\setn)$. So $f(X) = f(g(\setn)) = f \circ g (\setn)$. Therefore Corollary 8.1.9 follows from Proposition 8.1.8.
\end{proof}

\declareexercise{8.1.6}
\begin{proof}
If $A$ is finite, then there exists a bijection $f: A \to S = \{m\in \setn : 1 \le m \le \#A\}$. Since that $S$ is a subset of $\setn$, $f:A \to \setn$ is injective.

If $A$ is countable, then there is a bijection $f: A \to \setn$, which is itself injective.

We have proved that if $A$ is at most countable, then there is an injective function $f: A \to \setn$. 

On the other hand, if there is an injective map $f: A \to \setn$, then $f: A \to f(A)$ is a bijection. Since that $f(A)$ is a subset of $\setn$, it is at most countable, which implies that $A$ is at most countable.
\end{proof}

\declareexercise{8.1.7}
\begin{proof}
Since $f$ is bijective, for all $x \in X, \exists n \in \setn, f(n) = X$. But for all $n$, $h(2n) = f(n)$, so it means $h$ iterates every element of $X$. Similarly, $h$ iterates every element of $Y$. Thus, $X\cup Y \subseteq h(\setn)$. But by definition, $h(\setn) \subseteq X \cup Y$, so we must have $h(\setn) = X \cup Y$.

Both $X$ and $Y$ are infinite, so their union cannot be finite. According to Proposition 8.1.8, $X\cup Y$ thus can only be countable.
\end{proof}

\declareexercise{8.1.8}
\begin{proof}
Since that $X,Y$ are countable, there are two bijections: $f: \setn \to X, g: \setn \to Y$. Define $h:(m,n) \mapsto (f(m),g(n))$, and we can see that $h$ is a bijection from $\setn \times \setn$ to $X \times Y$.

Because $\setn \times \setn$ is countable, so is $X \times Y$.
\end{proof}

\declareexercise{8.1.9}
\begin{proof}
We might notice Corollary 8.1.13, which asserts that $\setn \times \setn$ is countable. So we may try to construct a bijection from $\setn \times \setn$ to $\bigcap_{\alpha \in I} A_\alpha$, which is plausible since that $\bigcap_{\alpha \in I}A_\alpha$ is \emph{like} $I \times A_{\alpha}$ (though not is).

For every $\alpha$, since that $A_\alpha$ is countable, there exists a bijection from $\setn$ to $A_\alpha$. However, the bijection may not be unique, so to select one bijection for each $\alpha$, we have to utilize the axiom of choice.

Now we have a sequence of bijections $\{f_n\}$. Define the function 
\[
g(m,n):=f_m(n);
\]
\end{proof}

\subsection{Summation on infinite sets}

\subsection{Uncountable sets}

\subsection{The axiom of choice}
\declareexercise{8.4.1}`
\begin{proof}
When $X = \varnothing$, $2^X = \{\varnothing\}$, whose cardinality is 1.
	
Let $Y = Y\setminus\{x_0\}\cup\{x_0\}$, where $\# Y = n+1 \wedge x_0 \in Y$. We have
\begin{align*}
&S \in 2^Y \\
&\equiv S \subseteq Y\\
&\equiv \forall x(x \in S \to x \in Y)\\
&\equiv \forall x\Big(x \notin S \vee \big(x \in S \wedge (x \in Y\setminus\{x\} \vee x = x_0)\big)\Big)
\end{align*}
For any $S$, $x_0 \in S$ is either true or false. If it is false, all $S$ that satisfy the above proposition are $S$ such that $\forall x (x \in S \to x \in Y\setminus\{x\})$, which means $S \in 2^{Y\setminus\{x\}}$. There are, by the induction hypothesis, $2^n$ distinct elements in $2^{Y\setminus\{x\}}$. If $x_0 \in S$, then $S$ can only be  
\end{proof}|

\subsection{Ordered sets}
\declareexercise{8.5.1}
It is both partially ordered and totally ordered, as the statement $\forall x \in \varnothing(P(x)) \equiv \forall x(x \in \varnothing \to P(x))$ is always a vacuous true statement.

But it is not well-ordered. It has only one subset, the $\varnothing$ itself, which obviously has no minimum element.

\declareexercise{8.5.2}

\declareexercise{8.5.8}
\begin{proof}
	We induct on the cardinality of the set. 
	
	Base case: if the set contains only one element $x$, then it is the maximum and minimum element.
	
	Inductive step: Assume that for a set of $n$ elements, the proposition is true. Let $S$ be a totally ordered set of $n+1$ elements, and let $x \in S$. Then $S = \{x\} \cup S \setminus \{x\}$. The cardinality of the set $\# (S \setminus \{x\}) = n$, so it has a maximum $M$ and a minimum element $m$ and $x \ne m \wedge x \ne M$.
	
	We must have $M \geq m$, otherwise we would have $M < m$, a contradiction. For $x$, we either have $x < m$, $m < x < M$ (if $m \neq M$), or $M < x$.
	
	On the first case, $x$ is the new minimal element ($\forall a \ne x, a \geq m > x$); on the second case, $m,M$ are still the minimal and maximal elements, respectively. on the third case, it is easy to prove that $x$ is the new maximal element. We can now close the induction.
\end{proof}

\declareexercise{8.5.12}
\begin{proof}
	Partial ordering: 
	\begin{enumerate}
		\item Reflexivity: if $(x,y) \leq_{X \times X} (x,y)$ because $x \leq x \wedge y \leq y$.
		\item Anti-symmetry: $(x,y) \leq (a,b) \wedge (a,b) \leq (x,y) \to x \leq a \wedge a \leq x \wedge y \leq b \wedge b \leq y \to x = y \wedge a = b$.
		\item Transitivity: $(x,y) \leq (a,b) \wedge (a,b) \leq (m,n) \to x \leq a \wedge a \leq m \wedge y \leq b \wedge b \leq n \to x\leq m \wedge y \leq n$.
	\end{enumerate}
	Total ordering:
\end{proof}


